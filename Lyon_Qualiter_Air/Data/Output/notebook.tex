
% Default to the notebook output style

    


% Inherit from the specified cell style.




    
\documentclass[11pt]{article}

    
    
    \usepackage[T1]{fontenc}
    % Nicer default font (+ math font) than Computer Modern for most use cases
    \usepackage{mathpazo}

    % Basic figure setup, for now with no caption control since it's done
    % automatically by Pandoc (which extracts ![](path) syntax from Markdown).
    \usepackage{graphicx}
    % We will generate all images so they have a width \maxwidth. This means
    % that they will get their normal width if they fit onto the page, but
    % are scaled down if they would overflow the margins.
    \makeatletter
    \def\maxwidth{\ifdim\Gin@nat@width>\linewidth\linewidth
    \else\Gin@nat@width\fi}
    \makeatother
    \let\Oldincludegraphics\includegraphics
    % Set max figure width to be 80% of text width, for now hardcoded.
    \renewcommand{\includegraphics}[1]{\Oldincludegraphics[width=.8\maxwidth]{#1}}
    % Ensure that by default, figures have no caption (until we provide a
    % proper Figure object with a Caption API and a way to capture that
    % in the conversion process - todo).
    \usepackage{caption}
    \DeclareCaptionLabelFormat{nolabel}{}
    \captionsetup{labelformat=nolabel}

    \usepackage{adjustbox} % Used to constrain images to a maximum size 
    \usepackage{xcolor} % Allow colors to be defined
    \usepackage{enumerate} % Needed for markdown enumerations to work
    \usepackage{geometry} % Used to adjust the document margins
    \usepackage{amsmath} % Equations
    \usepackage{amssymb} % Equations
    \usepackage{textcomp} % defines textquotesingle
    % Hack from http://tex.stackexchange.com/a/47451/13684:
    \AtBeginDocument{%
        \def\PYZsq{\textquotesingle}% Upright quotes in Pygmentized code
    }
    \usepackage{upquote} % Upright quotes for verbatim code
    \usepackage{eurosym} % defines \euro
    \usepackage[mathletters]{ucs} % Extended unicode (utf-8) support
    \usepackage[utf8x]{inputenc} % Allow utf-8 characters in the tex document
    \usepackage{fancyvrb} % verbatim replacement that allows latex
    \usepackage{grffile} % extends the file name processing of package graphics 
                         % to support a larger range 
    % The hyperref package gives us a pdf with properly built
    % internal navigation ('pdf bookmarks' for the table of contents,
    % internal cross-reference links, web links for URLs, etc.)
    \usepackage{hyperref}
    \usepackage{longtable} % longtable support required by pandoc >1.10
    \usepackage{booktabs}  % table support for pandoc > 1.12.2
    \usepackage[inline]{enumitem} % IRkernel/repr support (it uses the enumerate* environment)
    \usepackage[normalem]{ulem} % ulem is needed to support strikethroughs (\sout)
                                % normalem makes italics be italics, not underlines
    

    
    
    % Colors for the hyperref package
    \definecolor{urlcolor}{rgb}{0,.145,.698}
    \definecolor{linkcolor}{rgb}{.71,0.21,0.01}
    \definecolor{citecolor}{rgb}{.12,.54,.11}

    % ANSI colors
    \definecolor{ansi-black}{HTML}{3E424D}
    \definecolor{ansi-black-intense}{HTML}{282C36}
    \definecolor{ansi-red}{HTML}{E75C58}
    \definecolor{ansi-red-intense}{HTML}{B22B31}
    \definecolor{ansi-green}{HTML}{00A250}
    \definecolor{ansi-green-intense}{HTML}{007427}
    \definecolor{ansi-yellow}{HTML}{DDB62B}
    \definecolor{ansi-yellow-intense}{HTML}{B27D12}
    \definecolor{ansi-blue}{HTML}{208FFB}
    \definecolor{ansi-blue-intense}{HTML}{0065CA}
    \definecolor{ansi-magenta}{HTML}{D160C4}
    \definecolor{ansi-magenta-intense}{HTML}{A03196}
    \definecolor{ansi-cyan}{HTML}{60C6C8}
    \definecolor{ansi-cyan-intense}{HTML}{258F8F}
    \definecolor{ansi-white}{HTML}{C5C1B4}
    \definecolor{ansi-white-intense}{HTML}{A1A6B2}

    % commands and environments needed by pandoc snippets
    % extracted from the output of `pandoc -s`
    \providecommand{\tightlist}{%
      \setlength{\itemsep}{0pt}\setlength{\parskip}{0pt}}
    \DefineVerbatimEnvironment{Highlighting}{Verbatim}{commandchars=\\\{\}}
    % Add ',fontsize=\small' for more characters per line
    \newenvironment{Shaded}{}{}
    \newcommand{\KeywordTok}[1]{\textcolor[rgb]{0.00,0.44,0.13}{\textbf{{#1}}}}
    \newcommand{\DataTypeTok}[1]{\textcolor[rgb]{0.56,0.13,0.00}{{#1}}}
    \newcommand{\DecValTok}[1]{\textcolor[rgb]{0.25,0.63,0.44}{{#1}}}
    \newcommand{\BaseNTok}[1]{\textcolor[rgb]{0.25,0.63,0.44}{{#1}}}
    \newcommand{\FloatTok}[1]{\textcolor[rgb]{0.25,0.63,0.44}{{#1}}}
    \newcommand{\CharTok}[1]{\textcolor[rgb]{0.25,0.44,0.63}{{#1}}}
    \newcommand{\StringTok}[1]{\textcolor[rgb]{0.25,0.44,0.63}{{#1}}}
    \newcommand{\CommentTok}[1]{\textcolor[rgb]{0.38,0.63,0.69}{\textit{{#1}}}}
    \newcommand{\OtherTok}[1]{\textcolor[rgb]{0.00,0.44,0.13}{{#1}}}
    \newcommand{\AlertTok}[1]{\textcolor[rgb]{1.00,0.00,0.00}{\textbf{{#1}}}}
    \newcommand{\FunctionTok}[1]{\textcolor[rgb]{0.02,0.16,0.49}{{#1}}}
    \newcommand{\RegionMarkerTok}[1]{{#1}}
    \newcommand{\ErrorTok}[1]{\textcolor[rgb]{1.00,0.00,0.00}{\textbf{{#1}}}}
    \newcommand{\NormalTok}[1]{{#1}}
    
    % Additional commands for more recent versions of Pandoc
    \newcommand{\ConstantTok}[1]{\textcolor[rgb]{0.53,0.00,0.00}{{#1}}}
    \newcommand{\SpecialCharTok}[1]{\textcolor[rgb]{0.25,0.44,0.63}{{#1}}}
    \newcommand{\VerbatimStringTok}[1]{\textcolor[rgb]{0.25,0.44,0.63}{{#1}}}
    \newcommand{\SpecialStringTok}[1]{\textcolor[rgb]{0.73,0.40,0.53}{{#1}}}
    \newcommand{\ImportTok}[1]{{#1}}
    \newcommand{\DocumentationTok}[1]{\textcolor[rgb]{0.73,0.13,0.13}{\textit{{#1}}}}
    \newcommand{\AnnotationTok}[1]{\textcolor[rgb]{0.38,0.63,0.69}{\textbf{\textit{{#1}}}}}
    \newcommand{\CommentVarTok}[1]{\textcolor[rgb]{0.38,0.63,0.69}{\textbf{\textit{{#1}}}}}
    \newcommand{\VariableTok}[1]{\textcolor[rgb]{0.10,0.09,0.49}{{#1}}}
    \newcommand{\ControlFlowTok}[1]{\textcolor[rgb]{0.00,0.44,0.13}{\textbf{{#1}}}}
    \newcommand{\OperatorTok}[1]{\textcolor[rgb]{0.40,0.40,0.40}{{#1}}}
    \newcommand{\BuiltInTok}[1]{{#1}}
    \newcommand{\ExtensionTok}[1]{{#1}}
    \newcommand{\PreprocessorTok}[1]{\textcolor[rgb]{0.74,0.48,0.00}{{#1}}}
    \newcommand{\AttributeTok}[1]{\textcolor[rgb]{0.49,0.56,0.16}{{#1}}}
    \newcommand{\InformationTok}[1]{\textcolor[rgb]{0.38,0.63,0.69}{\textbf{\textit{{#1}}}}}
    \newcommand{\WarningTok}[1]{\textcolor[rgb]{0.38,0.63,0.69}{\textbf{\textit{{#1}}}}}
    
    
    % Define a nice break command that doesn't care if a line doesn't already
    % exist.
    \def\br{\hspace*{\fill} \\* }
    % Math Jax compatability definitions
    \def\gt{>}
    \def\lt{<}
    % Document parameters
    \title{Lyon\_Air\_Quality}
    
    
    

    % Pygments definitions
    
\makeatletter
\def\PY@reset{\let\PY@it=\relax \let\PY@bf=\relax%
    \let\PY@ul=\relax \let\PY@tc=\relax%
    \let\PY@bc=\relax \let\PY@ff=\relax}
\def\PY@tok#1{\csname PY@tok@#1\endcsname}
\def\PY@toks#1+{\ifx\relax#1\empty\else%
    \PY@tok{#1}\expandafter\PY@toks\fi}
\def\PY@do#1{\PY@bc{\PY@tc{\PY@ul{%
    \PY@it{\PY@bf{\PY@ff{#1}}}}}}}
\def\PY#1#2{\PY@reset\PY@toks#1+\relax+\PY@do{#2}}

\expandafter\def\csname PY@tok@w\endcsname{\def\PY@tc##1{\textcolor[rgb]{0.73,0.73,0.73}{##1}}}
\expandafter\def\csname PY@tok@c\endcsname{\let\PY@it=\textit\def\PY@tc##1{\textcolor[rgb]{0.25,0.50,0.50}{##1}}}
\expandafter\def\csname PY@tok@cp\endcsname{\def\PY@tc##1{\textcolor[rgb]{0.74,0.48,0.00}{##1}}}
\expandafter\def\csname PY@tok@k\endcsname{\let\PY@bf=\textbf\def\PY@tc##1{\textcolor[rgb]{0.00,0.50,0.00}{##1}}}
\expandafter\def\csname PY@tok@kp\endcsname{\def\PY@tc##1{\textcolor[rgb]{0.00,0.50,0.00}{##1}}}
\expandafter\def\csname PY@tok@kt\endcsname{\def\PY@tc##1{\textcolor[rgb]{0.69,0.00,0.25}{##1}}}
\expandafter\def\csname PY@tok@o\endcsname{\def\PY@tc##1{\textcolor[rgb]{0.40,0.40,0.40}{##1}}}
\expandafter\def\csname PY@tok@ow\endcsname{\let\PY@bf=\textbf\def\PY@tc##1{\textcolor[rgb]{0.67,0.13,1.00}{##1}}}
\expandafter\def\csname PY@tok@nb\endcsname{\def\PY@tc##1{\textcolor[rgb]{0.00,0.50,0.00}{##1}}}
\expandafter\def\csname PY@tok@nf\endcsname{\def\PY@tc##1{\textcolor[rgb]{0.00,0.00,1.00}{##1}}}
\expandafter\def\csname PY@tok@nc\endcsname{\let\PY@bf=\textbf\def\PY@tc##1{\textcolor[rgb]{0.00,0.00,1.00}{##1}}}
\expandafter\def\csname PY@tok@nn\endcsname{\let\PY@bf=\textbf\def\PY@tc##1{\textcolor[rgb]{0.00,0.00,1.00}{##1}}}
\expandafter\def\csname PY@tok@ne\endcsname{\let\PY@bf=\textbf\def\PY@tc##1{\textcolor[rgb]{0.82,0.25,0.23}{##1}}}
\expandafter\def\csname PY@tok@nv\endcsname{\def\PY@tc##1{\textcolor[rgb]{0.10,0.09,0.49}{##1}}}
\expandafter\def\csname PY@tok@no\endcsname{\def\PY@tc##1{\textcolor[rgb]{0.53,0.00,0.00}{##1}}}
\expandafter\def\csname PY@tok@nl\endcsname{\def\PY@tc##1{\textcolor[rgb]{0.63,0.63,0.00}{##1}}}
\expandafter\def\csname PY@tok@ni\endcsname{\let\PY@bf=\textbf\def\PY@tc##1{\textcolor[rgb]{0.60,0.60,0.60}{##1}}}
\expandafter\def\csname PY@tok@na\endcsname{\def\PY@tc##1{\textcolor[rgb]{0.49,0.56,0.16}{##1}}}
\expandafter\def\csname PY@tok@nt\endcsname{\let\PY@bf=\textbf\def\PY@tc##1{\textcolor[rgb]{0.00,0.50,0.00}{##1}}}
\expandafter\def\csname PY@tok@nd\endcsname{\def\PY@tc##1{\textcolor[rgb]{0.67,0.13,1.00}{##1}}}
\expandafter\def\csname PY@tok@s\endcsname{\def\PY@tc##1{\textcolor[rgb]{0.73,0.13,0.13}{##1}}}
\expandafter\def\csname PY@tok@sd\endcsname{\let\PY@it=\textit\def\PY@tc##1{\textcolor[rgb]{0.73,0.13,0.13}{##1}}}
\expandafter\def\csname PY@tok@si\endcsname{\let\PY@bf=\textbf\def\PY@tc##1{\textcolor[rgb]{0.73,0.40,0.53}{##1}}}
\expandafter\def\csname PY@tok@se\endcsname{\let\PY@bf=\textbf\def\PY@tc##1{\textcolor[rgb]{0.73,0.40,0.13}{##1}}}
\expandafter\def\csname PY@tok@sr\endcsname{\def\PY@tc##1{\textcolor[rgb]{0.73,0.40,0.53}{##1}}}
\expandafter\def\csname PY@tok@ss\endcsname{\def\PY@tc##1{\textcolor[rgb]{0.10,0.09,0.49}{##1}}}
\expandafter\def\csname PY@tok@sx\endcsname{\def\PY@tc##1{\textcolor[rgb]{0.00,0.50,0.00}{##1}}}
\expandafter\def\csname PY@tok@m\endcsname{\def\PY@tc##1{\textcolor[rgb]{0.40,0.40,0.40}{##1}}}
\expandafter\def\csname PY@tok@gh\endcsname{\let\PY@bf=\textbf\def\PY@tc##1{\textcolor[rgb]{0.00,0.00,0.50}{##1}}}
\expandafter\def\csname PY@tok@gu\endcsname{\let\PY@bf=\textbf\def\PY@tc##1{\textcolor[rgb]{0.50,0.00,0.50}{##1}}}
\expandafter\def\csname PY@tok@gd\endcsname{\def\PY@tc##1{\textcolor[rgb]{0.63,0.00,0.00}{##1}}}
\expandafter\def\csname PY@tok@gi\endcsname{\def\PY@tc##1{\textcolor[rgb]{0.00,0.63,0.00}{##1}}}
\expandafter\def\csname PY@tok@gr\endcsname{\def\PY@tc##1{\textcolor[rgb]{1.00,0.00,0.00}{##1}}}
\expandafter\def\csname PY@tok@ge\endcsname{\let\PY@it=\textit}
\expandafter\def\csname PY@tok@gs\endcsname{\let\PY@bf=\textbf}
\expandafter\def\csname PY@tok@gp\endcsname{\let\PY@bf=\textbf\def\PY@tc##1{\textcolor[rgb]{0.00,0.00,0.50}{##1}}}
\expandafter\def\csname PY@tok@go\endcsname{\def\PY@tc##1{\textcolor[rgb]{0.53,0.53,0.53}{##1}}}
\expandafter\def\csname PY@tok@gt\endcsname{\def\PY@tc##1{\textcolor[rgb]{0.00,0.27,0.87}{##1}}}
\expandafter\def\csname PY@tok@err\endcsname{\def\PY@bc##1{\setlength{\fboxsep}{0pt}\fcolorbox[rgb]{1.00,0.00,0.00}{1,1,1}{\strut ##1}}}
\expandafter\def\csname PY@tok@kc\endcsname{\let\PY@bf=\textbf\def\PY@tc##1{\textcolor[rgb]{0.00,0.50,0.00}{##1}}}
\expandafter\def\csname PY@tok@kd\endcsname{\let\PY@bf=\textbf\def\PY@tc##1{\textcolor[rgb]{0.00,0.50,0.00}{##1}}}
\expandafter\def\csname PY@tok@kn\endcsname{\let\PY@bf=\textbf\def\PY@tc##1{\textcolor[rgb]{0.00,0.50,0.00}{##1}}}
\expandafter\def\csname PY@tok@kr\endcsname{\let\PY@bf=\textbf\def\PY@tc##1{\textcolor[rgb]{0.00,0.50,0.00}{##1}}}
\expandafter\def\csname PY@tok@bp\endcsname{\def\PY@tc##1{\textcolor[rgb]{0.00,0.50,0.00}{##1}}}
\expandafter\def\csname PY@tok@fm\endcsname{\def\PY@tc##1{\textcolor[rgb]{0.00,0.00,1.00}{##1}}}
\expandafter\def\csname PY@tok@vc\endcsname{\def\PY@tc##1{\textcolor[rgb]{0.10,0.09,0.49}{##1}}}
\expandafter\def\csname PY@tok@vg\endcsname{\def\PY@tc##1{\textcolor[rgb]{0.10,0.09,0.49}{##1}}}
\expandafter\def\csname PY@tok@vi\endcsname{\def\PY@tc##1{\textcolor[rgb]{0.10,0.09,0.49}{##1}}}
\expandafter\def\csname PY@tok@vm\endcsname{\def\PY@tc##1{\textcolor[rgb]{0.10,0.09,0.49}{##1}}}
\expandafter\def\csname PY@tok@sa\endcsname{\def\PY@tc##1{\textcolor[rgb]{0.73,0.13,0.13}{##1}}}
\expandafter\def\csname PY@tok@sb\endcsname{\def\PY@tc##1{\textcolor[rgb]{0.73,0.13,0.13}{##1}}}
\expandafter\def\csname PY@tok@sc\endcsname{\def\PY@tc##1{\textcolor[rgb]{0.73,0.13,0.13}{##1}}}
\expandafter\def\csname PY@tok@dl\endcsname{\def\PY@tc##1{\textcolor[rgb]{0.73,0.13,0.13}{##1}}}
\expandafter\def\csname PY@tok@s2\endcsname{\def\PY@tc##1{\textcolor[rgb]{0.73,0.13,0.13}{##1}}}
\expandafter\def\csname PY@tok@sh\endcsname{\def\PY@tc##1{\textcolor[rgb]{0.73,0.13,0.13}{##1}}}
\expandafter\def\csname PY@tok@s1\endcsname{\def\PY@tc##1{\textcolor[rgb]{0.73,0.13,0.13}{##1}}}
\expandafter\def\csname PY@tok@mb\endcsname{\def\PY@tc##1{\textcolor[rgb]{0.40,0.40,0.40}{##1}}}
\expandafter\def\csname PY@tok@mf\endcsname{\def\PY@tc##1{\textcolor[rgb]{0.40,0.40,0.40}{##1}}}
\expandafter\def\csname PY@tok@mh\endcsname{\def\PY@tc##1{\textcolor[rgb]{0.40,0.40,0.40}{##1}}}
\expandafter\def\csname PY@tok@mi\endcsname{\def\PY@tc##1{\textcolor[rgb]{0.40,0.40,0.40}{##1}}}
\expandafter\def\csname PY@tok@il\endcsname{\def\PY@tc##1{\textcolor[rgb]{0.40,0.40,0.40}{##1}}}
\expandafter\def\csname PY@tok@mo\endcsname{\def\PY@tc##1{\textcolor[rgb]{0.40,0.40,0.40}{##1}}}
\expandafter\def\csname PY@tok@ch\endcsname{\let\PY@it=\textit\def\PY@tc##1{\textcolor[rgb]{0.25,0.50,0.50}{##1}}}
\expandafter\def\csname PY@tok@cm\endcsname{\let\PY@it=\textit\def\PY@tc##1{\textcolor[rgb]{0.25,0.50,0.50}{##1}}}
\expandafter\def\csname PY@tok@cpf\endcsname{\let\PY@it=\textit\def\PY@tc##1{\textcolor[rgb]{0.25,0.50,0.50}{##1}}}
\expandafter\def\csname PY@tok@c1\endcsname{\let\PY@it=\textit\def\PY@tc##1{\textcolor[rgb]{0.25,0.50,0.50}{##1}}}
\expandafter\def\csname PY@tok@cs\endcsname{\let\PY@it=\textit\def\PY@tc##1{\textcolor[rgb]{0.25,0.50,0.50}{##1}}}

\def\PYZbs{\char`\\}
\def\PYZus{\char`\_}
\def\PYZob{\char`\{}
\def\PYZcb{\char`\}}
\def\PYZca{\char`\^}
\def\PYZam{\char`\&}
\def\PYZlt{\char`\<}
\def\PYZgt{\char`\>}
\def\PYZsh{\char`\#}
\def\PYZpc{\char`\%}
\def\PYZdl{\char`\$}
\def\PYZhy{\char`\-}
\def\PYZsq{\char`\'}
\def\PYZdq{\char`\"}
\def\PYZti{\char`\~}
% for compatibility with earlier versions
\def\PYZat{@}
\def\PYZlb{[}
\def\PYZrb{]}
\makeatother


    % Exact colors from NB
    \definecolor{incolor}{rgb}{0.0, 0.0, 0.5}
    \definecolor{outcolor}{rgb}{0.545, 0.0, 0.0}



    
    % Prevent overflowing lines due to hard-to-break entities
    \sloppy 
    % Setup hyperref package
    \hypersetup{
      breaklinks=true,  % so long urls are correctly broken across lines
      colorlinks=true,
      urlcolor=urlcolor,
      linkcolor=linkcolor,
      citecolor=citecolor,
      }
    % Slightly bigger margins than the latex defaults
    
    \geometry{verbose,tmargin=1in,bmargin=1in,lmargin=1in,rmargin=1in}
    
    

    \begin{document}
    
    
    \maketitle
    
    

    
    \#

Lyon's Air Quality

By Junholv OBO

    \begin{Verbatim}[commandchars=\\\{\}]
{\color{incolor}In [{\color{incolor}719}]:} \PY{k+kn}{from} \PY{n+nn}{IPython}\PY{n+nn}{.}\PY{n+nn}{display} \PY{k}{import} \PY{n}{HTML}\PY{p}{,} \PY{n}{Image}\PY{p}{,} \PY{n}{YouTubeVideo}\PY{p}{,} \PY{n}{display}
          \PY{n}{Image}\PY{p}{(}\PY{l+s+s1}{\PYZsq{}}\PY{l+s+s1}{https://venissieuxinfos.fr/wp\PYZhy{}content/uploads/20170403\PYZus{}nouveau\PYZhy{}metroB\PYZhy{}exterieur\PYZus{}alstom.jpg}\PY{l+s+s1}{\PYZsq{}}\PY{p}{,} \PY{n}{width}\PY{o}{=}\PY{l+m+mi}{600}\PY{p}{)}
\end{Verbatim}

\texttt{\color{outcolor}Out[{\color{outcolor}719}]:}
    
    \begin{center}
    \adjustimage{max size={0.9\linewidth}{0.9\paperheight}}{output_1_0.jpeg}
    \end{center}
    { \hspace*{\fill} \\}
    

    La dernière étude publiée sur la qualité de l'air dans le métro de Lyon
date de 2002.

Cela fait donc plus de 17 ans que plus rien n'avait était rendus public.
J'ai pu remarquer sur beaucoups de réseaux sociaux, certains sans cesse
n'hésitaient pas à interpeller les autorité regulatrice tel que Sytral
ou un exploitant comme Keolis sur cette situation.

Depuis mon arrivé sur Paris, j'ai rémarquer que mal-grès que la ville ne
plaît pas forcément, certaines informations était mis à dispotion au
contribuable.

En effet, cette abscence d'informations s'est sentis à chaque fois que
la RATP communiquait leurs données sur le métro parisien; et non TCL à
leurs usagers lyonnais.

De plus, une transparence assujetti en Avril 2018, par la mise en place
d'une API permettant de connaître la qualité de l'air dans trois
stations: http://www.iseo.fr/ratp/RATP.php.

En parallèle, des réseaux de transports en commun comme celui de Lyon se
sont mis au travail de manière coordonnée pour étudier la compostion de
la pollution et particules dans je cite : ``Les enceintes ferroviaires
souterraines''.

Depuis 2015, le ministère de l'Environnement et l'INERIS (Institut
National de l'Environnement Industriel et des Risques) ont lancé un
travail sur un protocole d'expérimentation commun. Le Sytral et Keolis
ont participé à une phase expérimentale de mesures en 2017. Deux types
de mesures ont alors été réalisées:

-en mars 2017, des mesures des particules fines PM10 (inférieur à 10
micromètres, en comparaison un cheveu fait en moyenne 75 micromètre de
diamètre) ont été éffectuées durant 15 minutes dans chaque station en
heure de pointe, matin et soir.

-en juin 2017, des meures ont été réalisée en cotninu durant deux
semaines sur le quai de la ligne B de la station Saxe-Gambetta, avec une
semaine pour les particules fines PM10 et une semaine pour les
particules très fines PM2.5 (inférieures à 2,5 micromètres).
Contrairement à celles de mars 2017, ces mesures en continu permettent
de voir les variations dans le temps.

Aujourd'hui, le Sytral a accepté de transmettre l'ensemble de ces
mesures, que j'ai décidé de mettre en exergue avec une étude pour vous.

Vous devez savoir, que ces particules fines PM10 et très fines PM2.5 ne
sont pas présentes uniquement dans le réseau métro, mais bien dans l'air
ambiant, partout où nous respirons. Leur concentration dépend de
plusieurs facteurs, tout comme leurs composition qui peut être très
varié (particules rejtées par les voitures, résidus de freinage, pneus,
chauffage, composés organiques, cendres \ldots{}). Donc inutile de
charger sur Sytral ou Keolis après mon étude.

Ainsi, selon le Sytral, \textbf{``les niveaux d'empoussièrement sont
directement liés à la fréquence du passage du métro''. Quant à leur
nature, ``Les particules métalliques analysées à la station Saxe sont
composées à 88 \% de fer. Contrairement à l'air ambiant où les
particules proviennent majoritairement du trafic automobile, la
pollution des enceintes ferroviaires est issue principalement du
roulement et de l'usure des matériaux en frottement du matériel
roulant''.}

Plus ces particules sont petites, plus elles peuvent pénétrer loin dans
l'organisme. Selon le Centre international de recherche sur le cancer
(CIRC), ``6 à 11 \% des décès par cancer du poumon seraient attribuables
à l'exposition chronique aux particules fines''. Les particules fines
peuvent être également à l'origine de maladie pulmonaire,
cardio-vasculaire et être un facteur aggravant chez certaines personne

En France, selon le code de l'environnement, les valeurs limites pour la
protection de la santé en matière de PM10 sont de 50 µg/m³ en moyenne
sur 24 heures (à ne pas dépasser plus de trente-cinq fois par année
civile ) et 40 µg/ m³ en moyenne annuelle civile. L'objectif de qualité
est de 30 µg/ m³ en moyenne annuelle. Pour les PM2.5, les valeurs
limites sont de 25 µg/m³ en moyenne annuelle, l'objectif de qualité
étant à 10 µg/m³.

Dans tous les cas, ces limites basées sur des moyennes ne concernent pas
la seule exposition dans les transports en commun, mais bien celle lors
de l'ensemble de la journée. Les voyageurs ne passent que quelques
minutes dans le réseau métro, quand le personnel travaillant à
l'intérieur est forcément plus exposé, car présent plus longtemps. Selon
le Sytral : "Le Conseil Supérieur d'Hygiène publique de France (CSHPF) a
publié en 2001 un avis selon lequel ces valeurs (celles de la loi) ne
sont pas applicables dans les enceintes ferroviaires souterraines. Le
Conseil a proposé une méthode de calcul afin d'établir un seuil de
concentration limite en PM10 (Csout) en moyenne horaire, basé sur la
notion d'exposition cumulée dans l'air ambiant extérieur et à
l'intérieur des enceintes de métro, et respectant la valeur limite
journalière.

    The last published study on air quality in the Lyon metro was published
in 2002.

It has therefore been more than 17 years since nothing has been made
public. I have noticed on many social networks, some of them did not
hesitate to question regulatory authorities such as Sytral or an
operator such as Keolis about this situation.

Since my arrival in Paris, I have noticed that despite the fact that the
city does not necessarily like it, some information was made available
to the taxpayer.

Indeed, this lack of information was felt every time; RATP communicated
their data on the Parisian metro; not TCL to their Lyon's users.

In addition, transparency will be required in April 2018, through the
implementation of an API to determine air quality in three stations:
http://www.iseo.fr/ratp/RATP.php.

In parallel, public transport networks such as the one in Lyon have
started working in a coordinated way to study the composition of
pollution and particles and I quote: ``Underground railway enclosures''.

Since 2015, the Ministry of the Environment and INERIS (Institut
National de l'Environnement Industriel et des Risques) have been working
on a common experimental protocol. Le Sytral and Keolis participated in
an experimental phase of measurements in 2017. Two types of measurements
were then carried out:

-In March 2017, measurements of fine particulate matter PM10 (less than
10 micrometers, compared to a hair with an average diameter of 75
micrometers) were taken for 15 minutes at each station during peak
hours, morning and evening.

-In June 2017, measurements were taken in cotninu for two weeks on the
platform of line B of the Saxe-Gambetta station, with one week for fine
particles PM10 and one week for very fine particles PM2.5 (less than 2.5
micrometers). Unlike those of March 2017, these continuous measurements
allow us to see variations over time.

Today, Sytral has agreed to transmit all these measures, which I have
decided to highlight with a study for you.

You should know that these fine particles PM10 and very fine PM2.5 are
not only present in the metro network, but also in the ambient air,
wherever we breathe. Their concentration depends on several factors, as
well as their composition, which can be very varied (particles emitted
by cars, brake residues, tires, heating, organic compounds, ashes,
etc.). So there's no need to attack Sytral or Keolis after my study
please ..

Thus, according to the Sytral, \textbf{``dust levels are directly
related to the frequency of the metro's passage''. As for their nature,
``The metal particles analysed at the Saxony station are 88\% iron.
Unlike ambient air, where particles mainly come from motor traffic,
pollution in railway enclosures is mainly due to rolling and wear and
tear on the friction materials of rolling stock''.}

The smaller these particles are, the farther they can penetrate the
body. According to the International Agency for Research on Cancer
(IARC), ``6 to 11\% of lung cancer deaths are due to chronic exposure to
fine particles''. Fine particles can also cause lung disease,
cardiovascular disease and be an aggravating factor in some people.

In France, according to the French Environmental Code, the limit values
for health protection of PM10 are 50 µg/m³ on average over 24 hours (not
to be exceeded more than 35 times per calendar year) and 40 µg/m³ on
average per calendar year. The quality objective is 30 µg/m³ as an
annual average. For PM2.5, the limit values are 25 µg/m³ as an annual
average, the quality objective being 10 µg/m³.

In any case, these average-based limits do not only apply to exposure in
public transport, but also to exposure throughout the day. Travellers
spend only a few minutes in the metro system, when staff working indoors
are necessarily more exposed because they are present longer. According
to Le Sytral: "In 2001, the Conseil Supérieur d'Hygiène Publique de
France (CSHPF) published an opinion to the effect that these values
(those of the law) are not applicable in underground railways. The
Council proposed a calculation method to establish a threshold limit
concentration of PM10 (Csout) as an hourly average, based on the concept
of cumulative exposure in outdoor ambient air and inside metro
enclosures, and respecting the daily limit value.

    \begin{Verbatim}[commandchars=\\\{\}]
{\color{incolor}In [{\color{incolor}720}]:} \PY{c+c1}{\PYZsh{}Pandas library}
          \PY{k+kn}{import} \PY{n+nn}{pandas} \PY{k}{as} \PY{n+nn}{pd}
\end{Verbatim}


    \begin{Verbatim}[commandchars=\\\{\}]
{\color{incolor}In [{\color{incolor}721}]:} \PY{c+c1}{\PYZsh{}Nous ne savons pas a quoi va ressembler la datframe autant mettre des fonctions permet de tout voir}
          \PY{c+c1}{\PYZsh{}This function is for watch all dataframe row and column}
          
          \PY{n}{pd}\PY{o}{.}\PY{n}{set\PYZus{}option}\PY{p}{(}\PY{l+s+s1}{\PYZsq{}}\PY{l+s+s1}{display.max\PYZus{}rows}\PY{l+s+s1}{\PYZsq{}}\PY{p}{,}\PY{k+kc}{None}\PY{p}{,}\PY{l+s+s2}{\PYZdq{}}\PY{l+s+s2}{display.max\PYZus{}columns}\PY{l+s+s2}{\PYZdq{}}\PY{p}{,} \PY{k+kc}{None}\PY{p}{)}
          \PY{c+c1}{\PYZsh{} me montre toutes mes columns de ma df (df = dataframe)}
          \PY{c+c1}{\PYZsh{}me montre tout les ligne de macolumn ma df (df = dataframe)}
          \PY{c+c1}{\PYZsh{}None designe le nombre d\PYZsq{}element maximum, dans mon cas aucune limite}
          \PY{c+c1}{\PYZsh{}Nous aurions pu le remplacer par un chiffre.}
\end{Verbatim}


    \hypertarget{recuperons-nos-data-en-dataframe}{%
\subsection{Recuperons nos Data en
dataframe}\label{recuperons-nos-data-en-dataframe}}

\hypertarget{well-get-our-data-back}{%
\subsection{We'll get our Data back}\label{well-get-our-data-back}}

    \begin{Verbatim}[commandchars=\\\{\}]
{\color{incolor}In [{\color{incolor}722}]:} \PY{c+c1}{\PYZsh{}Ligne de Metro}
          \PY{c+c1}{\PYZsh{}Subway\PYZsq{}s line}
          \PY{n}{dfLigneA} \PY{o}{=} \PY{n}{pd}\PY{o}{.}\PY{n}{read\PYZus{}csv}\PY{p}{(}\PY{l+s+s2}{\PYZdq{}}\PY{l+s+s2}{Data/Dirty/LigneA.csv}\PY{l+s+s2}{\PYZdq{}}\PY{p}{,} \PY{n}{sep}\PY{o}{=}\PY{l+s+s2}{\PYZdq{}}\PY{l+s+s2}{;}\PY{l+s+s2}{\PYZdq{}}\PY{p}{)}
          \PY{n}{dfLigneB} \PY{o}{=} \PY{n}{pd}\PY{o}{.}\PY{n}{read\PYZus{}csv}\PY{p}{(}\PY{l+s+s2}{\PYZdq{}}\PY{l+s+s2}{Data/Dirty/LigneB.csv}\PY{l+s+s2}{\PYZdq{}}\PY{p}{,} \PY{n}{sep}\PY{o}{=}\PY{l+s+s2}{\PYZdq{}}\PY{l+s+s2}{;}\PY{l+s+s2}{\PYZdq{}}\PY{p}{)}
          \PY{n}{dfLigneC} \PY{o}{=} \PY{n}{pd}\PY{o}{.}\PY{n}{read\PYZus{}csv}\PY{p}{(}\PY{l+s+s2}{\PYZdq{}}\PY{l+s+s2}{Data/Dirty/LigneC.csv}\PY{l+s+s2}{\PYZdq{}}\PY{p}{,} \PY{n}{sep}\PY{o}{=}\PY{l+s+s2}{\PYZdq{}}\PY{l+s+s2}{;}\PY{l+s+s2}{\PYZdq{}}\PY{p}{)}
          \PY{n}{dfLigneD} \PY{o}{=} \PY{n}{pd}\PY{o}{.}\PY{n}{read\PYZus{}csv}\PY{p}{(}\PY{l+s+s2}{\PYZdq{}}\PY{l+s+s2}{Data/Dirty/LigneD.csv}\PY{l+s+s2}{\PYZdq{}}\PY{p}{,} \PY{n}{sep}\PY{o}{=}\PY{l+s+s2}{\PYZdq{}}\PY{l+s+s2}{;}\PY{l+s+s2}{\PYZdq{}}\PY{p}{)}
          
          \PY{c+c1}{\PYZsh{}Station metro}
          \PY{c+c1}{\PYZsh{}Subway\PYZsq{}s stop}
          \PY{n}{dfStationSaxe\PYZus{}PM10} \PY{o}{=} \PY{n}{pd}\PY{o}{.}\PY{n}{read\PYZus{}csv}\PY{p}{(}\PY{l+s+s2}{\PYZdq{}}\PY{l+s+s2}{Data/Dirty/Station Saxe\PYZhy{}Gambetta PM10.csv}\PY{l+s+s2}{\PYZdq{}}\PY{p}{,} \PY{n}{sep}\PY{o}{=}\PY{l+s+s2}{\PYZdq{}}\PY{l+s+s2}{;}\PY{l+s+s2}{\PYZdq{}}\PY{p}{)}
          \PY{n}{dfStationSaxePM2\PYZus{}5} \PY{o}{=} \PY{n}{pd}\PY{o}{.}\PY{n}{read\PYZus{}csv}\PY{p}{(}\PY{l+s+s2}{\PYZdq{}}\PY{l+s+s2}{Data/Dirty/Station Saxe\PYZhy{}Gambetta PM2.5.csv}\PY{l+s+s2}{\PYZdq{}}\PY{p}{,} \PY{n}{sep}\PY{o}{=}\PY{l+s+s2}{\PYZdq{}}\PY{l+s+s2}{;}\PY{l+s+s2}{\PYZdq{}}\PY{p}{)}
          
          \PY{c+c1}{\PYZsh{}Finiculaire}
          \PY{c+c1}{\PYZsh{}Finicular}
          \PY{n}{dfFuniculaireSaintJus} \PY{o}{=} \PY{n}{pd}\PY{o}{.}\PY{n}{read\PYZus{}csv}\PY{p}{(}\PY{l+s+s2}{\PYZdq{}}\PY{l+s+s2}{Data/Dirty/Funiculaire Saint\PYZhy{}Just.csv}\PY{l+s+s2}{\PYZdq{}}\PY{p}{,} \PY{n}{sep}\PY{o}{=}\PY{l+s+s2}{\PYZdq{}}\PY{l+s+s2}{;}\PY{l+s+s2}{\PYZdq{}}\PY{p}{)}
          \PY{n}{dfFuniculaireFourviere} \PY{o}{=} \PY{n}{pd}\PY{o}{.}\PY{n}{read\PYZus{}csv}\PY{p}{(}\PY{l+s+s2}{\PYZdq{}}\PY{l+s+s2}{Data/Dirty/Funiculaire Fourviere.csv}\PY{l+s+s2}{\PYZdq{}}\PY{p}{,} \PY{n}{sep}\PY{o}{=}\PY{l+s+s2}{\PYZdq{}}\PY{l+s+s2}{;}\PY{l+s+s2}{\PYZdq{}}\PY{p}{)}
\end{Verbatim}


    \hypertarget{ligne-de-metro}{%
\section{Ligne de Metro}\label{ligne-de-metro}}

\hypertarget{subways-line}{%
\section{Subway's line}\label{subways-line}}

    \hypertarget{dflignea}{%
\subsubsection{dfLigneA}\label{dflignea}}

    \begin{Verbatim}[commandchars=\\\{\}]
{\color{incolor}In [{\color{incolor}723}]:} \PY{n}{dfLigneA}\PY{o}{.}\PY{n}{head}\PY{p}{(}\PY{p}{)}
\end{Verbatim}


\begin{Verbatim}[commandchars=\\\{\}]
{\color{outcolor}Out[{\color{outcolor}723}]:}   Station (Matin / Soir)  PM10 µg/m³
          0                Moyenne         107
          1              La soie s         166
          2              La soie m         145
          3     Laurent Bonnevay s         112
          4     Laurent Bonnevay m         155
\end{Verbatim}
            
    \begin{Verbatim}[commandchars=\\\{\}]
{\color{incolor}In [{\color{incolor}724}]:} \PY{c+c1}{\PYZsh{}Ici la fonction apply remplace tout mes Oui par 1 et Non par 0}
          \PY{o}{\PYZpc{}}\PY{k}{timeit}
          
          \PY{n}{dfLigneA} \PY{o}{=} \PY{n}{pd}\PY{o}{.}\PY{n}{read\PYZus{}csv}\PY{p}{(}\PY{l+s+s2}{\PYZdq{}}\PY{l+s+s2}{Data/Dirty/LigneA.csv}\PY{l+s+s2}{\PYZdq{}}\PY{p}{,} \PY{n}{sep}\PY{o}{=}\PY{l+s+s2}{\PYZdq{}}\PY{l+s+s2}{;}\PY{l+s+s2}{\PYZdq{}}\PY{p}{)}
          \PY{c+c1}{\PYZsh{}New column = Take the last letter of my column }
          \PY{n}{dfLigneA}\PY{p}{[}\PY{l+s+s2}{\PYZdq{}}\PY{l+s+s2}{Matin/Soir}\PY{l+s+s2}{\PYZdq{}}\PY{p}{]} \PY{o}{=} \PY{n}{dfLigneA}\PY{p}{[}\PY{l+s+s2}{\PYZdq{}}\PY{l+s+s2}{Station (Matin / Soir)}\PY{l+s+s2}{\PYZdq{}}\PY{p}{]}
          
          \PY{c+c1}{\PYZsh{}Column Matin}
          \PY{n}{dfLigneA}\PY{p}{[}\PY{l+s+s2}{\PYZdq{}}\PY{l+s+s2}{Matin}\PY{l+s+s2}{\PYZdq{}}\PY{p}{]} \PY{o}{=} \PY{n}{dfLigneA}\PY{p}{[}\PY{l+s+s2}{\PYZdq{}}\PY{l+s+s2}{Matin/Soir}\PY{l+s+s2}{\PYZdq{}}\PY{p}{]}\PY{o}{.}\PY{n}{apply}\PY{p}{(}\PY{k}{lambda} \PY{n}{x}\PY{p}{:} \PY{l+m+mi}{1} \PY{k}{if} \PY{n}{x}\PY{o}{==}\PY{l+s+s1}{\PYZsq{}}\PY{l+s+s1}{m}\PY{l+s+s1}{\PYZsq{}} \PY{k}{else} \PY{l+m+mi}{0}\PY{p}{)}
          \PY{n}{dfLigneA}\PY{p}{[}\PY{l+s+s2}{\PYZdq{}}\PY{l+s+s2}{Matin}\PY{l+s+s2}{\PYZdq{}}\PY{p}{]} \PY{o}{=} \PY{n}{dfLigneA}\PY{p}{[}\PY{l+s+s2}{\PYZdq{}}\PY{l+s+s2}{Matin}\PY{l+s+s2}{\PYZdq{}}\PY{p}{]}\PY{o}{.}\PY{n}{astype}\PY{p}{(}\PY{n+nb}{int}\PY{p}{)} \PY{c+c1}{\PYZsh{}Column Matin become a column in int format }
          
          \PY{c+c1}{\PYZsh{}Column Soir}
          \PY{n}{dfLigneA}\PY{p}{[}\PY{l+s+s2}{\PYZdq{}}\PY{l+s+s2}{Soir}\PY{l+s+s2}{\PYZdq{}}\PY{p}{]} \PY{o}{=} \PY{n}{dfLigneA}\PY{p}{[}\PY{l+s+s2}{\PYZdq{}}\PY{l+s+s2}{Matin/Soir}\PY{l+s+s2}{\PYZdq{}}\PY{p}{]}\PY{o}{.}\PY{n}{apply}\PY{p}{(}\PY{k}{lambda} \PY{n}{x}\PY{p}{:} \PY{l+m+mi}{1} \PY{k}{if} \PY{n}{x}\PY{o}{==}\PY{l+s+s1}{\PYZsq{}}\PY{l+s+s1}{s}\PY{l+s+s1}{\PYZsq{}} \PY{k}{else} \PY{l+m+mi}{0}\PY{p}{)}
          \PY{n}{dfLigneA}\PY{p}{[}\PY{l+s+s2}{\PYZdq{}}\PY{l+s+s2}{Soir}\PY{l+s+s2}{\PYZdq{}}\PY{p}{]} \PY{o}{=} \PY{n}{dfLigneA}\PY{p}{[}\PY{l+s+s2}{\PYZdq{}}\PY{l+s+s2}{Soir}\PY{l+s+s2}{\PYZdq{}}\PY{p}{]}\PY{o}{.}\PY{n}{astype}\PY{p}{(}\PY{n+nb}{int}\PY{p}{)}\PY{c+c1}{\PYZsh{}Column Soir become a column in int format }
          
          \PY{c+c1}{\PYZsh{}Column Station}
          \PY{n}{dfLigneA}\PY{p}{[}\PY{l+s+s2}{\PYZdq{}}\PY{l+s+s2}{Station}\PY{l+s+s2}{\PYZdq{}}\PY{p}{]} \PY{o}{=} \PY{n}{dfLigneA}\PY{p}{[}\PY{l+s+s2}{\PYZdq{}}\PY{l+s+s2}{Station (Matin / Soir)}\PY{l+s+s2}{\PYZdq{}}\PY{p}{]}
          
          \PY{c+c1}{\PYZsh{}Supprimer la première ligne par l\PYZsq{}index}
          \PY{n}{dfLigneA}\PY{o}{.}\PY{n}{drop}\PY{p}{(}\PY{l+m+mi}{0}\PY{p}{,} \PY{n}{inplace}\PY{o}{=}\PY{k+kc}{True}\PY{p}{)}
          
          \PY{c+c1}{\PYZsh{}reindex}
          \PY{n}{dfLigneA} \PY{o}{=} \PY{n}{dfLigneA}\PY{o}{.}\PY{n}{reset\PYZus{}index}\PY{p}{(}\PY{n}{drop} \PY{o}{=} \PY{k+kc}{True}\PY{p}{)}
          
          \PY{c+c1}{\PYZsh{}remove unnecessary column}
          \PY{k}{del} \PY{n}{dfLigneA}\PY{p}{[}\PY{l+s+s2}{\PYZdq{}}\PY{l+s+s2}{Station (Matin / Soir)}\PY{l+s+s2}{\PYZdq{}}\PY{p}{]}
          \PY{k}{del} \PY{n}{dfLigneA}\PY{p}{[}\PY{l+s+s2}{\PYZdq{}}\PY{l+s+s2}{Matin/Soir}\PY{l+s+s2}{\PYZdq{}}\PY{p}{]}
          
          \PY{c+c1}{\PYZsh{}Change order of my column}
          \PY{n}{dfLigneA} \PY{o}{=} \PY{n}{dfLigneA}\PY{p}{[}\PY{p}{[}\PY{l+s+s2}{\PYZdq{}}\PY{l+s+s2}{Station}\PY{l+s+s2}{\PYZdq{}}\PY{p}{,}\PY{l+s+s1}{\PYZsq{}}\PY{l+s+s1}{PM10 µg/m³}\PY{l+s+s1}{\PYZsq{}}\PY{p}{,} \PY{l+s+s1}{\PYZsq{}}\PY{l+s+s1}{Matin}\PY{l+s+s1}{\PYZsq{}}\PY{p}{,} \PY{l+s+s1}{\PYZsq{}}\PY{l+s+s1}{Soir}\PY{l+s+s1}{\PYZsq{}}\PY{p}{]}\PY{p}{]}
          
          
          \PY{n}{dfLigneA}\PY{o}{.}\PY{n}{to\PYZus{}csv}\PY{p}{(}\PY{l+s+s1}{\PYZsq{}}\PY{l+s+s1}{/Users/Administrateur/Documents/ML/Mes projet/lyon/Qualiter\PYZus{}Air/Data/Clean/LigneA.csv}\PY{l+s+s1}{\PYZsq{}}\PY{p}{,} \PY{n}{index}\PY{o}{=}\PY{k+kc}{True}\PY{p}{,} \PY{n}{sep}\PY{o}{=}\PY{l+s+s1}{\PYZsq{}}\PY{l+s+s1}{;}\PY{l+s+s1}{\PYZsq{}}\PY{p}{,} \PY{n}{header}\PY{o}{=}\PY{k+kc}{True}\PY{p}{)}
          \PY{n}{dfLigneA}\PY{o}{.}\PY{n}{head}\PY{p}{(}\PY{p}{)}
\end{Verbatim}


\begin{Verbatim}[commandchars=\\\{\}]
{\color{outcolor}Out[{\color{outcolor}724}]:}               Station  PM10 µg/m³  Matin  Soir
          0           La soie s         166      0     0
          1           La soie m         145      0     0
          2  Laurent Bonnevay s         112      0     0
          3  Laurent Bonnevay m         155      0     0
          4            Cusset s         106      0     0
\end{Verbatim}
            
    \hypertarget{dfligneb}{%
\subsubsection{dfLigneB}\label{dfligneb}}

    \begin{Verbatim}[commandchars=\\\{\}]
{\color{incolor}In [{\color{incolor}725}]:} \PY{n}{dfLigneB}\PY{o}{.}\PY{n}{head}\PY{p}{(}\PY{p}{)}
\end{Verbatim}


\begin{Verbatim}[commandchars=\\\{\}]
{\color{outcolor}Out[{\color{outcolor}725}]:}         Station  PM10  µg/m³
          0       Moyenne           95
          1  Charpennes s           80
          2  Charpennes m          134
          3   Brotteaux s           74
          4   Brotteaux m          116
\end{Verbatim}
            
    \begin{Verbatim}[commandchars=\\\{\}]
{\color{incolor}In [{\color{incolor}726}]:} \PY{n}{dfLigneB} \PY{o}{=} \PY{n}{pd}\PY{o}{.}\PY{n}{read\PYZus{}csv}\PY{p}{(}\PY{l+s+s2}{\PYZdq{}}\PY{l+s+s2}{Data/Dirty/LigneB.csv}\PY{l+s+s2}{\PYZdq{}}\PY{p}{,} \PY{n}{sep}\PY{o}{=}\PY{l+s+s2}{\PYZdq{}}\PY{l+s+s2}{;}\PY{l+s+s2}{\PYZdq{}}\PY{p}{)}
          \PY{c+c1}{\PYZsh{}New column = Take the last letter of my column }
          \PY{n}{dfLigneB}\PY{p}{[}\PY{l+s+s2}{\PYZdq{}}\PY{l+s+s2}{Matin/Soir}\PY{l+s+s2}{\PYZdq{}}\PY{p}{]} \PY{o}{=} \PY{n}{dfLigneB}\PY{p}{[}\PY{l+s+s2}{\PYZdq{}}\PY{l+s+s2}{Station}\PY{l+s+s2}{\PYZdq{}}\PY{p}{]}\PY{o}{.}\PY{n}{str}\PY{p}{[}\PY{o}{\PYZhy{}}\PY{l+m+mi}{1}\PY{p}{:}\PY{p}{]}
          
          \PY{c+c1}{\PYZsh{}Column Matin}
          \PY{n}{dfLigneB}\PY{p}{[}\PY{l+s+s2}{\PYZdq{}}\PY{l+s+s2}{Matin}\PY{l+s+s2}{\PYZdq{}}\PY{p}{]} \PY{o}{=} \PY{n}{dfLigneB}\PY{p}{[}\PY{l+s+s2}{\PYZdq{}}\PY{l+s+s2}{Matin/Soir}\PY{l+s+s2}{\PYZdq{}}\PY{p}{]}\PY{o}{.}\PY{n}{apply}\PY{p}{(}\PY{k}{lambda} \PY{n}{x}\PY{p}{:} \PY{l+m+mi}{1} \PY{k}{if} \PY{n}{x}\PY{o}{==}\PY{l+s+s1}{\PYZsq{}}\PY{l+s+s1}{m}\PY{l+s+s1}{\PYZsq{}} \PY{k}{else} \PY{l+m+mi}{0}\PY{p}{)}
          \PY{n}{dfLigneB}\PY{p}{[}\PY{l+s+s2}{\PYZdq{}}\PY{l+s+s2}{Matin}\PY{l+s+s2}{\PYZdq{}}\PY{p}{]} \PY{o}{=} \PY{n}{dfLigneB}\PY{p}{[}\PY{l+s+s2}{\PYZdq{}}\PY{l+s+s2}{Matin}\PY{l+s+s2}{\PYZdq{}}\PY{p}{]}\PY{o}{.}\PY{n}{astype}\PY{p}{(}\PY{n+nb}{int}\PY{p}{)}
          
          \PY{c+c1}{\PYZsh{}Column Soir}
          \PY{n}{dfLigneB}\PY{p}{[}\PY{l+s+s2}{\PYZdq{}}\PY{l+s+s2}{Soir}\PY{l+s+s2}{\PYZdq{}}\PY{p}{]} \PY{o}{=} \PY{n}{dfLigneB}\PY{p}{[}\PY{l+s+s2}{\PYZdq{}}\PY{l+s+s2}{Matin/Soir}\PY{l+s+s2}{\PYZdq{}}\PY{p}{]}\PY{o}{.}\PY{n}{apply}\PY{p}{(}\PY{k}{lambda} \PY{n}{x}\PY{p}{:} \PY{l+m+mi}{1} \PY{k}{if} \PY{n}{x}\PY{o}{==}\PY{l+s+s1}{\PYZsq{}}\PY{l+s+s1}{s}\PY{l+s+s1}{\PYZsq{}} \PY{k}{else} \PY{l+m+mi}{0}\PY{p}{)}
          \PY{n}{dfLigneB}\PY{p}{[}\PY{l+s+s2}{\PYZdq{}}\PY{l+s+s2}{Soir}\PY{l+s+s2}{\PYZdq{}}\PY{p}{]} \PY{o}{=} \PY{n}{dfLigneB}\PY{p}{[}\PY{l+s+s2}{\PYZdq{}}\PY{l+s+s2}{Soir}\PY{l+s+s2}{\PYZdq{}}\PY{p}{]}\PY{o}{.}\PY{n}{astype}\PY{p}{(}\PY{n+nb}{int}\PY{p}{)}
          
          
          \PY{c+c1}{\PYZsh{}Supprimer la première ligne par l\PYZsq{}index}
          \PY{n}{dfLigneB}\PY{o}{.}\PY{n}{drop}\PY{p}{(}\PY{l+m+mi}{0}\PY{p}{,} \PY{n}{inplace} \PY{o}{=} \PY{k+kc}{True}\PY{p}{)}
          \PY{k}{del} \PY{n}{dfLigneB}\PY{p}{[}\PY{l+s+s2}{\PYZdq{}}\PY{l+s+s2}{Matin/Soir}\PY{l+s+s2}{\PYZdq{}}\PY{p}{]}
          
          \PY{c+c1}{\PYZsh{}reindex }
          \PY{n}{dfLigneB} \PY{o}{=} \PY{n}{dfLigneB}\PY{o}{.}\PY{n}{reset\PYZus{}index}\PY{p}{(}\PY{n}{drop}\PY{o}{=}\PY{k+kc}{True}\PY{p}{)}
          
          \PY{n}{dfLigneB}\PY{o}{.}\PY{n}{to\PYZus{}csv}\PY{p}{(}\PY{l+s+s1}{\PYZsq{}}\PY{l+s+s1}{/Users/Administrateur/Documents/ML/Mes projet/lyon/Qualiter\PYZus{}Air/Data/Clean/LigneB.csv}\PY{l+s+s1}{\PYZsq{}}\PY{p}{,} \PY{n}{index}\PY{o}{=}\PY{k+kc}{True}\PY{p}{,} \PY{n}{sep}\PY{o}{=}\PY{l+s+s1}{\PYZsq{}}\PY{l+s+s1}{;}\PY{l+s+s1}{\PYZsq{}}\PY{p}{,} \PY{n}{header}\PY{o}{=}\PY{k+kc}{True}\PY{p}{)}
          \PY{n}{dfLigneB}\PY{o}{.}\PY{n}{head}\PY{p}{(}\PY{p}{)}
\end{Verbatim}


\begin{Verbatim}[commandchars=\\\{\}]
{\color{outcolor}Out[{\color{outcolor}726}]:}         Station  PM10  µg/m³  Matin  Soir
          0  Charpennes s           80      0     1
          1  Charpennes m          134      1     0
          2   Brotteaux s           74      0     1
          3   Brotteaux m          116      1     0
          4   Part-Dieu s          111      0     1
\end{Verbatim}
            
    \hypertarget{dflignec}{%
\subsubsection{dfLigneC}\label{dflignec}}

    \begin{Verbatim}[commandchars=\\\{\}]
{\color{incolor}In [{\color{incolor}727}]:} \PY{n}{dfLigneC}\PY{o}{.}\PY{n}{head}\PY{p}{(}\PY{p}{)}
\end{Verbatim}


\begin{Verbatim}[commandchars=\\\{\}]
{\color{outcolor}Out[{\color{outcolor}727}]:}           Station  PM10   µg/ m³
          0         Moyenne             46
          1         Hénon s             26
          2         Hénon m             29
          3  Croix-Rousse s             40
          4  Croix-Rousse m             43
\end{Verbatim}
            
    \begin{Verbatim}[commandchars=\\\{\}]
{\color{incolor}In [{\color{incolor}728}]:} \PY{n}{dfLigneC} \PY{o}{=} \PY{n}{pd}\PY{o}{.}\PY{n}{read\PYZus{}csv}\PY{p}{(}\PY{l+s+s2}{\PYZdq{}}\PY{l+s+s2}{Data/Dirty/LigneC.csv}\PY{l+s+s2}{\PYZdq{}}\PY{p}{,} \PY{n}{sep}\PY{o}{=}\PY{l+s+s2}{\PYZdq{}}\PY{l+s+s2}{;}\PY{l+s+s2}{\PYZdq{}}\PY{p}{)}
          \PY{c+c1}{\PYZsh{}New column = Take the last letter of my column }
          \PY{n}{dfLigneC}\PY{p}{[}\PY{l+s+s2}{\PYZdq{}}\PY{l+s+s2}{Matin/Soir}\PY{l+s+s2}{\PYZdq{}}\PY{p}{]} \PY{o}{=} \PY{n}{dfLigneC}\PY{p}{[}\PY{l+s+s2}{\PYZdq{}}\PY{l+s+s2}{Station}\PY{l+s+s2}{\PYZdq{}}\PY{p}{]}\PY{o}{.}\PY{n}{str}\PY{p}{[}\PY{o}{\PYZhy{}}\PY{l+m+mi}{1}\PY{p}{:}\PY{p}{]}
          
          \PY{c+c1}{\PYZsh{}Column Matin}
          \PY{n}{dfLigneC}\PY{p}{[}\PY{l+s+s2}{\PYZdq{}}\PY{l+s+s2}{Matin}\PY{l+s+s2}{\PYZdq{}}\PY{p}{]} \PY{o}{=} \PY{n}{dfLigneC}\PY{p}{[}\PY{l+s+s2}{\PYZdq{}}\PY{l+s+s2}{Matin/Soir}\PY{l+s+s2}{\PYZdq{}}\PY{p}{]}\PY{o}{.}\PY{n}{apply}\PY{p}{(}\PY{k}{lambda} \PY{n}{x}\PY{p}{:} \PY{l+m+mi}{1} \PY{k}{if} \PY{n}{x}\PY{o}{==}\PY{l+s+s1}{\PYZsq{}}\PY{l+s+s1}{m}\PY{l+s+s1}{\PYZsq{}} \PY{k}{else} \PY{l+m+mi}{0}\PY{p}{)}
          \PY{n}{dfLigneC}\PY{p}{[}\PY{l+s+s2}{\PYZdq{}}\PY{l+s+s2}{Matin}\PY{l+s+s2}{\PYZdq{}}\PY{p}{]} \PY{o}{=} \PY{n}{dfLigneC}\PY{p}{[}\PY{l+s+s2}{\PYZdq{}}\PY{l+s+s2}{Matin}\PY{l+s+s2}{\PYZdq{}}\PY{p}{]}\PY{o}{.}\PY{n}{astype}\PY{p}{(}\PY{n+nb}{int}\PY{p}{)}
          
          \PY{c+c1}{\PYZsh{}Column Soir}
          \PY{n}{dfLigneC}\PY{p}{[}\PY{l+s+s2}{\PYZdq{}}\PY{l+s+s2}{Soir}\PY{l+s+s2}{\PYZdq{}}\PY{p}{]} \PY{o}{=} \PY{n}{dfLigneC}\PY{p}{[}\PY{l+s+s2}{\PYZdq{}}\PY{l+s+s2}{Matin/Soir}\PY{l+s+s2}{\PYZdq{}}\PY{p}{]}\PY{o}{.}\PY{n}{apply}\PY{p}{(}\PY{k}{lambda} \PY{n}{x}\PY{p}{:} \PY{l+m+mi}{1} \PY{k}{if} \PY{n}{x}\PY{o}{==}\PY{l+s+s1}{\PYZsq{}}\PY{l+s+s1}{s}\PY{l+s+s1}{\PYZsq{}} \PY{k}{else} \PY{l+m+mi}{0}\PY{p}{)}
          \PY{n}{dfLigneC}\PY{p}{[}\PY{l+s+s2}{\PYZdq{}}\PY{l+s+s2}{Soir}\PY{l+s+s2}{\PYZdq{}}\PY{p}{]} \PY{o}{=} \PY{n}{dfLigneC}\PY{p}{[}\PY{l+s+s2}{\PYZdq{}}\PY{l+s+s2}{Soir}\PY{l+s+s2}{\PYZdq{}}\PY{p}{]}\PY{o}{.}\PY{n}{astype}\PY{p}{(}\PY{n+nb}{int}\PY{p}{)}
          
          \PY{c+c1}{\PYZsh{}Column Station}
          \PY{c+c1}{\PYZsh{}dfLigneC[\PYZdq{}Station\PYZdq{}] = dfLigneC[\PYZdq{}Station\PYZdq{}]}
          
          \PY{c+c1}{\PYZsh{}Supprimer la première ligne par l\PYZsq{}index}
          \PY{n}{dfLigneC}\PY{o}{.}\PY{n}{drop}\PY{p}{(}\PY{l+m+mi}{0}\PY{p}{,} \PY{n}{inplace} \PY{o}{=} \PY{k+kc}{True}\PY{p}{)}
          \PY{k}{del} \PY{n}{dfLigneC}\PY{p}{[}\PY{l+s+s2}{\PYZdq{}}\PY{l+s+s2}{Matin/Soir}\PY{l+s+s2}{\PYZdq{}}\PY{p}{]}
          
          \PY{c+c1}{\PYZsh{}reindex }
          \PY{n}{dfLigneC} \PY{o}{=} \PY{n}{dfLigneC}\PY{o}{.}\PY{n}{reset\PYZus{}index}\PY{p}{(}\PY{n}{drop}\PY{o}{=}\PY{k+kc}{True}\PY{p}{)}
          
          \PY{n}{dfLigneC}\PY{o}{.}\PY{n}{to\PYZus{}csv}\PY{p}{(}\PY{l+s+s1}{\PYZsq{}}\PY{l+s+s1}{/Users/Administrateur/Documents/ML/Mes projet/lyon/Qualiter\PYZus{}Air/Data/Clean/LigneC.csv}\PY{l+s+s1}{\PYZsq{}}\PY{p}{,} \PY{n}{index}\PY{o}{=}\PY{k+kc}{True}\PY{p}{,} \PY{n}{sep}\PY{o}{=}\PY{l+s+s1}{\PYZsq{}}\PY{l+s+s1}{;}\PY{l+s+s1}{\PYZsq{}}\PY{p}{,} \PY{n}{header}\PY{o}{=}\PY{k+kc}{True}\PY{p}{)}
          \PY{n}{dfLigneC}\PY{o}{.}\PY{n}{head}\PY{p}{(}\PY{p}{)}
\end{Verbatim}


\begin{Verbatim}[commandchars=\\\{\}]
{\color{outcolor}Out[{\color{outcolor}728}]:}             Station  PM10   µg/ m³  Matin  Soir
          0           Hénon s             26      0     1
          1           Hénon m             29      1     0
          2    Croix-Rousse s             40      0     1
          3    Croix-Rousse m             43      1     0
          4  Hôtel de ville s             93      0     1
\end{Verbatim}
            
    \hypertarget{dfligned}{%
\subsection{dfLigneD}\label{dfligned}}

    \begin{Verbatim}[commandchars=\\\{\}]
{\color{incolor}In [{\color{incolor}729}]:} \PY{n}{dfLigneD} \PY{o}{=} \PY{n}{pd}\PY{o}{.}\PY{n}{read\PYZus{}csv}\PY{p}{(}\PY{l+s+s2}{\PYZdq{}}\PY{l+s+s2}{Data/Dirty/LigneD.csv}\PY{l+s+s2}{\PYZdq{}}\PY{p}{,} \PY{n}{sep}\PY{o}{=}\PY{l+s+s2}{\PYZdq{}}\PY{l+s+s2}{;}\PY{l+s+s2}{\PYZdq{}}\PY{p}{)}
          \PY{c+c1}{\PYZsh{}New column = Take the last letter of my column }
          \PY{n}{dfLigneD}\PY{p}{[}\PY{l+s+s2}{\PYZdq{}}\PY{l+s+s2}{Matin/Soir}\PY{l+s+s2}{\PYZdq{}}\PY{p}{]} \PY{o}{=} \PY{n}{dfLigneD}\PY{p}{[}\PY{l+s+s2}{\PYZdq{}}\PY{l+s+s2}{Station}\PY{l+s+s2}{\PYZdq{}}\PY{p}{]}\PY{o}{.}\PY{n}{str}\PY{p}{[}\PY{o}{\PYZhy{}}\PY{l+m+mi}{1}\PY{p}{:}\PY{p}{]}
          
          \PY{c+c1}{\PYZsh{}Column Matin}
          \PY{n}{dfLigneD}\PY{p}{[}\PY{l+s+s2}{\PYZdq{}}\PY{l+s+s2}{Matin}\PY{l+s+s2}{\PYZdq{}}\PY{p}{]} \PY{o}{=} \PY{n}{dfLigneD}\PY{p}{[}\PY{l+s+s2}{\PYZdq{}}\PY{l+s+s2}{Matin/Soir}\PY{l+s+s2}{\PYZdq{}}\PY{p}{]}\PY{o}{.}\PY{n}{apply}\PY{p}{(}\PY{k}{lambda} \PY{n}{x}\PY{p}{:} \PY{l+m+mi}{1} \PY{k}{if} \PY{n}{x}\PY{o}{==}\PY{l+s+s1}{\PYZsq{}}\PY{l+s+s1}{m}\PY{l+s+s1}{\PYZsq{}} \PY{k}{else} \PY{l+m+mi}{0}\PY{p}{)}
          \PY{n}{dfLigneD}\PY{p}{[}\PY{l+s+s2}{\PYZdq{}}\PY{l+s+s2}{Matin}\PY{l+s+s2}{\PYZdq{}}\PY{p}{]} \PY{o}{=} \PY{n}{dfLigneD}\PY{p}{[}\PY{l+s+s2}{\PYZdq{}}\PY{l+s+s2}{Matin}\PY{l+s+s2}{\PYZdq{}}\PY{p}{]}\PY{o}{.}\PY{n}{astype}\PY{p}{(}\PY{n+nb}{int}\PY{p}{)}
          
          \PY{c+c1}{\PYZsh{}Column Soir}
          \PY{n}{dfLigneD}\PY{p}{[}\PY{l+s+s2}{\PYZdq{}}\PY{l+s+s2}{Soir}\PY{l+s+s2}{\PYZdq{}}\PY{p}{]} \PY{o}{=} \PY{n}{dfLigneD}\PY{p}{[}\PY{l+s+s2}{\PYZdq{}}\PY{l+s+s2}{Matin/Soir}\PY{l+s+s2}{\PYZdq{}}\PY{p}{]}\PY{o}{.}\PY{n}{apply}\PY{p}{(}\PY{k}{lambda} \PY{n}{x}\PY{p}{:} \PY{l+m+mi}{1} \PY{k}{if} \PY{n}{x}\PY{o}{==}\PY{l+s+s1}{\PYZsq{}}\PY{l+s+s1}{s}\PY{l+s+s1}{\PYZsq{}} \PY{k}{else} \PY{l+m+mi}{0}\PY{p}{)}
          \PY{n}{dfLigneD}\PY{p}{[}\PY{l+s+s2}{\PYZdq{}}\PY{l+s+s2}{Soir}\PY{l+s+s2}{\PYZdq{}}\PY{p}{]} \PY{o}{=} \PY{n}{dfLigneD}\PY{p}{[}\PY{l+s+s2}{\PYZdq{}}\PY{l+s+s2}{Soir}\PY{l+s+s2}{\PYZdq{}}\PY{p}{]}\PY{o}{.}\PY{n}{astype}\PY{p}{(}\PY{n+nb}{int}\PY{p}{)}
          
          \PY{c+c1}{\PYZsh{}Column Station}
          \PY{c+c1}{\PYZsh{}dfLigneD[\PYZdq{}Station\PYZdq{}] = dfLigneD[\PYZdq{}Station\PYZdq{}].str[:\PYZhy{}1]}
          
          \PY{c+c1}{\PYZsh{}Supprimer la première ligne par l\PYZsq{}index}
          \PY{n}{dfLigneD}\PY{o}{.}\PY{n}{drop}\PY{p}{(}\PY{l+m+mi}{0}\PY{p}{,} \PY{n}{inplace} \PY{o}{=} \PY{k+kc}{True}\PY{p}{)}
          \PY{k}{del} \PY{n}{dfLigneD}\PY{p}{[}\PY{l+s+s2}{\PYZdq{}}\PY{l+s+s2}{Matin/Soir}\PY{l+s+s2}{\PYZdq{}}\PY{p}{]}
          
          \PY{c+c1}{\PYZsh{}reindex }
          \PY{n}{dfLigneD} \PY{o}{=} \PY{n}{dfLigneD}\PY{o}{.}\PY{n}{reset\PYZus{}index}\PY{p}{(}\PY{n}{drop}\PY{o}{=}\PY{k+kc}{True}\PY{p}{)}
          
          \PY{n}{dfLigneD}\PY{o}{.}\PY{n}{to\PYZus{}csv}\PY{p}{(}\PY{l+s+s1}{\PYZsq{}}\PY{l+s+s1}{/Users/Administrateur/Documents/ML/Mes projet/lyon/Qualiter\PYZus{}Air/Data/Clean/LigneD.csv}\PY{l+s+s1}{\PYZsq{}}\PY{p}{,} \PY{n}{index}\PY{o}{=}\PY{k+kc}{True}\PY{p}{,} \PY{n}{sep}\PY{o}{=}\PY{l+s+s1}{\PYZsq{}}\PY{l+s+s1}{;}\PY{l+s+s1}{\PYZsq{}}\PY{p}{,} \PY{n}{header}\PY{o}{=}\PY{k+kc}{True}\PY{p}{)}
          
          \PY{n}{dfLigneD}\PY{o}{.}\PY{n}{head}\PY{p}{(}\PY{p}{)}
\end{Verbatim}


\begin{Verbatim}[commandchars=\\\{\}]
{\color{outcolor}Out[{\color{outcolor}729}]:}            Station  PM10  µg/m³  Matin  Soir
          0  Gare de Vaise s           74      0     1
          1  Gare de Vaise m          104      1     0
          2          Valmy s          130      0     1
          3          Valmy m          129      1     0
          4  Gorge de Loup s          105      0     1
\end{Verbatim}
            
    \hypertarget{station-metro}{%
\section{Station metro}\label{station-metro}}

\hypertarget{subways-stop}{%
\section{Subway's stop}\label{subways-stop}}

    \hypertarget{station-saxe-pm10}{%
\subsubsection{Station Saxe PM10}\label{station-saxe-pm10}}

    \begin{Verbatim}[commandchars=\\\{\}]
{\color{incolor}In [{\color{incolor}730}]:} \PY{n}{dfStationSaxe\PYZus{}PM10}\PY{o}{.}\PY{n}{info}\PY{p}{(}\PY{p}{)}
          \PY{n}{dfStationSaxe\PYZus{}PM10}\PY{o}{.}\PY{n}{head}\PY{p}{(}\PY{p}{)}
\end{Verbatim}


    \begin{Verbatim}[commandchars=\\\{\}]
<class 'pandas.core.frame.DataFrame'>
RangeIndex: 164 entries, 0 to 163
Data columns (total 3 columns):
date heure                                         164 non-null object
PM10                                               164 non-null float64
Lyon Gerland (source Atmo-Auvergne-Rhône-Alpes)    164 non-null object
dtypes: float64(1), object(2)
memory usage: 3.9+ KB

    \end{Verbatim}

\begin{Verbatim}[commandchars=\\\{\}]
{\color{outcolor}Out[{\color{outcolor}730}]:}     date heure   PM10 Lyon Gerland (source Atmo-Auvergne-Rhône-Alpes)
          0   13-6 5h-6h    8.3                                               -
          1   13-6 6h-7h   51.9                                               6
          2   13-6 7h-8h  116.4                                              11
          3   13-6 8h-9h   67.0                                              18
          4  13-6 9h-10h    0.0                                              24
\end{Verbatim}
            
    \begin{Verbatim}[commandchars=\\\{\}]
{\color{incolor}In [{\color{incolor}731}]:} \PY{n}{dfStationSaxe\PYZus{}PM10}\PY{o}{.}\PY{n}{columns}
\end{Verbatim}


\begin{Verbatim}[commandchars=\\\{\}]
{\color{outcolor}Out[{\color{outcolor}731}]:} Index(['date heure', 'PM10',
                 'Lyon Gerland (source Atmo-Auvergne-Rhône-Alpes)'],
                dtype='object')
\end{Verbatim}
            
    \begin{Verbatim}[commandchars=\\\{\}]
{\color{incolor}In [{\color{incolor}732}]:} \PY{k+kn}{import} \PY{n+nn}{pandas} \PY{k}{as} \PY{n+nn}{pd}
          \PY{k+kn}{from} \PY{n+nn}{datetime} \PY{k}{import} \PY{n}{datetime}
          \PY{k+kn}{import} \PY{n+nn}{numpy} \PY{k}{as} \PY{n+nn}{np}
          
          \PY{c+c1}{\PYZsh{} Create data frame}
          \PY{n}{dfDate} \PY{o}{=} \PY{n}{pd}\PY{o}{.}\PY{n}{DataFrame}\PY{p}{(}\PY{p}{)}
          
          \PY{c+c1}{\PYZsh{} Create datetimes}
          \PY{n}{dfDate}\PY{p}{[}\PY{l+s+s1}{\PYZsq{}}\PY{l+s+s1}{Date}\PY{l+s+s1}{\PYZsq{}}\PY{p}{]} \PY{o}{=} \PY{n}{pd}\PY{o}{.}\PY{n}{date\PYZus{}range}\PY{p}{(}\PY{l+s+s1}{\PYZsq{}}\PY{l+s+s1}{13/06/2017}\PY{l+s+s1}{\PYZsq{}}\PY{p}{,} \PY{n}{periods} \PY{o}{=} \PY{l+m+mi}{170}\PY{p}{,} \PY{n}{freq} \PY{o}{=}\PY{l+s+s1}{\PYZsq{}}\PY{l+s+s1}{H}\PY{l+s+s1}{\PYZsq{}}\PY{p}{)}
          \PY{n}{dfDate} \PY{o}{=} \PY{n}{dfDate}\PY{o}{.}\PY{n}{iloc}\PY{p}{[}\PY{l+m+mi}{6}\PY{p}{:}\PY{p}{]}
          \PY{c+c1}{\PYZsh{}dfDate[\PYZsq{}data\PYZsq{}] = np.random.randint(0,100,size=(len(dfDate)))}
          
          \PY{c+c1}{\PYZsh{}add data from another df to this df}
          \PY{n}{dfDate} \PY{o}{=} \PY{n}{dfDate}\PY{o}{.}\PY{n}{reset\PYZus{}index}\PY{p}{(}\PY{n}{drop}\PY{o}{=}\PY{k+kc}{True}\PY{p}{)}
          \PY{n}{dfDate}\PY{p}{[}\PY{l+s+s1}{\PYZsq{}}\PY{l+s+s1}{PM10}\PY{l+s+s1}{\PYZsq{}}\PY{p}{]} \PY{o}{=} \PY{n}{dfStationSaxe\PYZus{}PM10}\PY{p}{[}\PY{l+s+s2}{\PYZdq{}}\PY{l+s+s2}{PM10}\PY{l+s+s2}{\PYZdq{}}\PY{p}{]}
          \PY{n}{dfDate}\PY{p}{[}\PY{l+s+s1}{\PYZsq{}}\PY{l+s+s1}{Lyon Gerland (source Atmo\PYZhy{}Auvergne\PYZhy{}Rhône\PYZhy{}Alpes)}\PY{l+s+s1}{\PYZsq{}}\PY{p}{]}\PY{o}{=} \PY{n}{dfStationSaxe\PYZus{}PM10}\PY{p}{[}\PY{l+s+s1}{\PYZsq{}}\PY{l+s+s1}{Lyon Gerland (source Atmo\PYZhy{}Auvergne\PYZhy{}Rhône\PYZhy{}Alpes)}\PY{l+s+s1}{\PYZsq{}}\PY{p}{]}
          
          \PY{c+c1}{\PYZsh{}replace some missing values}
          \PY{c+c1}{\PYZsh{}remplacer le mots \PYZdq{}Date\PYZdq{} en espace dans la column Date}
          \PY{n}{dfDate}\PY{p}{[}\PY{l+s+s1}{\PYZsq{}}\PY{l+s+s1}{Lyon Gerland (source Atmo\PYZhy{}Auvergne\PYZhy{}Rhône\PYZhy{}Alpes)}\PY{l+s+s1}{\PYZsq{}}\PY{p}{]}\PY{o}{.}\PY{n}{replace}\PY{p}{(}\PY{l+s+s1}{\PYZsq{}}\PY{l+s+s1}{\PYZhy{}}\PY{l+s+s1}{\PYZsq{}}\PY{p}{,} \PY{l+s+s1}{\PYZsq{}}\PY{l+s+s1}{\PYZsq{}}\PY{p}{,} \PY{n}{inplace}\PY{o}{=}\PY{k+kc}{True}\PY{p}{)}
          
          \PY{c+c1}{\PYZsh{}remplacer les espace vide par missing values:}
          \PY{n}{dfDate}\PY{o}{.}\PY{n}{replace}\PY{p}{(}\PY{l+s+s1}{\PYZsq{}}\PY{l+s+s1}{\PYZsq{}}\PY{p}{,} \PY{n}{np}\PY{o}{.}\PY{n}{nan}\PY{p}{,} \PY{n}{inplace}\PY{o}{=}\PY{k+kc}{True}\PY{p}{)}
          
          \PY{c+c1}{\PYZsh{}Mettre en index }
          \PY{n}{dfDate}\PY{o}{.}\PY{n}{set\PYZus{}index}\PY{p}{(}\PY{l+s+s1}{\PYZsq{}}\PY{l+s+s1}{Date}\PY{l+s+s1}{\PYZsq{}}\PY{p}{,} \PY{n}{inplace}\PY{o}{=}\PY{k+kc}{True}\PY{p}{)}
          
          \PY{n}{dfStationSaxe\PYZus{}PM10} \PY{o}{=} \PY{n}{dfDate}
          
          \PY{n}{dfStationSaxe\PYZus{}PM10}\PY{o}{.}\PY{n}{to\PYZus{}csv}\PY{p}{(}\PY{l+s+s1}{\PYZsq{}}\PY{l+s+s1}{/Users/Administrateur/Documents/ML/Mes projet/lyon/Qualiter\PYZus{}Air/Data/Clean/Station Saxe\PYZhy{}Gambetta PM10.csv}\PY{l+s+s1}{\PYZsq{}}\PY{p}{,} \PY{n}{index}\PY{o}{=}\PY{k+kc}{True}\PY{p}{,} \PY{n}{sep}\PY{o}{=}\PY{l+s+s1}{\PYZsq{}}\PY{l+s+s1}{;}\PY{l+s+s1}{\PYZsq{}}\PY{p}{,} \PY{n}{header}\PY{o}{=}\PY{k+kc}{True}\PY{p}{)}
          
          \PY{n}{dfStationSaxe\PYZus{}PM10}\PY{o}{.}\PY{n}{info}\PY{p}{(}\PY{p}{)}
          \PY{n}{dfStationSaxe\PYZus{}PM10}\PY{o}{.}\PY{n}{head}\PY{p}{(}\PY{p}{)}
\end{Verbatim}


    \begin{Verbatim}[commandchars=\\\{\}]
<class 'pandas.core.frame.DataFrame'>
DatetimeIndex: 164 entries, 2017-06-13 06:00:00 to 2017-06-20 01:00:00
Data columns (total 2 columns):
PM10                                               164 non-null float64
Lyon Gerland (source Atmo-Auvergne-Rhône-Alpes)    159 non-null object
dtypes: float64(1), object(1)
memory usage: 3.8+ KB

    \end{Verbatim}

\begin{Verbatim}[commandchars=\\\{\}]
{\color{outcolor}Out[{\color{outcolor}732}]:}                       PM10 Lyon Gerland (source Atmo-Auvergne-Rhône-Alpes)
          Date                                                                      
          2017-06-13 06:00:00    8.3                                             NaN
          2017-06-13 07:00:00   51.9                                               6
          2017-06-13 08:00:00  116.4                                              11
          2017-06-13 09:00:00   67.0                                              18
          2017-06-13 10:00:00    0.0                                              24
\end{Verbatim}
            
    \begin{Verbatim}[commandchars=\\\{\}]
{\color{incolor}In [{\color{incolor}733}]:} \PY{n}{dfStationSaxe\PYZus{}PM10}\PY{o}{.}\PY{n}{isnull}\PY{p}{(}\PY{p}{)}\PY{o}{.}\PY{n}{sum}\PY{p}{(}\PY{p}{)}
\end{Verbatim}


\begin{Verbatim}[commandchars=\\\{\}]
{\color{outcolor}Out[{\color{outcolor}733}]:} PM10                                               0
          Lyon Gerland (source Atmo-Auvergne-Rhône-Alpes)    5
          dtype: int64
\end{Verbatim}
            
    \hypertarget{station-saxe-pm2.5}{%
\subsubsection{Station Saxe PM2.5}\label{station-saxe-pm2.5}}

    \begin{Verbatim}[commandchars=\\\{\}]
{\color{incolor}In [{\color{incolor}734}]:} \PY{n}{dfStationSaxePM2\PYZus{}5}\PY{o}{.}\PY{n}{info}\PY{p}{(}\PY{p}{)}
          \PY{n}{dfStationSaxePM2\PYZus{}5}\PY{o}{.}\PY{n}{head}\PY{p}{(}\PY{p}{)}
\end{Verbatim}


    \begin{Verbatim}[commandchars=\\\{\}]
<class 'pandas.core.frame.DataFrame'>
RangeIndex: 167 entries, 0 to 166
Data columns (total 3 columns):
Date et heure                      167 non-null object
Moyenne horaire PM2.5 en µg/m3     167 non-null float64
Lyon Centre                        167 non-null object
dtypes: float64(1), object(2)
memory usage: 4.0+ KB

    \end{Verbatim}

\begin{Verbatim}[commandchars=\\\{\}]
{\color{outcolor}Out[{\color{outcolor}734}]:}   Date et heure  Moyenne horaire PM2.5 en µg/m3  Lyon Centre
          0    20-6 2h-3h                             15.8           -
          1    20-6 3h-4h                             29.1           -
          2    20-6 4h-5h                             17.7           -
          3    20-6 5h-6h                             18.7           -
          4    20-6 6h-7h                             28.0           -
\end{Verbatim}
            
    \begin{Verbatim}[commandchars=\\\{\}]
{\color{incolor}In [{\color{incolor}735}]:} \PY{n}{dfStationSaxePM2\PYZus{}5}\PY{o}{.}\PY{n}{columns}
\end{Verbatim}


\begin{Verbatim}[commandchars=\\\{\}]
{\color{outcolor}Out[{\color{outcolor}735}]:} Index(['Date et heure', 'Moyenne horaire PM2.5 en µg/m3 ', 'Lyon Centre'], dtype='object')
\end{Verbatim}
            
    \begin{Verbatim}[commandchars=\\\{\}]
{\color{incolor}In [{\color{incolor}736}]:} \PY{n}{dfStationSaxePM2\PYZus{}5} \PY{o}{=} \PY{n}{pd}\PY{o}{.}\PY{n}{read\PYZus{}csv}\PY{p}{(}\PY{l+s+s2}{\PYZdq{}}\PY{l+s+s2}{Data/Dirty/Station Saxe\PYZhy{}Gambetta PM2.5.csv}\PY{l+s+s2}{\PYZdq{}}\PY{p}{,} \PY{n}{sep}\PY{o}{=}\PY{l+s+s2}{\PYZdq{}}\PY{l+s+s2}{;}\PY{l+s+s2}{\PYZdq{}}\PY{p}{)}
          \PY{c+c1}{\PYZsh{} Create data frame}
          \PY{n}{dfDate2} \PY{o}{=} \PY{n}{pd}\PY{o}{.}\PY{n}{DataFrame}\PY{p}{(}\PY{p}{)}
          
          \PY{c+c1}{\PYZsh{} Create datetimes}
          \PY{n}{dfDate2}\PY{p}{[}\PY{l+s+s1}{\PYZsq{}}\PY{l+s+s1}{Date}\PY{l+s+s1}{\PYZsq{}}\PY{p}{]} \PY{o}{=} \PY{n}{pd}\PY{o}{.}\PY{n}{date\PYZus{}range}\PY{p}{(}\PY{l+s+s1}{\PYZsq{}}\PY{l+s+s1}{20/06/2017}\PY{l+s+s1}{\PYZsq{}}\PY{p}{,} \PY{n}{periods} \PY{o}{=} \PY{l+m+mi}{170}\PY{p}{,} \PY{n}{freq} \PY{o}{=}\PY{l+s+s1}{\PYZsq{}}\PY{l+s+s1}{H}\PY{l+s+s1}{\PYZsq{}}\PY{p}{)}
          \PY{n}{dfDate2} \PY{o}{=} \PY{n}{dfDate2}\PY{o}{.}\PY{n}{iloc}\PY{p}{[}\PY{l+m+mi}{3}\PY{p}{:}\PY{p}{]}
          
          
          \PY{c+c1}{\PYZsh{}add data from another df to this df}
          \PY{n}{dfDate2} \PY{o}{=} \PY{n}{dfDate2}\PY{o}{.}\PY{n}{reset\PYZus{}index}\PY{p}{(}\PY{n}{drop}\PY{o}{=}\PY{k+kc}{True}\PY{p}{)}
          \PY{n}{dfDate2}\PY{p}{[}\PY{l+s+s1}{\PYZsq{}}\PY{l+s+s1}{Moyenne horaire PM2.5 en µg/m3}\PY{l+s+s1}{\PYZsq{}}\PY{p}{]}\PY{o}{=} \PY{n}{dfStationSaxePM2\PYZus{}5}\PY{p}{[}\PY{l+s+s1}{\PYZsq{}}\PY{l+s+s1}{Moyenne horaire PM2.5 en µg/m3 }\PY{l+s+s1}{\PYZsq{}}\PY{p}{]}
          \PY{n}{dfDate2}\PY{p}{[}\PY{l+s+s1}{\PYZsq{}}\PY{l+s+s1}{Lyon Centre}\PY{l+s+s1}{\PYZsq{}}\PY{p}{]} \PY{o}{=}  \PY{n}{dfStationSaxePM2\PYZus{}5}\PY{p}{[}\PY{l+s+s1}{\PYZsq{}}\PY{l+s+s1}{Lyon Centre}\PY{l+s+s1}{\PYZsq{}}\PY{p}{]}
          
          \PY{c+c1}{\PYZsh{}replace some missing values}
          \PY{c+c1}{\PYZsh{}remplacer le mots \PYZdq{}Date\PYZdq{} en espace dans la column Date}
          \PY{n}{dfDate2}\PY{p}{[}\PY{l+s+s1}{\PYZsq{}}\PY{l+s+s1}{Lyon Centre}\PY{l+s+s1}{\PYZsq{}}\PY{p}{]}\PY{o}{.}\PY{n}{replace}\PY{p}{(}\PY{l+s+s1}{\PYZsq{}}\PY{l+s+s1}{\PYZhy{}}\PY{l+s+s1}{\PYZsq{}}\PY{p}{,} \PY{l+s+s1}{\PYZsq{}}\PY{l+s+s1}{\PYZsq{}}\PY{p}{,} \PY{n}{inplace}\PY{o}{=}\PY{k+kc}{True}\PY{p}{)}
          
          \PY{c+c1}{\PYZsh{}remplacer les espace vide par missing values:}
          \PY{n}{dfDate2}\PY{o}{.}\PY{n}{replace}\PY{p}{(}\PY{l+s+s1}{\PYZsq{}}\PY{l+s+s1}{\PYZsq{}}\PY{p}{,} \PY{n}{np}\PY{o}{.}\PY{n}{nan}\PY{p}{,} \PY{n}{inplace}\PY{o}{=}\PY{k+kc}{True}\PY{p}{)}
          
          \PY{c+c1}{\PYZsh{}Mettre en index }
          \PY{n}{dfDate2}\PY{o}{.}\PY{n}{set\PYZus{}index}\PY{p}{(}\PY{l+s+s1}{\PYZsq{}}\PY{l+s+s1}{Date}\PY{l+s+s1}{\PYZsq{}}\PY{p}{,} \PY{n}{inplace}\PY{o}{=}\PY{k+kc}{True}\PY{p}{)}
          
          \PY{n}{dfStationSaxePM2\PYZus{}5} \PY{o}{=} \PY{n}{dfDate2}
          
          \PY{n}{dfStationSaxePM2\PYZus{}5}\PY{o}{.}\PY{n}{to\PYZus{}csv}\PY{p}{(}\PY{l+s+s1}{\PYZsq{}}\PY{l+s+s1}{/Users/Administrateur/Documents/ML/Mes projet/lyon/Qualiter\PYZus{}Air/Data/Clean/Station Saxe\PYZhy{}Gambetta PM2.5.csv}\PY{l+s+s1}{\PYZsq{}}\PY{p}{,} \PY{n}{index}\PY{o}{=}\PY{k+kc}{True}\PY{p}{,} \PY{n}{sep}\PY{o}{=}\PY{l+s+s1}{\PYZsq{}}\PY{l+s+s1}{;}\PY{l+s+s1}{\PYZsq{}}\PY{p}{,} \PY{n}{header}\PY{o}{=}\PY{k+kc}{True}\PY{p}{)}
          
          \PY{n}{dfStationSaxePM2\PYZus{}5}\PY{o}{.}\PY{n}{info}\PY{p}{(}\PY{p}{)}
          \PY{n}{dfStationSaxePM2\PYZus{}5}\PY{o}{.}\PY{n}{head}\PY{p}{(}\PY{p}{)}
\end{Verbatim}


    \begin{Verbatim}[commandchars=\\\{\}]
<class 'pandas.core.frame.DataFrame'>
DatetimeIndex: 167 entries, 2017-06-20 03:00:00 to 2017-06-27 01:00:00
Data columns (total 2 columns):
Moyenne horaire PM2.5 en µg/m3    167 non-null float64
Lyon Centre                       84 non-null object
dtypes: float64(1), object(1)
memory usage: 3.9+ KB

    \end{Verbatim}

\begin{Verbatim}[commandchars=\\\{\}]
{\color{outcolor}Out[{\color{outcolor}736}]:}                      Moyenne horaire PM2.5 en µg/m3 Lyon Centre
          Date                                                           
          2017-06-20 03:00:00                            15.8         NaN
          2017-06-20 04:00:00                            29.1         NaN
          2017-06-20 05:00:00                            17.7         NaN
          2017-06-20 06:00:00                            18.7         NaN
          2017-06-20 07:00:00                            28.0         NaN
\end{Verbatim}
            
    \hypertarget{station-du-finiculaire}{%
\subsection{Station du Finiculaire}\label{station-du-finiculaire}}

\hypertarget{finiculars-stop}{%
\subsection{Finicular's Stop}\label{finiculars-stop}}

    \hypertarget{saint-jus}{%
\subsubsection{Saint Jus}\label{saint-jus}}

    \begin{Verbatim}[commandchars=\\\{\}]
{\color{incolor}In [{\color{incolor}737}]:} \PY{n}{dfFuniculaireSaintJus}\PY{o}{.}\PY{n}{info}\PY{p}{(}\PY{p}{)}
          \PY{n}{dfFuniculaireSaintJus}
\end{Verbatim}


    \begin{Verbatim}[commandchars=\\\{\}]
<class 'pandas.core.frame.DataFrame'>
RangeIndex: 5 entries, 0 to 4
Data columns (total 2 columns):
Station       5 non-null object
PM10 µg/m³    5 non-null int64
dtypes: int64(1), object(1)
memory usage: 160.0+ bytes

    \end{Verbatim}

\begin{Verbatim}[commandchars=\\\{\}]
{\color{outcolor}Out[{\color{outcolor}737}]:}               Station  PM10 µg/m³
          0             Moyenne          50
          1  Vieux Lyon St Ju s          26
          2  Vieux Lyon St Ju m         111
          3        Saint Just s          26
          4        Saint Just m          40
\end{Verbatim}
            
    \begin{Verbatim}[commandchars=\\\{\}]
{\color{incolor}In [{\color{incolor}738}]:} \PY{n}{dfFuniculaireSaintJus} \PY{o}{=} \PY{n}{pd}\PY{o}{.}\PY{n}{read\PYZus{}csv}\PY{p}{(}\PY{l+s+s2}{\PYZdq{}}\PY{l+s+s2}{Data/Dirty/Funiculaire Saint\PYZhy{}Just.csv}\PY{l+s+s2}{\PYZdq{}}\PY{p}{,} \PY{n}{sep}\PY{o}{=}\PY{l+s+s2}{\PYZdq{}}\PY{l+s+s2}{;}\PY{l+s+s2}{\PYZdq{}}\PY{p}{)}
          
          \PY{c+c1}{\PYZsh{}New column = Take the last letter of my column }
          \PY{n}{dfFuniculaireSaintJus}\PY{p}{[}\PY{l+s+s2}{\PYZdq{}}\PY{l+s+s2}{Matin/Soir}\PY{l+s+s2}{\PYZdq{}}\PY{p}{]} \PY{o}{=} \PY{n}{dfFuniculaireSaintJus}\PY{p}{[}\PY{l+s+s2}{\PYZdq{}}\PY{l+s+s2}{Station}\PY{l+s+s2}{\PYZdq{}}\PY{p}{]}\PY{o}{.}\PY{n}{str}\PY{p}{[}\PY{o}{\PYZhy{}}\PY{l+m+mi}{1}\PY{p}{:}\PY{p}{]}
          
          \PY{c+c1}{\PYZsh{}Column Matin}
          \PY{n}{dfFuniculaireSaintJus}\PY{p}{[}\PY{l+s+s2}{\PYZdq{}}\PY{l+s+s2}{Matin}\PY{l+s+s2}{\PYZdq{}}\PY{p}{]} \PY{o}{=} \PY{n}{dfFuniculaireSaintJus}\PY{p}{[}\PY{l+s+s2}{\PYZdq{}}\PY{l+s+s2}{Matin/Soir}\PY{l+s+s2}{\PYZdq{}}\PY{p}{]}\PY{o}{.}\PY{n}{apply}\PY{p}{(}\PY{k}{lambda} \PY{n}{x}\PY{p}{:} \PY{l+m+mi}{1} \PY{k}{if} \PY{n}{x}\PY{o}{==}\PY{l+s+s1}{\PYZsq{}}\PY{l+s+s1}{m}\PY{l+s+s1}{\PYZsq{}} \PY{k}{else} \PY{l+m+mi}{0}\PY{p}{)}
          \PY{n}{dfFuniculaireSaintJus}\PY{p}{[}\PY{l+s+s2}{\PYZdq{}}\PY{l+s+s2}{Matin}\PY{l+s+s2}{\PYZdq{}}\PY{p}{]} \PY{o}{=} \PY{n}{dfFuniculaireSaintJus}\PY{p}{[}\PY{l+s+s2}{\PYZdq{}}\PY{l+s+s2}{Matin}\PY{l+s+s2}{\PYZdq{}}\PY{p}{]}\PY{o}{.}\PY{n}{astype}\PY{p}{(}\PY{n+nb}{int}\PY{p}{)}
          
          \PY{c+c1}{\PYZsh{}Column Soir}
          \PY{n}{dfFuniculaireSaintJus}\PY{p}{[}\PY{l+s+s2}{\PYZdq{}}\PY{l+s+s2}{Soir}\PY{l+s+s2}{\PYZdq{}}\PY{p}{]} \PY{o}{=} \PY{n}{dfFuniculaireSaintJus}\PY{p}{[}\PY{l+s+s2}{\PYZdq{}}\PY{l+s+s2}{Matin/Soir}\PY{l+s+s2}{\PYZdq{}}\PY{p}{]}\PY{o}{.}\PY{n}{apply}\PY{p}{(}\PY{k}{lambda} \PY{n}{x}\PY{p}{:} \PY{l+m+mi}{1} \PY{k}{if} \PY{n}{x}\PY{o}{==}\PY{l+s+s1}{\PYZsq{}}\PY{l+s+s1}{s}\PY{l+s+s1}{\PYZsq{}} \PY{k}{else} \PY{l+m+mi}{0}\PY{p}{)}
          \PY{n}{dfFuniculaireSaintJus}\PY{p}{[}\PY{l+s+s2}{\PYZdq{}}\PY{l+s+s2}{Soir}\PY{l+s+s2}{\PYZdq{}}\PY{p}{]} \PY{o}{=} \PY{n}{dfFuniculaireSaintJus}\PY{p}{[}\PY{l+s+s2}{\PYZdq{}}\PY{l+s+s2}{Soir}\PY{l+s+s2}{\PYZdq{}}\PY{p}{]}\PY{o}{.}\PY{n}{astype}\PY{p}{(}\PY{n+nb}{int}\PY{p}{)}
          
          
          
          \PY{c+c1}{\PYZsh{}Supprimer la première ligne par l\PYZsq{}index}
          \PY{n}{dfFuniculaireSaintJus}\PY{o}{.}\PY{n}{drop}\PY{p}{(}\PY{l+m+mi}{0}\PY{p}{,} \PY{n}{inplace} \PY{o}{=} \PY{k+kc}{True}\PY{p}{)}
          \PY{k}{del} \PY{n}{dfFuniculaireSaintJus}\PY{p}{[}\PY{l+s+s2}{\PYZdq{}}\PY{l+s+s2}{Matin/Soir}\PY{l+s+s2}{\PYZdq{}}\PY{p}{]}
          
          \PY{c+c1}{\PYZsh{}reindex }
          \PY{n}{dfFuniculaireSaintJus} \PY{o}{=} \PY{n}{dfFuniculaireSaintJus}\PY{o}{.}\PY{n}{reset\PYZus{}index}\PY{p}{(}\PY{n}{drop}\PY{o}{=}\PY{k+kc}{True}\PY{p}{)}
          
          \PY{n}{dfFuniculaireSaintJus}\PY{o}{.}\PY{n}{to\PYZus{}csv}\PY{p}{(}\PY{l+s+s1}{\PYZsq{}}\PY{l+s+s1}{/Users/Administrateur/Documents/ML/Mes projet/lyon/Qualiter\PYZus{}Air/Data/Clean/Funiculaire Saint\PYZhy{}Just.csv}\PY{l+s+s1}{\PYZsq{}}\PY{p}{,} \PY{n}{index}\PY{o}{=}\PY{k+kc}{True}\PY{p}{,} \PY{n}{sep}\PY{o}{=}\PY{l+s+s1}{\PYZsq{}}\PY{l+s+s1}{;}\PY{l+s+s1}{\PYZsq{}}\PY{p}{,} \PY{n}{header}\PY{o}{=}\PY{k+kc}{True}\PY{p}{)}
          \PY{n}{dfFuniculaireSaintJus}
\end{Verbatim}


\begin{Verbatim}[commandchars=\\\{\}]
{\color{outcolor}Out[{\color{outcolor}738}]:}               Station  PM10 µg/m³  Matin  Soir
          0  Vieux Lyon St Ju s          26      0     1
          1  Vieux Lyon St Ju m         111      1     0
          2        Saint Just s          26      0     1
          3        Saint Just m          40      1     0
\end{Verbatim}
            
    \hypertarget{fourviuxe8re}{%
\subsubsection{Fourvière}\label{fourviuxe8re}}

    \begin{Verbatim}[commandchars=\\\{\}]
{\color{incolor}In [{\color{incolor}739}]:} \PY{n}{dfFuniculaireFourviere}\PY{o}{.}\PY{n}{info}\PY{p}{(}\PY{p}{)}
          \PY{n}{dfFuniculaireFourviere}
\end{Verbatim}


    \begin{Verbatim}[commandchars=\\\{\}]
<class 'pandas.core.frame.DataFrame'>
RangeIndex: 5 entries, 0 to 4
Data columns (total 2 columns):
Station       5 non-null object
PM10 µg/m³    5 non-null int64
dtypes: int64(1), object(1)
memory usage: 160.0+ bytes

    \end{Verbatim}

\begin{Verbatim}[commandchars=\\\{\}]
{\color{outcolor}Out[{\color{outcolor}739}]:}                   Station  PM10 µg/m³
          0                 Moyenne          38
          1             Fourvière s          22
          2             Fourvière m          22
          3  Vieux Lyon Fourvière s          30
          4  Vieux Lyon Fourvière m          81
\end{Verbatim}
            
    \begin{Verbatim}[commandchars=\\\{\}]
{\color{incolor}In [{\color{incolor}740}]:} \PY{n}{dfFuniculaireFourviere} \PY{o}{=} \PY{n}{pd}\PY{o}{.}\PY{n}{read\PYZus{}csv}\PY{p}{(}\PY{l+s+s2}{\PYZdq{}}\PY{l+s+s2}{Data/Dirty/Funiculaire Fourviere.csv}\PY{l+s+s2}{\PYZdq{}}\PY{p}{,} \PY{n}{sep}\PY{o}{=}\PY{l+s+s2}{\PYZdq{}}\PY{l+s+s2}{;}\PY{l+s+s2}{\PYZdq{}}\PY{p}{)}
          
          
          \PY{c+c1}{\PYZsh{}New column = Take the last letter of my column }
          \PY{n}{dfFuniculaireFourviere}\PY{p}{[}\PY{l+s+s2}{\PYZdq{}}\PY{l+s+s2}{Matin/Soir}\PY{l+s+s2}{\PYZdq{}}\PY{p}{]} \PY{o}{=} \PY{n}{dfFuniculaireFourviere}\PY{p}{[}\PY{l+s+s2}{\PYZdq{}}\PY{l+s+s2}{Station}\PY{l+s+s2}{\PYZdq{}}\PY{p}{]}\PY{o}{.}\PY{n}{str}\PY{p}{[}\PY{o}{\PYZhy{}}\PY{l+m+mi}{1}\PY{p}{:}\PY{p}{]}
          
          \PY{c+c1}{\PYZsh{}Column Matin}
          \PY{n}{dfFuniculaireFourviere}\PY{p}{[}\PY{l+s+s2}{\PYZdq{}}\PY{l+s+s2}{Matin}\PY{l+s+s2}{\PYZdq{}}\PY{p}{]} \PY{o}{=} \PY{n}{dfFuniculaireFourviere}\PY{p}{[}\PY{l+s+s2}{\PYZdq{}}\PY{l+s+s2}{Matin/Soir}\PY{l+s+s2}{\PYZdq{}}\PY{p}{]}\PY{o}{.}\PY{n}{apply}\PY{p}{(}\PY{k}{lambda} \PY{n}{x}\PY{p}{:} \PY{l+m+mi}{1} \PY{k}{if} \PY{n}{x}\PY{o}{==}\PY{l+s+s1}{\PYZsq{}}\PY{l+s+s1}{m}\PY{l+s+s1}{\PYZsq{}} \PY{k}{else} \PY{l+m+mi}{0}\PY{p}{)}
          \PY{n}{dfFuniculaireFourviere}\PY{p}{[}\PY{l+s+s2}{\PYZdq{}}\PY{l+s+s2}{Matin}\PY{l+s+s2}{\PYZdq{}}\PY{p}{]} \PY{o}{=} \PY{n}{dfFuniculaireFourviere}\PY{p}{[}\PY{l+s+s2}{\PYZdq{}}\PY{l+s+s2}{Matin}\PY{l+s+s2}{\PYZdq{}}\PY{p}{]}\PY{o}{.}\PY{n}{astype}\PY{p}{(}\PY{n+nb}{int}\PY{p}{)}
          
          \PY{c+c1}{\PYZsh{}Column Soir}
          \PY{n}{dfFuniculaireFourviere}\PY{p}{[}\PY{l+s+s2}{\PYZdq{}}\PY{l+s+s2}{Soir}\PY{l+s+s2}{\PYZdq{}}\PY{p}{]} \PY{o}{=} \PY{n}{dfFuniculaireFourviere}\PY{p}{[}\PY{l+s+s2}{\PYZdq{}}\PY{l+s+s2}{Matin/Soir}\PY{l+s+s2}{\PYZdq{}}\PY{p}{]}\PY{o}{.}\PY{n}{apply}\PY{p}{(}\PY{k}{lambda} \PY{n}{x}\PY{p}{:} \PY{l+m+mi}{1} \PY{k}{if} \PY{n}{x}\PY{o}{==}\PY{l+s+s1}{\PYZsq{}}\PY{l+s+s1}{s}\PY{l+s+s1}{\PYZsq{}} \PY{k}{else} \PY{l+m+mi}{0}\PY{p}{)}
          \PY{n}{dfFuniculaireFourviere}\PY{p}{[}\PY{l+s+s2}{\PYZdq{}}\PY{l+s+s2}{Soir}\PY{l+s+s2}{\PYZdq{}}\PY{p}{]} \PY{o}{=} \PY{n}{dfFuniculaireFourviere}\PY{p}{[}\PY{l+s+s2}{\PYZdq{}}\PY{l+s+s2}{Soir}\PY{l+s+s2}{\PYZdq{}}\PY{p}{]}\PY{o}{.}\PY{n}{astype}\PY{p}{(}\PY{n+nb}{int}\PY{p}{)}
          
          
          \PY{c+c1}{\PYZsh{}Supprimer la première ligne par l\PYZsq{}index}
          \PY{n}{dfFuniculaireFourviere}\PY{o}{.}\PY{n}{drop}\PY{p}{(}\PY{l+m+mi}{0}\PY{p}{,} \PY{n}{inplace} \PY{o}{=} \PY{k+kc}{True}\PY{p}{)}
          \PY{k}{del} \PY{n}{dfFuniculaireFourviere}\PY{p}{[}\PY{l+s+s2}{\PYZdq{}}\PY{l+s+s2}{Matin/Soir}\PY{l+s+s2}{\PYZdq{}}\PY{p}{]}
          
          \PY{c+c1}{\PYZsh{}reindex }
          \PY{n}{dfFuniculaireFourviere} \PY{o}{=} \PY{n}{dfFuniculaireFourviere}\PY{o}{.}\PY{n}{reset\PYZus{}index}\PY{p}{(}\PY{n}{drop}\PY{o}{=}\PY{k+kc}{True}\PY{p}{)}
          
          \PY{n}{dfFuniculaireFourviere}\PY{o}{.}\PY{n}{to\PYZus{}csv}\PY{p}{(}\PY{l+s+s1}{\PYZsq{}}\PY{l+s+s1}{/Users/Administrateur/Documents/ML/Mes projet/lyon/Qualiter\PYZus{}Air/Data/Clean/Funiculaire Fourviere.csv}\PY{l+s+s1}{\PYZsq{}}\PY{p}{,} \PY{n}{index}\PY{o}{=}\PY{k+kc}{True}\PY{p}{,} \PY{n}{sep}\PY{o}{=}\PY{l+s+s1}{\PYZsq{}}\PY{l+s+s1}{;}\PY{l+s+s1}{\PYZsq{}}\PY{p}{,} \PY{n}{header}\PY{o}{=}\PY{k+kc}{True}\PY{p}{)}
          \PY{n}{dfFuniculaireFourviere}
\end{Verbatim}


\begin{Verbatim}[commandchars=\\\{\}]
{\color{outcolor}Out[{\color{outcolor}740}]:}                   Station  PM10 µg/m³  Matin  Soir
          0             Fourvière s          22      0     1
          1             Fourvière m          22      1     0
          2  Vieux Lyon Fourvière s          30      0     1
          3  Vieux Lyon Fourvière m          81      1     0
\end{Verbatim}
            
    \hypertarget{analyses}{%
\section{Analyses}\label{analyses}}

    2017, la valeur de référence pour Lyon est de 226 µg/m³". Les mesures de
mars 2017 en station (mesure de 15 minutes des PM10)

2017, the reference value for Lyon is 226 µg/m³". March 2017
measurements in stations (15-minute measurement of PM10)

    \begin{Verbatim}[commandchars=\\\{\}]
{\color{incolor}In [{\color{incolor}741}]:} \PY{c+c1}{\PYZsh{} visualization libraries}
          \PY{k+kn}{import} \PY{n+nn}{matplotlib}\PY{n+nn}{.}\PY{n+nn}{pyplot} \PY{k}{as} \PY{n+nn}{plt}
          \PY{k+kn}{import} \PY{n+nn}{seaborn} \PY{k}{as} \PY{n+nn}{sns}
          
          
          \PY{c+c1}{\PYZsh{} print the graphs in the notebook}
          \PY{o}{\PYZpc{}}\PY{k}{matplotlib} inline
          
          \PY{c+c1}{\PYZsh{} set seaborn style to white}
          \PY{n}{sns}\PY{o}{.}\PY{n}{set\PYZus{}style}\PY{p}{(}\PY{l+s+s2}{\PYZdq{}}\PY{l+s+s2}{white}\PY{l+s+s2}{\PYZdq{}}\PY{p}{)}
\end{Verbatim}


    \hypertarget{ligne-a}{%
\subsection{Ligne A}\label{ligne-a}}

    \begin{Verbatim}[commandchars=\\\{\}]
{\color{incolor}In [{\color{incolor}742}]:} \PY{n}{plt}\PY{o}{.}\PY{n}{rcdefaults}\PY{p}{(}\PY{p}{)}
          \PY{n}{fig}\PY{p}{,} \PY{n}{ax} \PY{o}{=} \PY{n}{plt}\PY{o}{.}\PY{n}{subplots}\PY{p}{(}\PY{p}{)}
          
          
          \PY{n}{people} \PY{o}{=} \PY{n}{dfLigneA}\PY{p}{[}\PY{l+s+s1}{\PYZsq{}}\PY{l+s+s1}{Station}\PY{l+s+s1}{\PYZsq{}}\PY{p}{]}
          \PY{n}{y\PYZus{}pos} \PY{o}{=} \PY{n}{np}\PY{o}{.}\PY{n}{arange}\PY{p}{(}\PY{n+nb}{len}\PY{p}{(}\PY{n}{people}\PY{p}{)}\PY{p}{)}
          
          
          \PY{n}{performance} \PY{o}{=} \PY{n}{dfLigneA}\PY{p}{[}\PY{l+s+s1}{\PYZsq{}}\PY{l+s+s1}{PM10 µg/m³}\PY{l+s+s1}{\PYZsq{}}\PY{p}{]}
          
          \PY{n}{error} \PY{o}{=} \PY{n}{np}\PY{o}{.}\PY{n}{random}\PY{o}{.}\PY{n}{rand}\PY{p}{(}\PY{n+nb}{len}\PY{p}{(}\PY{n}{people}\PY{p}{)}\PY{p}{)}
          
          \PY{n}{ax}\PY{o}{.}\PY{n}{barh}\PY{p}{(}\PY{n}{y\PYZus{}pos}\PY{p}{,} \PY{n}{performance}\PY{p}{,} \PY{n}{xerr}\PY{o}{=}\PY{n}{error}\PY{p}{,} \PY{n}{align}\PY{o}{=}\PY{l+s+s1}{\PYZsq{}}\PY{l+s+s1}{center}\PY{l+s+s1}{\PYZsq{}}\PY{p}{,}
                  \PY{n}{color}\PY{o}{=}\PY{l+s+s1}{\PYZsq{}}\PY{l+s+s1}{red}\PY{l+s+s1}{\PYZsq{}}\PY{p}{,} \PY{n}{ecolor}\PY{o}{=}\PY{l+s+s1}{\PYZsq{}}\PY{l+s+s1}{black}\PY{l+s+s1}{\PYZsq{}}\PY{p}{)}
          \PY{n}{ax}\PY{o}{.}\PY{n}{set\PYZus{}yticks}\PY{p}{(}\PY{n}{y\PYZus{}pos}\PY{p}{)}
          \PY{n}{ax}\PY{o}{.}\PY{n}{set\PYZus{}yticklabels}\PY{p}{(}\PY{n}{people}\PY{p}{)}
          \PY{n}{ax}\PY{o}{.}\PY{n}{invert\PYZus{}yaxis}\PY{p}{(}\PY{p}{)}  \PY{c+c1}{\PYZsh{} labels read top\PYZhy{}to\PYZhy{}bottom}
          \PY{n}{ax}\PY{o}{.}\PY{n}{set\PYZus{}xlabel}\PY{p}{(}\PY{l+s+s1}{\PYZsq{}}\PY{l+s+s1}{PM10 µg/m³}\PY{l+s+s1}{\PYZsq{}}\PY{p}{)}
          \PY{n}{ax}\PY{o}{.}\PY{n}{set\PYZus{}title}\PY{p}{(}\PY{l+s+s1}{\PYZsq{}}\PY{l+s+s1}{Ligne A (m = matin, s = soir)}\PY{l+s+s1}{\PYZsq{}}\PY{p}{)}
          
          \PY{n}{plt}\PY{o}{.}\PY{n}{show}\PY{p}{(}\PY{p}{)}
\end{Verbatim}


    \begin{center}
    \adjustimage{max size={0.9\linewidth}{0.9\paperheight}}{output_41_0.png}
    \end{center}
    { \hspace*{\fill} \\}
    
    \hypertarget{ligne-b}{%
\subsection{Ligne B}\label{ligne-b}}

    \begin{Verbatim}[commandchars=\\\{\}]
{\color{incolor}In [{\color{incolor}743}]:} \PY{n}{dfLigneB}\PY{o}{.}\PY{n}{head}\PY{p}{(}\PY{p}{)}
\end{Verbatim}


\begin{Verbatim}[commandchars=\\\{\}]
{\color{outcolor}Out[{\color{outcolor}743}]:}         Station  PM10  µg/m³  Matin  Soir
          0  Charpennes s           80      0     1
          1  Charpennes m          134      1     0
          2   Brotteaux s           74      0     1
          3   Brotteaux m          116      1     0
          4   Part-Dieu s          111      0     1
\end{Verbatim}
            
    \begin{Verbatim}[commandchars=\\\{\}]
{\color{incolor}In [{\color{incolor}744}]:} \PY{n}{dfLigneB}\PY{o}{.}\PY{n}{columns} \PY{o}{=} \PY{p}{[}\PY{l+s+s1}{\PYZsq{}}\PY{l+s+s1}{Station}\PY{l+s+s1}{\PYZsq{}}\PY{p}{,} \PY{l+s+s1}{\PYZsq{}}\PY{l+s+s1}{PM10 µg/m³}\PY{l+s+s1}{\PYZsq{}}\PY{p}{,} \PY{l+s+s1}{\PYZsq{}}\PY{l+s+s1}{Matin}\PY{l+s+s1}{\PYZsq{}}\PY{p}{,} \PY{l+s+s1}{\PYZsq{}}\PY{l+s+s1}{Soir}\PY{l+s+s1}{\PYZsq{}}\PY{p}{]}
          \PY{n}{dfLigneB}\PY{o}{.}\PY{n}{columns}
\end{Verbatim}


\begin{Verbatim}[commandchars=\\\{\}]
{\color{outcolor}Out[{\color{outcolor}744}]:} Index(['Station', 'PM10 µg/m³', 'Matin', 'Soir'], dtype='object')
\end{Verbatim}
            
    \begin{Verbatim}[commandchars=\\\{\}]
{\color{incolor}In [{\color{incolor}745}]:} \PY{n}{plt}\PY{o}{.}\PY{n}{rcdefaults}\PY{p}{(}\PY{p}{)}
          \PY{n}{fig}\PY{p}{,} \PY{n}{ax} \PY{o}{=} \PY{n}{plt}\PY{o}{.}\PY{n}{subplots}\PY{p}{(}\PY{p}{)}
          
          
          \PY{n}{people} \PY{o}{=} \PY{n}{dfLigneB}\PY{p}{[}\PY{l+s+s1}{\PYZsq{}}\PY{l+s+s1}{Station}\PY{l+s+s1}{\PYZsq{}}\PY{p}{]}
          \PY{n}{y\PYZus{}pos} \PY{o}{=} \PY{n}{np}\PY{o}{.}\PY{n}{arange}\PY{p}{(}\PY{n+nb}{len}\PY{p}{(}\PY{n}{people}\PY{p}{)}\PY{p}{)}
          
          
          \PY{n}{performance} \PY{o}{=} \PY{n}{dfLigneB}\PY{p}{[}\PY{l+s+s1}{\PYZsq{}}\PY{l+s+s1}{PM10 µg/m³}\PY{l+s+s1}{\PYZsq{}}\PY{p}{]}
          
          \PY{n}{error} \PY{o}{=} \PY{n}{np}\PY{o}{.}\PY{n}{random}\PY{o}{.}\PY{n}{rand}\PY{p}{(}\PY{n+nb}{len}\PY{p}{(}\PY{n}{people}\PY{p}{)}\PY{p}{)}
          
          \PY{n}{ax}\PY{o}{.}\PY{n}{barh}\PY{p}{(}\PY{n}{y\PYZus{}pos}\PY{p}{,} \PY{n}{performance}\PY{p}{,} \PY{n}{xerr}\PY{o}{=}\PY{n}{error}\PY{p}{,} \PY{n}{align}\PY{o}{=}\PY{l+s+s1}{\PYZsq{}}\PY{l+s+s1}{center}\PY{l+s+s1}{\PYZsq{}}\PY{p}{,}
                  \PY{n}{color}\PY{o}{=}\PY{l+s+s1}{\PYZsq{}}\PY{l+s+s1}{blue}\PY{l+s+s1}{\PYZsq{}}\PY{p}{,} \PY{n}{ecolor}\PY{o}{=}\PY{l+s+s1}{\PYZsq{}}\PY{l+s+s1}{black}\PY{l+s+s1}{\PYZsq{}}\PY{p}{)}
          \PY{n}{ax}\PY{o}{.}\PY{n}{set\PYZus{}yticks}\PY{p}{(}\PY{n}{y\PYZus{}pos}\PY{p}{)}
          \PY{n}{ax}\PY{o}{.}\PY{n}{set\PYZus{}yticklabels}\PY{p}{(}\PY{n}{people}\PY{p}{)}
          \PY{n}{ax}\PY{o}{.}\PY{n}{invert\PYZus{}yaxis}\PY{p}{(}\PY{p}{)}  \PY{c+c1}{\PYZsh{} labels read top\PYZhy{}to\PYZhy{}bottom}
          \PY{n}{ax}\PY{o}{.}\PY{n}{set\PYZus{}xlabel}\PY{p}{(}\PY{l+s+s1}{\PYZsq{}}\PY{l+s+s1}{PM10 µg/m³}\PY{l+s+s1}{\PYZsq{}}\PY{p}{)}
          \PY{n}{ax}\PY{o}{.}\PY{n}{set\PYZus{}title}\PY{p}{(}\PY{l+s+s1}{\PYZsq{}}\PY{l+s+s1}{Ligne B (m = matin, s = soir)}\PY{l+s+s1}{\PYZsq{}}\PY{p}{)}
          
          \PY{n}{plt}\PY{o}{.}\PY{n}{show}\PY{p}{(}\PY{p}{)}
\end{Verbatim}


    \begin{center}
    \adjustimage{max size={0.9\linewidth}{0.9\paperheight}}{output_45_0.png}
    \end{center}
    { \hspace*{\fill} \\}
    
    \hypertarget{ligne-c}{%
\subsubsection{Ligne C}\label{ligne-c}}

    \begin{Verbatim}[commandchars=\\\{\}]
{\color{incolor}In [{\color{incolor}746}]:} \PY{n}{dfLigneC}\PY{o}{.}\PY{n}{head}\PY{p}{(}\PY{p}{)}
\end{Verbatim}


\begin{Verbatim}[commandchars=\\\{\}]
{\color{outcolor}Out[{\color{outcolor}746}]:}             Station  PM10   µg/ m³  Matin  Soir
          0           Hénon s             26      0     1
          1           Hénon m             29      1     0
          2    Croix-Rousse s             40      0     1
          3    Croix-Rousse m             43      1     0
          4  Hôtel de ville s             93      0     1
\end{Verbatim}
            
    \begin{Verbatim}[commandchars=\\\{\}]
{\color{incolor}In [{\color{incolor}747}]:} \PY{n}{dfLigneC}\PY{o}{.}\PY{n}{columns} \PY{o}{=} \PY{p}{[}\PY{l+s+s1}{\PYZsq{}}\PY{l+s+s1}{Station}\PY{l+s+s1}{\PYZsq{}}\PY{p}{,} \PY{l+s+s1}{\PYZsq{}}\PY{l+s+s1}{PM10 µg/m³}\PY{l+s+s1}{\PYZsq{}}\PY{p}{,} \PY{l+s+s1}{\PYZsq{}}\PY{l+s+s1}{Matin}\PY{l+s+s1}{\PYZsq{}}\PY{p}{,} \PY{l+s+s1}{\PYZsq{}}\PY{l+s+s1}{Soir}\PY{l+s+s1}{\PYZsq{}}\PY{p}{]}
          \PY{n}{dfLigneC}\PY{o}{.}\PY{n}{columns}
\end{Verbatim}


\begin{Verbatim}[commandchars=\\\{\}]
{\color{outcolor}Out[{\color{outcolor}747}]:} Index(['Station', 'PM10 µg/m³', 'Matin', 'Soir'], dtype='object')
\end{Verbatim}
            
    \begin{Verbatim}[commandchars=\\\{\}]
{\color{incolor}In [{\color{incolor}748}]:} \PY{n}{plt}\PY{o}{.}\PY{n}{rcdefaults}\PY{p}{(}\PY{p}{)}
          \PY{n}{fig}\PY{p}{,} \PY{n}{ax} \PY{o}{=} \PY{n}{plt}\PY{o}{.}\PY{n}{subplots}\PY{p}{(}\PY{p}{)}
          
          
          \PY{n}{people} \PY{o}{=} \PY{n}{dfLigneC}\PY{p}{[}\PY{l+s+s1}{\PYZsq{}}\PY{l+s+s1}{Station}\PY{l+s+s1}{\PYZsq{}}\PY{p}{]}
          \PY{n}{y\PYZus{}pos} \PY{o}{=} \PY{n}{np}\PY{o}{.}\PY{n}{arange}\PY{p}{(}\PY{n+nb}{len}\PY{p}{(}\PY{n}{people}\PY{p}{)}\PY{p}{)}
          
          
          \PY{n}{performance} \PY{o}{=} \PY{n}{dfLigneC}\PY{p}{[}\PY{l+s+s1}{\PYZsq{}}\PY{l+s+s1}{PM10 µg/m³}\PY{l+s+s1}{\PYZsq{}}\PY{p}{]}
          
          \PY{n}{error} \PY{o}{=} \PY{n}{np}\PY{o}{.}\PY{n}{random}\PY{o}{.}\PY{n}{rand}\PY{p}{(}\PY{n+nb}{len}\PY{p}{(}\PY{n}{people}\PY{p}{)}\PY{p}{)}
          
          \PY{n}{ax}\PY{o}{.}\PY{n}{barh}\PY{p}{(}\PY{n}{y\PYZus{}pos}\PY{p}{,} \PY{n}{performance}\PY{p}{,} \PY{n}{xerr}\PY{o}{=}\PY{n}{error}\PY{p}{,} \PY{n}{align}\PY{o}{=}\PY{l+s+s1}{\PYZsq{}}\PY{l+s+s1}{center}\PY{l+s+s1}{\PYZsq{}}\PY{p}{,}
                  \PY{n}{color}\PY{o}{=}\PY{l+s+s1}{\PYZsq{}}\PY{l+s+s1}{orange}\PY{l+s+s1}{\PYZsq{}}\PY{p}{,} \PY{n}{ecolor}\PY{o}{=}\PY{l+s+s1}{\PYZsq{}}\PY{l+s+s1}{black}\PY{l+s+s1}{\PYZsq{}}\PY{p}{)}
          \PY{n}{ax}\PY{o}{.}\PY{n}{set\PYZus{}yticks}\PY{p}{(}\PY{n}{y\PYZus{}pos}\PY{p}{)}
          \PY{n}{ax}\PY{o}{.}\PY{n}{set\PYZus{}yticklabels}\PY{p}{(}\PY{n}{people}\PY{p}{)}
          \PY{n}{ax}\PY{o}{.}\PY{n}{invert\PYZus{}yaxis}\PY{p}{(}\PY{p}{)}  \PY{c+c1}{\PYZsh{} labels read top\PYZhy{}to\PYZhy{}bottom}
          \PY{n}{ax}\PY{o}{.}\PY{n}{set\PYZus{}xlabel}\PY{p}{(}\PY{l+s+s1}{\PYZsq{}}\PY{l+s+s1}{PM10 µg/m³}\PY{l+s+s1}{\PYZsq{}}\PY{p}{)}
          \PY{n}{ax}\PY{o}{.}\PY{n}{set\PYZus{}title}\PY{p}{(}\PY{l+s+s1}{\PYZsq{}}\PY{l+s+s1}{Ligne C (m = matin, s = soir)}\PY{l+s+s1}{\PYZsq{}}\PY{p}{)}
          
          \PY{n}{plt}\PY{o}{.}\PY{n}{show}\PY{p}{(}\PY{p}{)}
\end{Verbatim}


    \begin{center}
    \adjustimage{max size={0.9\linewidth}{0.9\paperheight}}{output_49_0.png}
    \end{center}
    { \hspace*{\fill} \\}
    
    \hypertarget{ligne-d}{%
\subsubsection{Ligne D}\label{ligne-d}}

    \begin{Verbatim}[commandchars=\\\{\}]
{\color{incolor}In [{\color{incolor}749}]:} \PY{n}{dfLigneD}\PY{o}{.}\PY{n}{head}\PY{p}{(}\PY{p}{)}
\end{Verbatim}


\begin{Verbatim}[commandchars=\\\{\}]
{\color{outcolor}Out[{\color{outcolor}749}]:}            Station  PM10  µg/m³  Matin  Soir
          0  Gare de Vaise s           74      0     1
          1  Gare de Vaise m          104      1     0
          2          Valmy s          130      0     1
          3          Valmy m          129      1     0
          4  Gorge de Loup s          105      0     1
\end{Verbatim}
            
    \begin{Verbatim}[commandchars=\\\{\}]
{\color{incolor}In [{\color{incolor}750}]:} \PY{n}{dfLigneD}\PY{o}{.}\PY{n}{columns} \PY{o}{=} \PY{p}{[}\PY{l+s+s1}{\PYZsq{}}\PY{l+s+s1}{Station}\PY{l+s+s1}{\PYZsq{}}\PY{p}{,} \PY{l+s+s1}{\PYZsq{}}\PY{l+s+s1}{PM10 µg/m³}\PY{l+s+s1}{\PYZsq{}}\PY{p}{,} \PY{l+s+s1}{\PYZsq{}}\PY{l+s+s1}{Matin}\PY{l+s+s1}{\PYZsq{}}\PY{p}{,} \PY{l+s+s1}{\PYZsq{}}\PY{l+s+s1}{Soir}\PY{l+s+s1}{\PYZsq{}}\PY{p}{]}
          \PY{n}{dfLigneD}\PY{o}{.}\PY{n}{columns}
\end{Verbatim}


\begin{Verbatim}[commandchars=\\\{\}]
{\color{outcolor}Out[{\color{outcolor}750}]:} Index(['Station', 'PM10 µg/m³', 'Matin', 'Soir'], dtype='object')
\end{Verbatim}
            
    \begin{Verbatim}[commandchars=\\\{\}]
{\color{incolor}In [{\color{incolor}751}]:} \PY{n}{plt}\PY{o}{.}\PY{n}{rcdefaults}\PY{p}{(}\PY{p}{)}
          \PY{n}{fig}\PY{p}{,} \PY{n}{ax} \PY{o}{=} \PY{n}{plt}\PY{o}{.}\PY{n}{subplots}\PY{p}{(}\PY{p}{)}
          
          
          \PY{n}{people} \PY{o}{=} \PY{n}{dfLigneD}\PY{p}{[}\PY{l+s+s1}{\PYZsq{}}\PY{l+s+s1}{Station}\PY{l+s+s1}{\PYZsq{}}\PY{p}{]}
          \PY{n}{y\PYZus{}pos} \PY{o}{=} \PY{n}{np}\PY{o}{.}\PY{n}{arange}\PY{p}{(}\PY{n+nb}{len}\PY{p}{(}\PY{n}{people}\PY{p}{)}\PY{p}{)}
          
          
          \PY{n}{performance} \PY{o}{=} \PY{n}{dfLigneD}\PY{p}{[}\PY{l+s+s1}{\PYZsq{}}\PY{l+s+s1}{PM10 µg/m³}\PY{l+s+s1}{\PYZsq{}}\PY{p}{]}
          
          \PY{n}{error} \PY{o}{=} \PY{n}{np}\PY{o}{.}\PY{n}{random}\PY{o}{.}\PY{n}{rand}\PY{p}{(}\PY{n+nb}{len}\PY{p}{(}\PY{n}{people}\PY{p}{)}\PY{p}{)}
          
          \PY{n}{ax}\PY{o}{.}\PY{n}{barh}\PY{p}{(}\PY{n}{y\PYZus{}pos}\PY{p}{,} \PY{n}{performance}\PY{p}{,} \PY{n}{xerr}\PY{o}{=}\PY{n}{error}\PY{p}{,} \PY{n}{align}\PY{o}{=}\PY{l+s+s1}{\PYZsq{}}\PY{l+s+s1}{center}\PY{l+s+s1}{\PYZsq{}}\PY{p}{,}
                  \PY{n}{color}\PY{o}{=}\PY{l+s+s1}{\PYZsq{}}\PY{l+s+s1}{darkgreen}\PY{l+s+s1}{\PYZsq{}}\PY{p}{,} \PY{n}{ecolor}\PY{o}{=}\PY{l+s+s1}{\PYZsq{}}\PY{l+s+s1}{black}\PY{l+s+s1}{\PYZsq{}}\PY{p}{)}
          \PY{n}{ax}\PY{o}{.}\PY{n}{set\PYZus{}yticks}\PY{p}{(}\PY{n}{y\PYZus{}pos}\PY{p}{)}
          \PY{n}{ax}\PY{o}{.}\PY{n}{set\PYZus{}yticklabels}\PY{p}{(}\PY{n}{people}\PY{p}{)}
          \PY{n}{ax}\PY{o}{.}\PY{n}{invert\PYZus{}yaxis}\PY{p}{(}\PY{p}{)}  \PY{c+c1}{\PYZsh{} labels read top\PYZhy{}to\PYZhy{}bottom}
          \PY{n}{ax}\PY{o}{.}\PY{n}{set\PYZus{}xlabel}\PY{p}{(}\PY{l+s+s1}{\PYZsq{}}\PY{l+s+s1}{PM10 µg/m³}\PY{l+s+s1}{\PYZsq{}}\PY{p}{)}
          \PY{n}{ax}\PY{o}{.}\PY{n}{set\PYZus{}title}\PY{p}{(}\PY{l+s+s1}{\PYZsq{}}\PY{l+s+s1}{Ligne D (m = matin, s = soir)}\PY{l+s+s1}{\PYZsq{}}\PY{p}{)}
          
          \PY{n}{plt}\PY{o}{.}\PY{n}{show}\PY{p}{(}\PY{p}{)}
\end{Verbatim}


    \begin{center}
    \adjustimage{max size={0.9\linewidth}{0.9\paperheight}}{output_53_0.png}
    \end{center}
    { \hspace*{\fill} \\}
    
    \hypertarget{finiculaire-saint-jus}{%
\subsubsection{Finiculaire Saint Jus}\label{finiculaire-saint-jus}}

    \begin{Verbatim}[commandchars=\\\{\}]
{\color{incolor}In [{\color{incolor}752}]:} \PY{n}{dfFuniculaireSaintJus}
\end{Verbatim}


\begin{Verbatim}[commandchars=\\\{\}]
{\color{outcolor}Out[{\color{outcolor}752}]:}               Station  PM10 µg/m³  Matin  Soir
          0  Vieux Lyon St Ju s          26      0     1
          1  Vieux Lyon St Ju m         111      1     0
          2        Saint Just s          26      0     1
          3        Saint Just m          40      1     0
\end{Verbatim}
            
    \begin{Verbatim}[commandchars=\\\{\}]
{\color{incolor}In [{\color{incolor}753}]:} \PY{n}{dfFuniculaireSaintJus}\PY{o}{.}\PY{n}{columns} \PY{o}{=} \PY{p}{[}\PY{l+s+s1}{\PYZsq{}}\PY{l+s+s1}{Station}\PY{l+s+s1}{\PYZsq{}}\PY{p}{,} \PY{l+s+s1}{\PYZsq{}}\PY{l+s+s1}{PM10 µg/m³}\PY{l+s+s1}{\PYZsq{}}\PY{p}{,} \PY{l+s+s1}{\PYZsq{}}\PY{l+s+s1}{Matin}\PY{l+s+s1}{\PYZsq{}}\PY{p}{,} \PY{l+s+s1}{\PYZsq{}}\PY{l+s+s1}{Soir}\PY{l+s+s1}{\PYZsq{}}\PY{p}{]}
          \PY{n}{dfFuniculaireSaintJus}\PY{o}{.}\PY{n}{columns}
\end{Verbatim}


\begin{Verbatim}[commandchars=\\\{\}]
{\color{outcolor}Out[{\color{outcolor}753}]:} Index(['Station', 'PM10 µg/m³', 'Matin', 'Soir'], dtype='object')
\end{Verbatim}
            
    \begin{Verbatim}[commandchars=\\\{\}]
{\color{incolor}In [{\color{incolor}754}]:} \PY{c+c1}{\PYZsh{} group by the Country}
          \PY{n}{Station} \PY{o}{=} \PY{n}{dfFuniculaireSaintJus}\PY{o}{.}\PY{n}{groupby}\PY{p}{(}\PY{l+s+s1}{\PYZsq{}}\PY{l+s+s1}{Station}\PY{l+s+s1}{\PYZsq{}}\PY{p}{)}\PY{o}{.}\PY{n}{sum}\PY{p}{(}\PY{p}{)}
          
          \PY{c+c1}{\PYZsh{} we sort the value and get the first 10 high parking shape}
          \PY{n}{Station} \PY{o}{=} \PY{n}{Station}\PY{o}{.}\PY{n}{sort\PYZus{}values}\PY{p}{(}\PY{n}{by} \PY{o}{=} \PY{l+s+s1}{\PYZsq{}}\PY{l+s+s1}{PM10 µg/m³}\PY{l+s+s1}{\PYZsq{}}\PY{p}{,}\PY{n}{ascending} \PY{o}{=} \PY{k+kc}{False}\PY{p}{)}
          
          \PY{c+c1}{\PYZsh{} create the plot}
          \PY{n}{Station}\PY{p}{[}\PY{l+s+s1}{\PYZsq{}}\PY{l+s+s1}{PM10 µg/m³}\PY{l+s+s1}{\PYZsq{}}\PY{p}{]}\PY{o}{.}\PY{n}{plot}\PY{p}{(}\PY{n}{kind}\PY{o}{=}\PY{l+s+s1}{\PYZsq{}}\PY{l+s+s1}{bar}\PY{l+s+s1}{\PYZsq{}}\PY{p}{,} \PY{n}{color}\PY{o}{=}\PY{l+s+s1}{\PYZsq{}}\PY{l+s+s1}{lightgreen}\PY{l+s+s1}{\PYZsq{}}\PY{p}{)}
          
          \PY{c+c1}{\PYZsh{} Set the title and labels}
          \PY{n}{plt}\PY{o}{.}\PY{n}{xlabel}\PY{p}{(}\PY{l+s+s1}{\PYZsq{}}\PY{l+s+s1}{Station}\PY{l+s+s1}{\PYZsq{}}\PY{p}{,} \PY{n}{fontsize}\PY{o}{=}\PY{l+m+mi}{18}\PY{p}{)}
          \PY{n}{plt}\PY{o}{.}\PY{n}{ylabel}\PY{p}{(}\PY{l+s+s1}{\PYZsq{}}\PY{l+s+s1}{PM10 µg/m³}\PY{l+s+s1}{\PYZsq{}}\PY{p}{,}\PY{n}{fontsize}\PY{o}{=}\PY{l+m+mi}{18}\PY{p}{)}
          \PY{n}{plt}\PY{o}{.}\PY{n}{title}\PY{p}{(}\PY{l+s+s1}{\PYZsq{}}\PY{l+s+s1}{Funiculaire Saint Jus (m = matin, s = soir)}\PY{l+s+s1}{\PYZsq{}}\PY{p}{,} \PY{n}{fontsize}\PY{o}{=}\PY{l+m+mi}{18}\PY{p}{)}
          
          \PY{c+c1}{\PYZsh{} show the plot}
          \PY{n}{plt}\PY{o}{.}\PY{n}{show}\PY{p}{(}\PY{p}{)}
\end{Verbatim}


    \begin{center}
    \adjustimage{max size={0.9\linewidth}{0.9\paperheight}}{output_57_0.png}
    \end{center}
    { \hspace*{\fill} \\}
    
    \hypertarget{finiculaire-fourviere}{%
\subsubsection{Finiculaire Fourviere}\label{finiculaire-fourviere}}

    \begin{Verbatim}[commandchars=\\\{\}]
{\color{incolor}In [{\color{incolor}755}]:} \PY{n}{dfFuniculaireFourviere}
\end{Verbatim}


\begin{Verbatim}[commandchars=\\\{\}]
{\color{outcolor}Out[{\color{outcolor}755}]:}                   Station  PM10 µg/m³  Matin  Soir
          0             Fourvière s          22      0     1
          1             Fourvière m          22      1     0
          2  Vieux Lyon Fourvière s          30      0     1
          3  Vieux Lyon Fourvière m          81      1     0
\end{Verbatim}
            
    \begin{Verbatim}[commandchars=\\\{\}]
{\color{incolor}In [{\color{incolor}756}]:} \PY{n}{dfFuniculaireFourviere}\PY{o}{.}\PY{n}{columns} \PY{o}{=} \PY{p}{[}\PY{l+s+s1}{\PYZsq{}}\PY{l+s+s1}{Station}\PY{l+s+s1}{\PYZsq{}}\PY{p}{,} \PY{l+s+s1}{\PYZsq{}}\PY{l+s+s1}{PM10 µg/m³}\PY{l+s+s1}{\PYZsq{}}\PY{p}{,} \PY{l+s+s1}{\PYZsq{}}\PY{l+s+s1}{Matin}\PY{l+s+s1}{\PYZsq{}}\PY{p}{,} \PY{l+s+s1}{\PYZsq{}}\PY{l+s+s1}{Soir}\PY{l+s+s1}{\PYZsq{}}\PY{p}{]}
          \PY{n}{dfFuniculaireFourviere}\PY{o}{.}\PY{n}{columns}
\end{Verbatim}


\begin{Verbatim}[commandchars=\\\{\}]
{\color{outcolor}Out[{\color{outcolor}756}]:} Index(['Station', 'PM10 µg/m³', 'Matin', 'Soir'], dtype='object')
\end{Verbatim}
            
    \begin{Verbatim}[commandchars=\\\{\}]
{\color{incolor}In [{\color{incolor}757}]:} \PY{c+c1}{\PYZsh{} group by the Country}
          \PY{n}{Station} \PY{o}{=} \PY{n}{dfFuniculaireFourviere}\PY{o}{.}\PY{n}{groupby}\PY{p}{(}\PY{l+s+s1}{\PYZsq{}}\PY{l+s+s1}{Station}\PY{l+s+s1}{\PYZsq{}}\PY{p}{)}\PY{o}{.}\PY{n}{sum}\PY{p}{(}\PY{p}{)}
          
          \PY{c+c1}{\PYZsh{} we sort the value and get the first 10 high parking shape}
          \PY{n}{Station} \PY{o}{=} \PY{n}{Station}\PY{o}{.}\PY{n}{sort\PYZus{}values}\PY{p}{(}\PY{n}{by} \PY{o}{=} \PY{l+s+s1}{\PYZsq{}}\PY{l+s+s1}{PM10 µg/m³}\PY{l+s+s1}{\PYZsq{}}\PY{p}{,}\PY{n}{ascending} \PY{o}{=} \PY{k+kc}{False}\PY{p}{)}
          
          \PY{c+c1}{\PYZsh{} create the plot}
          \PY{n}{Station}\PY{p}{[}\PY{l+s+s1}{\PYZsq{}}\PY{l+s+s1}{PM10 µg/m³}\PY{l+s+s1}{\PYZsq{}}\PY{p}{]}\PY{o}{.}\PY{n}{plot}\PY{p}{(}\PY{n}{kind}\PY{o}{=}\PY{l+s+s1}{\PYZsq{}}\PY{l+s+s1}{bar}\PY{l+s+s1}{\PYZsq{}}\PY{p}{,} \PY{n}{color}\PY{o}{=}\PY{l+s+s1}{\PYZsq{}}\PY{l+s+s1}{lightgreen}\PY{l+s+s1}{\PYZsq{}}\PY{p}{)}
          
          \PY{c+c1}{\PYZsh{} Set the title and labels}
          \PY{n}{plt}\PY{o}{.}\PY{n}{xlabel}\PY{p}{(}\PY{l+s+s1}{\PYZsq{}}\PY{l+s+s1}{Station}\PY{l+s+s1}{\PYZsq{}}\PY{p}{,} \PY{n}{fontsize}\PY{o}{=}\PY{l+m+mi}{18}\PY{p}{)}
          \PY{n}{plt}\PY{o}{.}\PY{n}{ylabel}\PY{p}{(}\PY{l+s+s1}{\PYZsq{}}\PY{l+s+s1}{PM10 µg/m³}\PY{l+s+s1}{\PYZsq{}}\PY{p}{,}\PY{n}{fontsize}\PY{o}{=}\PY{l+m+mi}{18}\PY{p}{)}
          \PY{n}{plt}\PY{o}{.}\PY{n}{title}\PY{p}{(}\PY{l+s+s1}{\PYZsq{}}\PY{l+s+s1}{Funiculaire de Fourviere (m = matin, s = soir)}\PY{l+s+s1}{\PYZsq{}}\PY{p}{,} \PY{n}{fontsize}\PY{o}{=}\PY{l+m+mi}{18}\PY{p}{)}
          
          \PY{c+c1}{\PYZsh{} show the plot}
          \PY{n}{plt}\PY{o}{.}\PY{n}{show}\PY{p}{(}\PY{p}{)}
\end{Verbatim}


    \begin{center}
    \adjustimage{max size={0.9\linewidth}{0.9\paperheight}}{output_61_0.png}
    \end{center}
    { \hspace*{\fill} \\}
    
    Ces instantanés ne permettent pas de dégager des variations précises.
Cependant ils mettent en lumière que lors des mesures, la ligne A est
celle où les particules fines PM10 sont les plus présentes avec une
moyenne de 107 µg/ m³. À l'inverse, la ligne C et le funiculaire
Fourvière sont celles où elles sont en concentration plus faible, avec
des moyennes respectivement à 46 µg/ m³ et 38,75 µg/ m³. Une différence
qui s'explique grâce à des portions à ciel ouvert. Enfin, la
concentration la plus forte a été relevée le 14 mars 2017 dans la
station de La Soie sur la ligne A (166 µg/ m³).

    These snapshots do not allow precise variations to be identified.
However, they highlight that during measurements, line A is the one
where fine particles PM10 are most present with an average of 107 µg/m³.
Conversely, line C and the Fourvière funicular are the ones where they
are in lower concentration, with averages of 46 µg/m³ and 38.75 µg/m³
respectively. This difference can be explained by the open-air portions.
Finally, the highest concentration was found on 14 March 2017 in La Soie
station on line A (166 µg/m³).

    \hypertarget{station-metro}{%
\subsubsection{Station metro}\label{station-metro}}

\hypertarget{subways-stop}{%
\subsubsection{Subway's stop}\label{subways-stop}}

\hypertarget{le-rond-point-le-plus-utilisuxe9s-dans-le-ruxe9seaux-tcl}{%
\subsubsection{Le rond point le plus utilisés dans le réseaux
TCL}\label{le-rond-point-le-plus-utilisuxe9s-dans-le-ruxe9seaux-tcl}}

\hypertarget{the-most-used-roundabout-in-the-tcl-network}{%
\subsubsection{The most used roundabout in the TCL
network}\label{the-most-used-roundabout-in-the-tcl-network}}

    \hypertarget{station-de-saxe-pour-le-pm10}{%
\subsubsection{Station de Saxe Pour le
PM10}\label{station-de-saxe-pour-le-pm10}}

\hypertarget{saxes-stop-for-pm10}{%
\subsubsection{Saxe's Stop for PM10}\label{saxes-stop-for-pm10}}

    Les mesures de juin 2017 à la station Saxe-Gambetta pour les PM10
réalisées du 13 juin 2017 au 20 juin 2017 (En comparaison avec les
données fournies par Atmo-Auvergne-Rhône-Alpes pour la même période pour
la station de mesure de l'air extérieur à Lyon Gerland).

    June 2017 measurements at the Saxe-Gambetta station for PM10 from 13
June 2017 to 20 June 2017 (In comparison with the data provided by
Atmo-Auvergne-Rhône-Alpes for the same period for the outdoor air
measurement station in Lyon Gerland).

    \begin{Verbatim}[commandchars=\\\{\}]
{\color{incolor}In [{\color{incolor}758}]:} \PY{n}{dfStationSaxe\PYZus{}PM10}\PY{o}{.}\PY{n}{reset\PYZus{}index}\PY{p}{(}\PY{n}{inplace} \PY{o}{=} \PY{k+kc}{True}\PY{p}{)}
          \PY{n}{dfStationSaxe\PYZus{}PM10}\PY{o}{.}\PY{n}{loc}\PY{p}{[}\PY{l+m+mi}{0}\PY{p}{,} \PY{l+s+s1}{\PYZsq{}}\PY{l+s+s1}{Lyon Gerland (source Atmo\PYZhy{}Auvergne\PYZhy{}Rhône\PYZhy{}Alpes)}\PY{l+s+s1}{\PYZsq{}}\PY{p}{]} \PY{o}{=} \PY{l+s+s2}{\PYZdq{}}\PY{l+s+s2}{4}\PY{l+s+s2}{\PYZdq{}}
          \PY{n}{dfStationSaxe\PYZus{}PM10}\PY{o}{.}\PY{n}{loc}\PY{p}{[}\PY{l+m+mi}{23}\PY{p}{,} \PY{l+s+s1}{\PYZsq{}}\PY{l+s+s1}{Lyon Gerland (source Atmo\PYZhy{}Auvergne\PYZhy{}Rhône\PYZhy{}Alpes)}\PY{l+s+s1}{\PYZsq{}}\PY{p}{]} \PY{o}{=} \PY{l+s+s2}{\PYZdq{}}\PY{l+s+s2}{3}\PY{l+s+s2}{\PYZdq{}}
          
          \PY{n}{dfStationSaxe\PYZus{}PM10}\PY{p}{[}\PY{l+s+s1}{\PYZsq{}}\PY{l+s+s1}{Lyon Gerland (source Atmo\PYZhy{}Auvergne\PYZhy{}Rhône\PYZhy{}Alpes)}\PY{l+s+s1}{\PYZsq{}}\PY{p}{]}\PY{o}{=} \PY{n}{dfStationSaxe\PYZus{}PM10}\PY{p}{[}\PY{l+s+s1}{\PYZsq{}}\PY{l+s+s1}{Lyon Gerland (source Atmo\PYZhy{}Auvergne\PYZhy{}Rhône\PYZhy{}Alpes)}\PY{l+s+s1}{\PYZsq{}}\PY{p}{]}\PY{o}{.}\PY{n}{fillna}\PY{p}{(}\PY{n}{dfStationSaxe\PYZus{}PM10}\PY{p}{[}\PY{l+s+s1}{\PYZsq{}}\PY{l+s+s1}{Lyon Gerland (source Atmo\PYZhy{}Auvergne\PYZhy{}Rhône\PYZhy{}Alpes)}\PY{l+s+s1}{\PYZsq{}}\PY{p}{]}\PY{o}{.}\PY{n}{median}\PY{p}{(}\PY{p}{)}\PY{p}{)}
          \PY{n}{dfStationSaxe\PYZus{}PM10}\PY{p}{[}\PY{l+s+s1}{\PYZsq{}}\PY{l+s+s1}{Lyon Gerland (source Atmo\PYZhy{}Auvergne\PYZhy{}Rhône\PYZhy{}Alpes)}\PY{l+s+s1}{\PYZsq{}}\PY{p}{]} \PY{o}{=} \PY{n}{dfStationSaxe\PYZus{}PM10}\PY{p}{[}\PY{l+s+s1}{\PYZsq{}}\PY{l+s+s1}{Lyon Gerland (source Atmo\PYZhy{}Auvergne\PYZhy{}Rhône\PYZhy{}Alpes)}\PY{l+s+s1}{\PYZsq{}}\PY{p}{]}\PY{o}{.}\PY{n}{astype}\PY{p}{(}\PY{n+nb}{float}\PY{p}{)}
          
          \PY{n}{dfStationSaxe\PYZus{}PM10}\PY{o}{.}\PY{n}{set\PYZus{}index}\PY{p}{(}\PY{l+s+s2}{\PYZdq{}}\PY{l+s+s2}{Date}\PY{l+s+s2}{\PYZdq{}}\PY{p}{,} \PY{n}{inplace} \PY{o}{=} \PY{k+kc}{True}\PY{p}{)}
          \PY{n}{dfStationSaxe\PYZus{}PM10}\PY{o}{.}\PY{n}{head}\PY{p}{(}\PY{p}{)}
\end{Verbatim}


\begin{Verbatim}[commandchars=\\\{\}]
{\color{outcolor}Out[{\color{outcolor}758}]:}                       PM10  Lyon Gerland (source Atmo-Auvergne-Rhône-Alpes)
          Date                                                                       
          2017-06-13 06:00:00    8.3                                              4.0
          2017-06-13 07:00:00   51.9                                              6.0
          2017-06-13 08:00:00  116.4                                             11.0
          2017-06-13 09:00:00   67.0                                             18.0
          2017-06-13 10:00:00    0.0                                             24.0
\end{Verbatim}
            
    \begin{Verbatim}[commandchars=\\\{\}]
{\color{incolor}In [{\color{incolor}759}]:} \PY{n}{dfStationSaxe\PYZus{}PM10}\PY{o}{.}\PY{n}{plot}\PY{p}{(}\PY{n}{kind}\PY{o}{=}\PY{l+s+s1}{\PYZsq{}}\PY{l+s+s1}{bar}\PY{l+s+s1}{\PYZsq{}}\PY{p}{,} \PY{n}{stacked}\PY{o}{=}\PY{k+kc}{True}\PY{p}{,} \PY{n}{figsize}\PY{o}{=}\PY{p}{(}\PY{l+m+mi}{50}\PY{p}{,}\PY{l+m+mi}{30}\PY{p}{)}\PY{p}{)}
\end{Verbatim}


\begin{Verbatim}[commandchars=\\\{\}]
{\color{outcolor}Out[{\color{outcolor}759}]:} <matplotlib.axes.\_subplots.AxesSubplot at 0x1a63358f28>
\end{Verbatim}
            
    \begin{center}
    \adjustimage{max size={0.9\linewidth}{0.9\paperheight}}{output_69_1.png}
    \end{center}
    { \hspace*{\fill} \\}
    
    Les résultats des mesures pour les PM10 à la station Saxe-Gambetta
dépassent à cinq reprises la valeur de référence Csout de 226 µg/m³.
Selon le Sytral, la valeur très forte de 378,9 µg/m³ relevée le 19 juin
est ``due à une découpe de panneaux en bois réalisé à proximité de
l'analyseur''. Dans tous les cas, la concentration en PM10 reste bien
plus élevée dans le réseau métro que celle relevée sur le site extérieur
de Lyon Gerland par Atmo-Auvergne-Rhône-Alpes. Quand à l'extérieur la
moyenne est de 20 µg/m³, elle est de 83 µg/m³ dans la station
Saxe-Gambetta. En comparant les deux jeux de données, la concentration
des particules fines est donc quatre fois plus élevée dans le réseau
métro.

    The measurement results for PM10 at the Saxony-Gambetta station exceed
the Csout reference value of 226 µg/m³ five times. According to Sytral,
the very high value of 378.9 µg/m³ recorded on 19 June is ``due to a
cutting of wooden panels made near the analyser''. In any case, the PM10
concentration remains much higher in the metro network than that
recorded on the Lyon Gerland outdoor site by Atmo-Auvergne-Rhône-Alpes.
When outdoors the average is 20 µg/m³, it is 83 µg/m³ in the
Saxony-Gambetta station. By comparing the two datasets, the
concentration of fine particles is therefore four times higher in the
metro network.

    \hypertarget{station-de-saxe-pour-le-pm2.5}{%
\subsubsection{Station de Saxe Pour le
PM2.5}\label{station-de-saxe-pour-le-pm2.5}}

\hypertarget{saxes-stop-for-pm2.5}{%
\subsubsection{Saxe's Stop for PM2.5}\label{saxes-stop-for-pm2.5}}

    Les mesures de juin 2017 à la station Saxe-Gambetta pour les PM2.5
réalisées du 20 juin 2017 au 27 juin 2017 (En comparaison avec les
données fournies, lorsqu'elles sont disponibles, par
Atmo-Auvergne-Rhône-Alpes pour la station de mesure de l'air extérieur à
Lyon Centre).

    June 2017 measurements at the Saxe-Gambetta station for PM2.5 from 20
June 2017 to 27 June 2017 (In comparison with the data provided, when
available, by Atmo-Auvergne-Rhône-Alpes for the outdoor air measurement
station in Lyon Centre).

    \begin{Verbatim}[commandchars=\\\{\}]
{\color{incolor}In [{\color{incolor}760}]:} \PY{n}{dfStationSaxePM2\PYZus{}5}\PY{p}{[}\PY{l+s+s1}{\PYZsq{}}\PY{l+s+s1}{Lyon Centre}\PY{l+s+s1}{\PYZsq{}}\PY{p}{]}\PY{o}{=} \PY{n}{dfStationSaxePM2\PYZus{}5}\PY{p}{[}\PY{l+s+s1}{\PYZsq{}}\PY{l+s+s1}{Lyon Centre}\PY{l+s+s1}{\PYZsq{}}\PY{p}{]}\PY{o}{.}\PY{n}{fillna}\PY{p}{(}\PY{n}{dfStationSaxePM2\PYZus{}5}\PY{p}{[}\PY{l+s+s1}{\PYZsq{}}\PY{l+s+s1}{Lyon Centre}\PY{l+s+s1}{\PYZsq{}}\PY{p}{]}\PY{o}{.}\PY{n}{median}\PY{p}{(}\PY{p}{)}\PY{p}{)}
          \PY{n}{dfStationSaxePM2\PYZus{}5}\PY{p}{[}\PY{l+s+s1}{\PYZsq{}}\PY{l+s+s1}{Lyon Centre}\PY{l+s+s1}{\PYZsq{}}\PY{p}{]} \PY{o}{=} \PY{n}{dfStationSaxePM2\PYZus{}5}\PY{p}{[}\PY{l+s+s1}{\PYZsq{}}\PY{l+s+s1}{Lyon Centre}\PY{l+s+s1}{\PYZsq{}}\PY{p}{]}\PY{o}{.}\PY{n}{astype}\PY{p}{(}\PY{n+nb}{float}\PY{p}{)}
          \PY{n}{dfStationSaxePM2\PYZus{}5}\PY{o}{.}\PY{n}{head}\PY{p}{(}\PY{p}{)}
\end{Verbatim}


\begin{Verbatim}[commandchars=\\\{\}]
{\color{outcolor}Out[{\color{outcolor}760}]:}                      Moyenne horaire PM2.5 en µg/m3  Lyon Centre
          Date                                                            
          2017-06-20 03:00:00                            15.8         13.0
          2017-06-20 04:00:00                            29.1         13.0
          2017-06-20 05:00:00                            17.7         13.0
          2017-06-20 06:00:00                            18.7         13.0
          2017-06-20 07:00:00                            28.0         13.0
\end{Verbatim}
            
    \begin{Verbatim}[commandchars=\\\{\}]
{\color{incolor}In [{\color{incolor}761}]:} \PY{n}{dfStationSaxePM2\PYZus{}5}\PY{o}{.}\PY{n}{plot}\PY{p}{(}\PY{n}{kind}\PY{o}{=}\PY{l+s+s1}{\PYZsq{}}\PY{l+s+s1}{bar}\PY{l+s+s1}{\PYZsq{}}\PY{p}{,} \PY{n}{stacked}\PY{o}{=}\PY{k+kc}{True}\PY{p}{,} \PY{n}{figsize}\PY{o}{=}\PY{p}{(}\PY{l+m+mi}{50}\PY{p}{,}\PY{l+m+mi}{30}\PY{p}{)}\PY{p}{)}
\end{Verbatim}


\begin{Verbatim}[commandchars=\\\{\}]
{\color{outcolor}Out[{\color{outcolor}761}]:} <matplotlib.axes.\_subplots.AxesSubplot at 0x1a49426978>
\end{Verbatim}
            
    \begin{center}
    \adjustimage{max size={0.9\linewidth}{0.9\paperheight}}{output_76_1.png}
    \end{center}
    { \hspace*{\fill} \\}
    
    On peut souligner une récurrence dans les mesures avec des fortes
concentrations de PM2.5 en début de soirée, entre 18h et 21h. Les
valeurs dépassent alors régulièrement les 100 µg/m³. Seule exception, le
dimanche 25 juin, où le seuil est légèrement dépassé une seule fois
(100.1 µg/m³). Selon une source interne, en heure de pointe, la
fréquence des rames joue dans l'augmentation de ces valeurs.

Enfin, en absence de données pour le site de Gerland pour cette période
sur la page d'Atmo-Auvergne-Rhône-Alpes, nous utilisons ici celles du
site de Lyon Centre pour réaliser une comparaison avec l'air extérieur.
A noter, des mesures sont manquantes sur certains créneaux horaires.
Ainsi, en comparant uniquement les créneaux où les données existent dans
les deux cas, et en gardant en tête les limites inhérentes à cette
absence de données sur l'intégralité de la période, la moyenne est de 53
µg/m³ dans la station Saxe-Gambetta contre 13 µg/m³ relevés sur le site
Lyon Centre. Dès lors, la concentration de PM2.5 est quatre fois plus
élevée dans le métro qu'à l'extérieur sur cette durée.

    A recurrence in measurements with high concentrations of PM2.5 can be
noted in the early evening, between 6 pm and 9 pm. The values then
regularly exceed 100 µg/m³. The only exception is Sunday, June 25, when
the threshold is slightly exceeded only once (100.1 µg/m³). According to
an internal source, at peak times, the frequency of trains plays a role
in increasing these values.

Finally, in the absence of data for the Gerland site for this period on
the Atmo-Auvergne-Rhône-Alpes page, we use here those of the Lyon Centre
site to make a comparison with the outside air. It should be noted that
measurements are missing in some time slots. Thus, by comparing only the
slots where data exist in both cases, and keeping in mind the inherent
limitations of this lack of data over the entire period, the average is
53 µg/m³ in the Saxe-Gambetta station compared to 13 µg/m³ recorded at
the Lyon Centre site. As a result, the concentration of PM2.5 is four
times higher in the metro than outside over this period.

    \hypertarget{analyse-des-donnuxe9es}{%
\section{Analyse des données}\label{analyse-des-donnuxe9es}}

\hypertarget{data-analyses}{%
\section{Data Analyses}\label{data-analyses}}

    \begin{Verbatim}[commandchars=\\\{\}]
{\color{incolor}In [{\color{incolor}762}]:} \PY{c+c1}{\PYZsh{}Dans le cas du PM2.5}
          \PY{c+c1}{\PYZsh{}in case of PM2.5}
          \PY{k}{def} \PY{n+nf}{plot\PYZus{}df}\PY{p}{(}\PY{n}{col}\PY{p}{)}\PY{p}{:}
              \PY{n}{plt}\PY{o}{.}\PY{n}{figure}\PY{p}{(}\PY{n}{figsize}\PY{o}{=}\PY{p}{(}\PY{l+m+mi}{17}\PY{p}{,} \PY{l+m+mi}{8}\PY{p}{)}\PY{p}{)}
              \PY{n}{plt}\PY{o}{.}\PY{n}{plot}\PY{p}{(}\PY{n}{dfStationSaxePM2\PYZus{}5}\PY{p}{[}\PY{n}{col}\PY{p}{]}\PY{p}{)}
              \PY{n}{plt}\PY{o}{.}\PY{n}{xlabel}\PY{p}{(}\PY{l+s+s1}{\PYZsq{}}\PY{l+s+s1}{Time}\PY{l+s+s1}{\PYZsq{}}\PY{p}{)}
              \PY{n}{plt}\PY{o}{.}\PY{n}{ylabel}\PY{p}{(}\PY{n}{col}\PY{p}{)}
              \PY{n}{plt}\PY{o}{.}\PY{n}{grid}\PY{p}{(}\PY{k+kc}{False}\PY{p}{)}
              \PY{n}{plt}\PY{o}{.}\PY{n}{show}\PY{p}{(}\PY{p}{)}
              
          \PY{k}{for} \PY{n}{col} \PY{o+ow}{in} \PY{n}{dfStationSaxePM2\PYZus{}5}\PY{o}{.}\PY{n}{columns}\PY{p}{:}
              \PY{n}{plot\PYZus{}df}\PY{p}{(}\PY{n}{col}\PY{p}{)}
              
\end{Verbatim}


    \begin{center}
    \adjustimage{max size={0.9\linewidth}{0.9\paperheight}}{output_80_0.png}
    \end{center}
    { \hspace*{\fill} \\}
    
    \begin{center}
    \adjustimage{max size={0.9\linewidth}{0.9\paperheight}}{output_80_1.png}
    \end{center}
    { \hspace*{\fill} \\}
    
    \begin{Verbatim}[commandchars=\\\{\}]
{\color{incolor}In [{\color{incolor}763}]:} \PY{c+c1}{\PYZsh{}Dans le cas du PM10}
          \PY{c+c1}{\PYZsh{}in case of PM10}
          \PY{k}{def} \PY{n+nf}{plot\PYZus{}df}\PY{p}{(}\PY{n}{col}\PY{p}{)}\PY{p}{:}
              \PY{n}{plt}\PY{o}{.}\PY{n}{figure}\PY{p}{(}\PY{n}{figsize}\PY{o}{=}\PY{p}{(}\PY{l+m+mi}{17}\PY{p}{,} \PY{l+m+mi}{8}\PY{p}{)}\PY{p}{)}
              \PY{n}{plt}\PY{o}{.}\PY{n}{plot}\PY{p}{(}\PY{n}{dfStationSaxe\PYZus{}PM10}\PY{p}{[}\PY{n}{col}\PY{p}{]}\PY{p}{)}
              \PY{n}{plt}\PY{o}{.}\PY{n}{xlabel}\PY{p}{(}\PY{l+s+s1}{\PYZsq{}}\PY{l+s+s1}{Time}\PY{l+s+s1}{\PYZsq{}}\PY{p}{)}
              \PY{n}{plt}\PY{o}{.}\PY{n}{ylabel}\PY{p}{(}\PY{n}{col}\PY{p}{)}
              \PY{n}{plt}\PY{o}{.}\PY{n}{grid}\PY{p}{(}\PY{k+kc}{False}\PY{p}{)}
              \PY{n}{plt}\PY{o}{.}\PY{n}{show}\PY{p}{(}\PY{p}{)}
              
          \PY{k}{for} \PY{n}{col} \PY{o+ow}{in} \PY{n}{dfStationSaxe\PYZus{}PM10}\PY{o}{.}\PY{n}{columns}\PY{p}{:}
              \PY{n}{plot\PYZus{}df}\PY{p}{(}\PY{n}{col}\PY{p}{)}
              
\end{Verbatim}


    \begin{center}
    \adjustimage{max size={0.9\linewidth}{0.9\paperheight}}{output_81_0.png}
    \end{center}
    { \hspace*{\fill} \\}
    
    \begin{center}
    \adjustimage{max size={0.9\linewidth}{0.9\paperheight}}{output_81_1.png}
    \end{center}
    { \hspace*{\fill} \\}
    
    \begin{Verbatim}[commandchars=\\\{\}]
{\color{incolor}In [{\color{incolor}764}]:} \PY{n}{dfStationSaxe\PYZus{}PM10}\PY{o}{.}\PY{n}{columns}
          \PY{c+c1}{\PYZsh{}dfStationSaxePM2\PYZus{}5.columns}
\end{Verbatim}


\begin{Verbatim}[commandchars=\\\{\}]
{\color{outcolor}Out[{\color{outcolor}764}]:} Index(['PM10', 'Lyon Gerland (source Atmo-Auvergne-Rhône-Alpes)'], dtype='object')
\end{Verbatim}
            
    \begin{Verbatim}[commandchars=\\\{\}]
{\color{incolor}In [{\color{incolor}765}]:} \PY{c+c1}{\PYZsh{}on a besoin que qu\PYZsq{}une column , ma cible }
          \PY{c+c1}{\PYZsh{}We need juste one column (my target)}
          \PY{k}{del} \PY{n}{dfStationSaxePM2\PYZus{}5}\PY{p}{[}\PY{l+s+s1}{\PYZsq{}}\PY{l+s+s1}{Lyon Centre}\PY{l+s+s1}{\PYZsq{}}\PY{p}{]}
          \PY{k}{del} \PY{n}{dfStationSaxe\PYZus{}PM10}\PY{p}{[}\PY{l+s+s1}{\PYZsq{}}\PY{l+s+s1}{Lyon Gerland (source Atmo\PYZhy{}Auvergne\PYZhy{}Rhône\PYZhy{}Alpes)}\PY{l+s+s1}{\PYZsq{}}\PY{p}{]}
\end{Verbatim}


    \begin{Verbatim}[commandchars=\\\{\}]
{\color{incolor}In [{\color{incolor}766}]:} \PY{n}{dfStationSaxePM2\PYZus{}5}\PY{o}{.}\PY{n}{head}\PY{p}{(}\PY{p}{)}
\end{Verbatim}


\begin{Verbatim}[commandchars=\\\{\}]
{\color{outcolor}Out[{\color{outcolor}766}]:}                      Moyenne horaire PM2.5 en µg/m3
          Date                                               
          2017-06-20 03:00:00                            15.8
          2017-06-20 04:00:00                            29.1
          2017-06-20 05:00:00                            17.7
          2017-06-20 06:00:00                            18.7
          2017-06-20 07:00:00                            28.0
\end{Verbatim}
            
    \begin{Verbatim}[commandchars=\\\{\}]
{\color{incolor}In [{\color{incolor}767}]:} \PY{n}{dfStationSaxe\PYZus{}PM10}\PY{o}{.}\PY{n}{head}\PY{p}{(}\PY{p}{)}
\end{Verbatim}


\begin{Verbatim}[commandchars=\\\{\}]
{\color{outcolor}Out[{\color{outcolor}767}]:}                       PM10
          Date                      
          2017-06-13 06:00:00    8.3
          2017-06-13 07:00:00   51.9
          2017-06-13 08:00:00  116.4
          2017-06-13 09:00:00   67.0
          2017-06-13 10:00:00    0.0
\end{Verbatim}
            
    \hypertarget{modelisation}{%
\subsection{Modelisation}\label{modelisation}}

    Si vous avez pas prophet (c'est ce que je vais utiliser) pour le
télécharger faites ceci : If you have not prohphet (is what i will use),
do this :

\begin{verbatim}
                      anaconda: conda install gcc
                                puis (next)
                        conda install -c conda-forge fbprophet
\end{verbatim}

    \begin{Verbatim}[commandchars=\\\{\}]
{\color{incolor}In [{\color{incolor}768}]:} \PY{k+kn}{from} \PY{n+nn}{fbprophet} \PY{k}{import} \PY{n}{Prophet}
          \PY{k+kn}{import} \PY{n+nn}{logging}
          
          \PY{c+c1}{\PYZsh{}enleve les error le truc rouge}
          \PY{c+c1}{\PYZsh{}remove red error}
          \PY{c+c1}{\PYZsh{}logging.getLogger().setLevel(logging.ERROR)}
\end{Verbatim}


    \begin{Verbatim}[commandchars=\\\{\}]
{\color{incolor}In [{\color{incolor}769}]:} \PY{c+c1}{\PYZsh{}Ensuite, Prophet exige que la colonne Date soit}
          \PY{c+c1}{\PYZsh{}nommée ds et que la colonne des résultats soit nommée y:}
          \PY{c+c1}{\PYZsh{}\PYZsh{}Next, require you your column of data named ds and for result y}
          
          \PY{n}{dfStationPM2\PYZus{}5} \PY{o}{=} \PY{n}{dfStationSaxePM2\PYZus{}5}\PY{o}{.}\PY{n}{reset\PYZus{}index}\PY{p}{(}\PY{p}{)}
          \PY{n}{dfStationPM2\PYZus{}5}\PY{o}{.}\PY{n}{columns} \PY{o}{=} \PY{p}{[}\PY{l+s+s1}{\PYZsq{}}\PY{l+s+s1}{ds}\PY{l+s+s1}{\PYZsq{}}\PY{p}{,} \PY{l+s+s1}{\PYZsq{}}\PY{l+s+s1}{y}\PY{l+s+s1}{\PYZsq{}}\PY{p}{]}
          \PY{n}{dfStationPM2\PYZus{}5}\PY{o}{.}\PY{n}{info}\PY{p}{(}\PY{p}{)}
          \PY{n}{dfStationPM2\PYZus{}5}\PY{o}{.}\PY{n}{head}\PY{p}{(}\PY{p}{)}
\end{Verbatim}


    \begin{Verbatim}[commandchars=\\\{\}]
<class 'pandas.core.frame.DataFrame'>
RangeIndex: 167 entries, 0 to 166
Data columns (total 2 columns):
ds    167 non-null datetime64[ns]
y     167 non-null float64
dtypes: datetime64[ns](1), float64(1)
memory usage: 2.7 KB

    \end{Verbatim}

\begin{Verbatim}[commandchars=\\\{\}]
{\color{outcolor}Out[{\color{outcolor}769}]:}                    ds     y
          0 2017-06-20 03:00:00  15.8
          1 2017-06-20 04:00:00  29.1
          2 2017-06-20 05:00:00  17.7
          3 2017-06-20 06:00:00  18.7
          4 2017-06-20 07:00:00  28.0
\end{Verbatim}
            
    \begin{Verbatim}[commandchars=\\\{\}]
{\color{incolor}In [{\color{incolor}770}]:} \PY{n}{dfStationPM10} \PY{o}{=} \PY{n}{dfStationSaxe\PYZus{}PM10}\PY{o}{.}\PY{n}{reset\PYZus{}index}\PY{p}{(}\PY{p}{)}
          \PY{n}{dfStationPM10}\PY{o}{.}\PY{n}{columns} \PY{o}{=} \PY{p}{[}\PY{l+s+s1}{\PYZsq{}}\PY{l+s+s1}{ds}\PY{l+s+s1}{\PYZsq{}}\PY{p}{,} \PY{l+s+s1}{\PYZsq{}}\PY{l+s+s1}{y}\PY{l+s+s1}{\PYZsq{}}\PY{p}{]}
          \PY{n}{dfStationPM10}\PY{o}{.}\PY{n}{info}\PY{p}{(}\PY{p}{)}
          \PY{n}{dfStationPM10}\PY{o}{.}\PY{n}{head}\PY{p}{(}\PY{p}{)}
\end{Verbatim}


    \begin{Verbatim}[commandchars=\\\{\}]
<class 'pandas.core.frame.DataFrame'>
RangeIndex: 164 entries, 0 to 163
Data columns (total 2 columns):
ds    164 non-null datetime64[ns]
y     164 non-null float64
dtypes: datetime64[ns](1), float64(1)
memory usage: 2.6 KB

    \end{Verbatim}

\begin{Verbatim}[commandchars=\\\{\}]
{\color{outcolor}Out[{\color{outcolor}770}]:}                    ds      y
          0 2017-06-13 06:00:00    8.3
          1 2017-06-13 07:00:00   51.9
          2 2017-06-13 08:00:00  116.4
          3 2017-06-13 09:00:00   67.0
          4 2017-06-13 10:00:00    0.0
\end{Verbatim}
            
    \begin{Verbatim}[commandchars=\\\{\}]
{\color{incolor}In [{\color{incolor}792}]:} \PY{c+c1}{\PYZsh{}Ensuite, nous définissons un dataset de training. }
          \PY{c+c1}{\PYZsh{}Pour cela, nous utiliserons les 50 dernières entrées à des fins de prédiction et de validation.}
          \PY{c+c1}{\PYZsh{}After we defined an training dataset}
          \PY{c+c1}{\PYZsh{}For that, we will use the 160 last in at the end for prediction and validation}
          
          \PY{n}{prediction\PYZus{}size} \PY{o}{=} \PY{l+m+mi}{28}
          \PY{n}{train\PYZus{}dfStationPM2\PYZus{}5} \PY{o}{=} \PY{n}{dfStationPM2\PYZus{}5}\PY{p}{[}\PY{p}{:}\PY{o}{\PYZhy{}}\PY{n}{prediction\PYZus{}size}\PY{p}{]}
\end{Verbatim}


    \begin{Verbatim}[commandchars=\\\{\}]
{\color{incolor}In [{\color{incolor}849}]:} \PY{n}{prediction\PYZus{}size2} \PY{o}{=} \PY{l+m+mi}{30}
          \PY{n}{train\PYZus{}dfStationPM10} \PY{o}{=} \PY{n}{dfStationPM10}\PY{p}{[}\PY{p}{:}\PY{o}{\PYZhy{}}\PY{n}{prediction\PYZus{}size2}\PY{p}{]}
\end{Verbatim}


    \begin{Verbatim}[commandchars=\\\{\}]
{\color{incolor}In [{\color{incolor}850}]:} \PY{c+c1}{\PYZsh{}Ensuite, nous initialisons simplement Prophet, }
          \PY{c+c1}{\PYZsh{}ajustons le modèle aux données et commençons à faire des prédictions!}
          
          \PY{n}{m} \PY{o}{=} \PY{n}{Prophet}\PY{p}{(}\PY{p}{)}
          
          \PY{n}{m}\PY{o}{.}\PY{n}{fit}\PY{p}{(}\PY{n}{train\PYZus{}dfStationPM2\PYZus{}5}\PY{p}{)}
          
          \PY{n}{future} \PY{o}{=} \PY{n}{m}\PY{o}{.}\PY{n}{make\PYZus{}future\PYZus{}dataframe}\PY{p}{(}\PY{n}{periods}\PY{o}{=}\PY{n}{prediction\PYZus{}size}\PY{p}{)}
          \PY{n}{forecast} \PY{o}{=} \PY{n}{m}\PY{o}{.}\PY{n}{predict}\PY{p}{(}\PY{n}{future}\PY{p}{)}
          \PY{n}{forecast}\PY{o}{.}\PY{n}{head}\PY{p}{(}\PY{p}{)}
\end{Verbatim}


    \begin{Verbatim}[commandchars=\\\{\}]
INFO:fbprophet:Disabling yearly seasonality. Run prophet with yearly\_seasonality=True to override this.
INFO:fbprophet:Disabling weekly seasonality. Run prophet with weekly\_seasonality=True to override this.

    \end{Verbatim}

\begin{Verbatim}[commandchars=\\\{\}]
{\color{outcolor}Out[{\color{outcolor}850}]:}                    ds      trend  yhat\_lower  yhat\_upper  trend\_lower  \textbackslash{}
          0 2017-06-20 03:00:00  57.471511   17.469957   49.348317    57.471511   
          1 2017-06-20 04:00:00  57.867778    2.551926   36.869406    57.867778   
          2 2017-06-20 05:00:00  58.264045   -3.644959   29.655108    58.264045   
          3 2017-06-20 06:00:00  58.660312    0.565262   34.281879    58.660312   
          4 2017-06-20 07:00:00  59.056579   19.335748   54.101177    59.056579   
          
             trend\_upper  additive\_terms  additive\_terms\_lower  additive\_terms\_upper  \textbackslash{}
          0    57.471511      -23.812109            -23.812109            -23.812109   
          1    57.867778      -38.329106            -38.329106            -38.329106   
          2    58.264045      -45.501083            -45.501083            -45.501083   
          3    58.660312      -39.411188            -39.411188            -39.411188   
          4    59.056579      -22.594546            -22.594546            -22.594546   
          
                 daily  daily\_lower  daily\_upper  multiplicative\_terms  \textbackslash{}
          0 -23.812109   -23.812109   -23.812109                   0.0   
          1 -38.329106   -38.329106   -38.329106                   0.0   
          2 -45.501083   -45.501083   -45.501083                   0.0   
          3 -39.411188   -39.411188   -39.411188                   0.0   
          4 -22.594546   -22.594546   -22.594546                   0.0   
          
             multiplicative\_terms\_lower  multiplicative\_terms\_upper       yhat  
          0                         0.0                         0.0  33.659402  
          1                         0.0                         0.0  19.538672  
          2                         0.0                         0.0  12.762962  
          3                         0.0                         0.0  19.249123  
          4                         0.0                         0.0  36.462032  
\end{Verbatim}
            
    \begin{Verbatim}[commandchars=\\\{\}]
{\color{incolor}In [{\color{incolor}851}]:} \PY{n}{m2} \PY{o}{=} \PY{n}{Prophet}\PY{p}{(}\PY{p}{)}
          
          \PY{n}{m2}\PY{o}{.}\PY{n}{fit}\PY{p}{(}\PY{n}{train\PYZus{}dfStationPM10}\PY{p}{)}
          
          \PY{n}{future2} \PY{o}{=} \PY{n}{m2}\PY{o}{.}\PY{n}{make\PYZus{}future\PYZus{}dataframe}\PY{p}{(}\PY{n}{periods}\PY{o}{=}\PY{n}{prediction\PYZus{}size2}\PY{p}{)}
          \PY{n}{forecast2} \PY{o}{=} \PY{n}{m2}\PY{o}{.}\PY{n}{predict}\PY{p}{(}\PY{n}{future2}\PY{p}{)}
          \PY{n}{forecast2}\PY{o}{.}\PY{n}{head}\PY{p}{(}\PY{p}{)}
\end{Verbatim}


    \begin{Verbatim}[commandchars=\\\{\}]
INFO:fbprophet:Disabling yearly seasonality. Run prophet with yearly\_seasonality=True to override this.
INFO:fbprophet:Disabling weekly seasonality. Run prophet with weekly\_seasonality=True to override this.

    \end{Verbatim}

\begin{Verbatim}[commandchars=\\\{\}]
{\color{outcolor}Out[{\color{outcolor}851}]:}                    ds       trend  yhat\_lower  yhat\_upper  trend\_lower  \textbackslash{}
          0 2017-06-13 06:00:00  103.234019   -7.833990   77.107077   103.234019   
          1 2017-06-13 07:00:00  103.812949   23.406598  109.531184   103.812949   
          2 2017-06-13 08:00:00  104.391880   54.671973  138.507745   104.391880   
          3 2017-06-13 09:00:00  104.970810   71.848144  157.544195   104.970810   
          4 2017-06-13 10:00:00  105.549740   83.352582  168.387099   105.549740   
          
             trend\_upper  additive\_terms  additive\_terms\_lower  additive\_terms\_upper  \textbackslash{}
          0   103.234019      -68.867400            -68.867400            -68.867400   
          1   103.812949      -38.559782            -38.559782            -38.559782   
          2   104.391880       -7.695599             -7.695599             -7.695599   
          3   104.970810       11.649394             11.649394             11.649394   
          4   105.549740       17.642914             17.642914             17.642914   
          
                 daily  daily\_lower  daily\_upper  multiplicative\_terms  \textbackslash{}
          0 -68.867400   -68.867400   -68.867400                   0.0   
          1 -38.559782   -38.559782   -38.559782                   0.0   
          2  -7.695599    -7.695599    -7.695599                   0.0   
          3  11.649394    11.649394    11.649394                   0.0   
          4  17.642914    17.642914    17.642914                   0.0   
          
             multiplicative\_terms\_lower  multiplicative\_terms\_upper        yhat  
          0                         0.0                         0.0   34.366619  
          1                         0.0                         0.0   65.253167  
          2                         0.0                         0.0   96.696281  
          3                         0.0                         0.0  116.620203  
          4                         0.0                         0.0  123.192654  
\end{Verbatim}
            
    Nickel! Ici, yhat représente la prédiction, tandis que yhat\_lower et
yhat\_upper représentent respectivement la limite inférieure et
supérieure de cette prédiction.

Voila! Here, yhat represents the prediction, while yhat\_lower and
yhat\_upper represent the lower and upper limits of this prediction
respectively.

    \begin{Verbatim}[commandchars=\\\{\}]
{\color{incolor}In [{\color{incolor}852}]:} \PY{n}{m}\PY{o}{.}\PY{n}{plot}\PY{p}{(}\PY{n}{forecast}\PY{p}{)}
\end{Verbatim}

\texttt{\color{outcolor}Out[{\color{outcolor}852}]:}
    
    \begin{center}
    \adjustimage{max size={0.9\linewidth}{0.9\paperheight}}{output_96_0.png}
    \end{center}
    { \hspace*{\fill} \\}
    

    \begin{center}
    \adjustimage{max size={0.9\linewidth}{0.9\paperheight}}{output_96_1.png}
    \end{center}
    { \hspace*{\fill} \\}
    
    \begin{Verbatim}[commandchars=\\\{\}]
{\color{incolor}In [{\color{incolor}853}]:} \PY{n}{m2}\PY{o}{.}\PY{n}{plot}\PY{p}{(}\PY{n}{forecast}\PY{p}{)}
\end{Verbatim}

\texttt{\color{outcolor}Out[{\color{outcolor}853}]:}
    
    \begin{center}
    \adjustimage{max size={0.9\linewidth}{0.9\paperheight}}{output_97_0.png}
    \end{center}
    { \hspace*{\fill} \\}
    

    \begin{center}
    \adjustimage{max size={0.9\linewidth}{0.9\paperheight}}{output_97_1.png}
    \end{center}
    { \hspace*{\fill} \\}
    
    Comme vous pouvez le constater, Prophet a simplement utilisé une ligne
droite pour prédire la concentration future de PM2.5 et c'est le même
car pour le PM10

As you can see, Prophet simply used a straight line to predict the
future concentration of PM2.5 and it is the same because for PM10

    \begin{Verbatim}[commandchars=\\\{\}]
{\color{incolor}In [{\color{incolor}854}]:} \PY{n}{m}\PY{o}{.}\PY{n}{plot\PYZus{}components}\PY{p}{(}\PY{n}{forecast}\PY{p}{)}
\end{Verbatim}

\texttt{\color{outcolor}Out[{\color{outcolor}854}]:}
    
    \begin{center}
    \adjustimage{max size={0.9\linewidth}{0.9\paperheight}}{output_99_0.png}
    \end{center}
    { \hspace*{\fill} \\}
    

    \begin{center}
    \adjustimage{max size={0.9\linewidth}{0.9\paperheight}}{output_99_1.png}
    \end{center}
    { \hspace*{\fill} \\}
    
    \begin{Verbatim}[commandchars=\\\{\}]
{\color{incolor}In [{\color{incolor}855}]:} \PY{n}{m2}\PY{o}{.}\PY{n}{plot\PYZus{}components}\PY{p}{(}\PY{n}{forecast2}\PY{p}{)}
\end{Verbatim}

\texttt{\color{outcolor}Out[{\color{outcolor}855}]:}
    
    \begin{center}
    \adjustimage{max size={0.9\linewidth}{0.9\paperheight}}{output_100_0.png}
    \end{center}
    { \hspace*{\fill} \\}
    

    \begin{center}
    \adjustimage{max size={0.9\linewidth}{0.9\paperheight}}{output_100_1.png}
    \end{center}
    { \hspace*{\fill} \\}
    
    Ici, Prophet a seulement identifié une tendance à la hausse sans aucune
saisonnalité.

Maintenant, évaluons les performances du modèle en calculant son erreur
moyenne en pourcentage absolu (Mean Absolute Percentage Error -- MAPE)
et son erreur moyenne absolue (Mean Absolute Error -- MAE):

Here, Prophet has only identified an upward trend without any
seasonality.

Now, let's evaluate the model's performance by calculating its Mean
Absolute Percentage Error (MAPE) and Mean Absolute Error (MAE):

    \begin{Verbatim}[commandchars=\\\{\}]
{\color{incolor}In [{\color{incolor}856}]:} \PY{c+c1}{\PYZsh{} Définit une fonction qui crée une DataFrame contenant les prédictions et valeurs actuelles}
          \PY{c+c1}{\PYZsh{} Defines a function that creates a DataFrame containing current predictions and values}
          \PY{k}{def} \PY{n+nf}{make\PYZus{}comparison\PYZus{}dataframe}\PY{p}{(}\PY{n}{historical}\PY{p}{,} \PY{n}{forecast}\PY{p}{)}\PY{p}{:}
              \PY{k}{return} \PY{n}{forecast}\PY{o}{.}\PY{n}{set\PYZus{}index}\PY{p}{(}\PY{l+s+s1}{\PYZsq{}}\PY{l+s+s1}{ds}\PY{l+s+s1}{\PYZsq{}}\PY{p}{)}\PY{p}{[}\PY{p}{[}\PY{l+s+s1}{\PYZsq{}}\PY{l+s+s1}{yhat}\PY{l+s+s1}{\PYZsq{}}\PY{p}{,} \PY{l+s+s1}{\PYZsq{}}\PY{l+s+s1}{yhat\PYZus{}lower}\PY{l+s+s1}{\PYZsq{}}\PY{p}{,} \PY{l+s+s1}{\PYZsq{}}\PY{l+s+s1}{yhat\PYZus{}upper}\PY{l+s+s1}{\PYZsq{}}\PY{p}{]}\PY{p}{]}\PY{o}{.}\PY{n}{join}\PY{p}{(}\PY{n}{historical}\PY{o}{.}\PY{n}{set\PYZus{}index}\PY{p}{(}\PY{l+s+s1}{\PYZsq{}}\PY{l+s+s1}{ds}\PY{l+s+s1}{\PYZsq{}}\PY{p}{)}\PY{p}{)}
          
          \PY{n}{cmp\PYZus{}dfStationPM2\PYZus{}5} \PY{o}{=} \PY{n}{make\PYZus{}comparison\PYZus{}dataframe}\PY{p}{(}\PY{n}{dfStationPM2\PYZus{}5}\PY{p}{,} \PY{n}{forecast}\PY{p}{)}
\end{Verbatim}


    \begin{Verbatim}[commandchars=\\\{\}]
{\color{incolor}In [{\color{incolor}857}]:} \PY{k}{def} \PY{n+nf}{make\PYZus{}comparison\PYZus{}dataframe}\PY{p}{(}\PY{n}{historical2}\PY{p}{,} \PY{n}{forecast2}\PY{p}{)}\PY{p}{:}
              \PY{k}{return} \PY{n}{forecast2}\PY{o}{.}\PY{n}{set\PYZus{}index}\PY{p}{(}\PY{l+s+s1}{\PYZsq{}}\PY{l+s+s1}{ds}\PY{l+s+s1}{\PYZsq{}}\PY{p}{)}\PY{p}{[}\PY{p}{[}\PY{l+s+s1}{\PYZsq{}}\PY{l+s+s1}{yhat}\PY{l+s+s1}{\PYZsq{}}\PY{p}{,} \PY{l+s+s1}{\PYZsq{}}\PY{l+s+s1}{yhat\PYZus{}lower}\PY{l+s+s1}{\PYZsq{}}\PY{p}{,} \PY{l+s+s1}{\PYZsq{}}\PY{l+s+s1}{yhat\PYZus{}upper}\PY{l+s+s1}{\PYZsq{}}\PY{p}{]}\PY{p}{]}\PY{o}{.}\PY{n}{join}\PY{p}{(}\PY{n}{historical2}\PY{o}{.}\PY{n}{set\PYZus{}index}\PY{p}{(}\PY{l+s+s1}{\PYZsq{}}\PY{l+s+s1}{ds}\PY{l+s+s1}{\PYZsq{}}\PY{p}{)}\PY{p}{)}
          
          \PY{n}{cmp\PYZus{}dfStationPM10} \PY{o}{=} \PY{n}{make\PYZus{}comparison\PYZus{}dataframe}\PY{p}{(}\PY{n}{dfStationPM10}\PY{p}{,} \PY{n}{forecast2}\PY{p}{)}
\end{Verbatim}


    \begin{Verbatim}[commandchars=\\\{\}]
{\color{incolor}In [{\color{incolor}858}]:} \PY{c+c1}{\PYZsh{} Définit une fonction qui calcule MAPE et MAE}
          \PY{c+c1}{\PYZsh{} Defines a function who calcul MAPE and MAE}
          \PY{k}{def} \PY{n+nf}{calculate\PYZus{}forecast\PYZus{}errors}\PY{p}{(}\PY{n}{dfStationPM2\PYZus{}5}\PY{p}{,} \PY{n}{prediction\PYZus{}size}\PY{p}{)}\PY{p}{:}
              
              \PY{n}{dfStationPM2\PYZus{}5} \PY{o}{=} \PY{n}{dfStationPM2\PYZus{}5}\PY{o}{.}\PY{n}{copy}\PY{p}{(}\PY{p}{)}
              
              \PY{n}{dfStationPM2\PYZus{}5}\PY{p}{[}\PY{l+s+s1}{\PYZsq{}}\PY{l+s+s1}{e}\PY{l+s+s1}{\PYZsq{}}\PY{p}{]} \PY{o}{=} \PY{n}{dfStationPM2\PYZus{}5}\PY{p}{[}\PY{l+s+s1}{\PYZsq{}}\PY{l+s+s1}{y}\PY{l+s+s1}{\PYZsq{}}\PY{p}{]} \PY{o}{\PYZhy{}} \PY{n}{dfStationPM2\PYZus{}5}\PY{p}{[}\PY{l+s+s1}{\PYZsq{}}\PY{l+s+s1}{yhat}\PY{l+s+s1}{\PYZsq{}}\PY{p}{]}
              \PY{n}{dfStationPM2\PYZus{}5}\PY{p}{[}\PY{l+s+s1}{\PYZsq{}}\PY{l+s+s1}{p}\PY{l+s+s1}{\PYZsq{}}\PY{p}{]} \PY{o}{=} \PY{l+m+mi}{100} \PY{o}{*} \PY{n}{dfStationPM2\PYZus{}5}\PY{p}{[}\PY{l+s+s1}{\PYZsq{}}\PY{l+s+s1}{e}\PY{l+s+s1}{\PYZsq{}}\PY{p}{]} \PY{o}{/} \PY{n}{dfStationPM2\PYZus{}5}\PY{p}{[}\PY{l+s+s1}{\PYZsq{}}\PY{l+s+s1}{y}\PY{l+s+s1}{\PYZsq{}}\PY{p}{]}
              
              \PY{n}{predicted\PYZus{}part} \PY{o}{=} \PY{n}{dfStationPM2\PYZus{}5}\PY{p}{[}\PY{o}{\PYZhy{}}\PY{n}{prediction\PYZus{}size}\PY{p}{:}\PY{p}{]}
              
              \PY{n}{error\PYZus{}mean} \PY{o}{=} \PY{k}{lambda} \PY{n}{error\PYZus{}name}\PY{p}{:} \PY{n}{np}\PY{o}{.}\PY{n}{mean}\PY{p}{(}\PY{n}{np}\PY{o}{.}\PY{n}{abs}\PY{p}{(}\PY{n}{predicted\PYZus{}part}\PY{p}{[}\PY{n}{error\PYZus{}name}\PY{p}{]}\PY{p}{)}\PY{p}{)}
              
              \PY{k}{return} \PY{p}{\PYZob{}}\PY{l+s+s1}{\PYZsq{}}\PY{l+s+s1}{MAPE}\PY{l+s+s1}{\PYZsq{}}\PY{p}{:} \PY{n}{error\PYZus{}mean}\PY{p}{(}\PY{l+s+s1}{\PYZsq{}}\PY{l+s+s1}{p}\PY{l+s+s1}{\PYZsq{}}\PY{p}{)}\PY{p}{,} \PY{l+s+s1}{\PYZsq{}}\PY{l+s+s1}{MAE}\PY{l+s+s1}{\PYZsq{}}\PY{p}{:} \PY{n}{error\PYZus{}mean}\PY{p}{(}\PY{l+s+s1}{\PYZsq{}}\PY{l+s+s1}{e}\PY{l+s+s1}{\PYZsq{}}\PY{p}{)}\PY{p}{\PYZcb{}}
\end{Verbatim}


    \begin{Verbatim}[commandchars=\\\{\}]
{\color{incolor}In [{\color{incolor}859}]:} \PY{k}{def} \PY{n+nf}{calculate\PYZus{}forecast\PYZus{}errors2}\PY{p}{(}\PY{n}{dfStationPM10}\PY{p}{,} \PY{n}{prediction\PYZus{}size2}\PY{p}{)}\PY{p}{:}
              
              \PY{n}{dfStationPM10} \PY{o}{=} \PY{n}{dfStationPM10}\PY{o}{.}\PY{n}{copy}\PY{p}{(}\PY{p}{)}
              
              \PY{n}{dfStationPM10}\PY{p}{[}\PY{l+s+s1}{\PYZsq{}}\PY{l+s+s1}{e}\PY{l+s+s1}{\PYZsq{}}\PY{p}{]} \PY{o}{=} \PY{n}{dfStationPM10}\PY{p}{[}\PY{l+s+s1}{\PYZsq{}}\PY{l+s+s1}{y}\PY{l+s+s1}{\PYZsq{}}\PY{p}{]} \PY{o}{\PYZhy{}} \PY{n}{dfStationPM10}\PY{p}{[}\PY{l+s+s1}{\PYZsq{}}\PY{l+s+s1}{yhat}\PY{l+s+s1}{\PYZsq{}}\PY{p}{]}
              \PY{n}{dfStationPM10}\PY{p}{[}\PY{l+s+s1}{\PYZsq{}}\PY{l+s+s1}{p}\PY{l+s+s1}{\PYZsq{}}\PY{p}{]} \PY{o}{=} \PY{l+m+mi}{100} \PY{o}{*} \PY{n}{dfStationPM10}\PY{p}{[}\PY{l+s+s1}{\PYZsq{}}\PY{l+s+s1}{e}\PY{l+s+s1}{\PYZsq{}}\PY{p}{]} \PY{o}{/} \PY{n}{dfStationPM10}\PY{p}{[}\PY{l+s+s1}{\PYZsq{}}\PY{l+s+s1}{y}\PY{l+s+s1}{\PYZsq{}}\PY{p}{]}
              
              \PY{n}{predicted\PYZus{}part2} \PY{o}{=} \PY{n}{dfStationPM10}\PY{p}{[}\PY{o}{\PYZhy{}}\PY{n}{prediction\PYZus{}size2}\PY{p}{:}\PY{p}{]}
              
              \PY{n}{error\PYZus{}mean} \PY{o}{=} \PY{k}{lambda} \PY{n}{error\PYZus{}name}\PY{p}{:} \PY{n}{np}\PY{o}{.}\PY{n}{mean}\PY{p}{(}\PY{n}{np}\PY{o}{.}\PY{n}{abs}\PY{p}{(}\PY{n}{predicted\PYZus{}part2}\PY{p}{[}\PY{n}{error\PYZus{}name}\PY{p}{]}\PY{p}{)}\PY{p}{)}
              
              \PY{k}{return} \PY{p}{\PYZob{}}\PY{l+s+s1}{\PYZsq{}}\PY{l+s+s1}{MAPE2}\PY{l+s+s1}{\PYZsq{}}\PY{p}{:} \PY{n}{error\PYZus{}mean}\PY{p}{(}\PY{l+s+s1}{\PYZsq{}}\PY{l+s+s1}{p}\PY{l+s+s1}{\PYZsq{}}\PY{p}{)}\PY{p}{,} \PY{l+s+s1}{\PYZsq{}}\PY{l+s+s1}{MAE2}\PY{l+s+s1}{\PYZsq{}}\PY{p}{:} \PY{n}{error\PYZus{}mean}\PY{p}{(}\PY{l+s+s1}{\PYZsq{}}\PY{l+s+s1}{e}\PY{l+s+s1}{\PYZsq{}}\PY{p}{)}\PY{p}{\PYZcb{}}
\end{Verbatim}


    \begin{Verbatim}[commandchars=\\\{\}]
{\color{incolor}In [{\color{incolor}860}]:} \PY{c+c1}{\PYZsh{} Affiche MAPE et MAE}
          \PY{c+c1}{\PYZsh{} Display MAPE et MAE}
          
          \PY{k}{for} \PY{n}{err\PYZus{}name}\PY{p}{,} \PY{n}{err\PYZus{}value} \PY{o+ow}{in} \PY{n}{calculate\PYZus{}forecast\PYZus{}errors}\PY{p}{(}\PY{n}{cmp\PYZus{}dfStationPM2\PYZus{}5}\PY{p}{,} \PY{n}{prediction\PYZus{}size}\PY{p}{)}\PY{o}{.}\PY{n}{items}\PY{p}{(}\PY{p}{)}\PY{p}{:}
              \PY{n+nb}{print}\PY{p}{(}\PY{n}{err\PYZus{}name}\PY{p}{,} \PY{n}{err\PYZus{}value}\PY{p}{)}
\end{Verbatim}


    \begin{Verbatim}[commandchars=\\\{\}]
MAPE 25.13286319954733
MAE 19.880094790841937

    \end{Verbatim}

    \begin{Verbatim}[commandchars=\\\{\}]
{\color{incolor}In [{\color{incolor}861}]:} \PY{k}{for} \PY{n}{err\PYZus{}name2}\PY{p}{,} \PY{n}{err\PYZus{}value2} \PY{o+ow}{in} \PY{n}{calculate\PYZus{}forecast\PYZus{}errors2}\PY{p}{(}\PY{n}{cmp\PYZus{}dfStationPM10}\PY{p}{,} \PY{n}{prediction\PYZus{}size2}\PY{p}{)}\PY{o}{.}\PY{n}{items}\PY{p}{(}\PY{p}{)}\PY{p}{:}
              \PY{n+nb}{print}\PY{p}{(}\PY{n}{err\PYZus{}name2}\PY{p}{,} \PY{n}{err\PYZus{}value2}\PY{p}{)}
\end{Verbatim}


    \begin{Verbatim}[commandchars=\\\{\}]
MAPE2 47.041877465796944
MAE2 109.4194069854437

    \end{Verbatim}

    Bon on n'a le MAPE est à 25,13\% et le MAE à 19,88 franchement je trouve
que c'est pas si mal! N'oubliez pas que j'ai pas du tout ajusté le
modèle. Après bon c'est vrai que si quelqu'un voit 37 pourcent d'erreur,
il peut avoir peur; mais je pense pouvoir faire mieux si j'ai le temps.

Well, we don't have MAPE is at 25.13\% and MAE at 19.88, frankly I think
it's not so bad! Remember, I didn't adjust the model at all. After all,
it's true that if someone sees 25 percent error, they may be afraid; but
I think I can do better if I have time.

    Pour le MP10 sans commentaire, 50 pourcent d'erreur c'est énorme.
Pourtant le MAE était bon .. Je manque de données, y'avais trop de vide
de base. Ok je me justifie, d'accord j'ai compris.

    \begin{Verbatim}[commandchars=\\\{\}]
{\color{incolor}In [{\color{incolor}862}]:} \PY{c+c1}{\PYZsh{}Enfin, décrivons la prévision avec ses limites supérieure et inférieure:}
          \PY{n}{plt}\PY{o}{.}\PY{n}{figure}\PY{p}{(}\PY{n}{figsize}\PY{o}{=}\PY{p}{(}\PY{l+m+mi}{17}\PY{p}{,} \PY{l+m+mi}{8}\PY{p}{)}\PY{p}{)}
          \PY{n}{plt}\PY{o}{.}\PY{n}{plot}\PY{p}{(}\PY{n}{cmp\PYZus{}dfStationPM2\PYZus{}5}\PY{p}{[}\PY{l+s+s1}{\PYZsq{}}\PY{l+s+s1}{yhat}\PY{l+s+s1}{\PYZsq{}}\PY{p}{]}\PY{p}{)}
          \PY{n}{plt}\PY{o}{.}\PY{n}{plot}\PY{p}{(}\PY{n}{cmp\PYZus{}dfStationPM2\PYZus{}5}\PY{p}{[}\PY{l+s+s1}{\PYZsq{}}\PY{l+s+s1}{yhat\PYZus{}lower}\PY{l+s+s1}{\PYZsq{}}\PY{p}{]}\PY{p}{)}
          \PY{n}{plt}\PY{o}{.}\PY{n}{plot}\PY{p}{(}\PY{n}{cmp\PYZus{}dfStationPM2\PYZus{}5}\PY{p}{[}\PY{l+s+s1}{\PYZsq{}}\PY{l+s+s1}{yhat\PYZus{}upper}\PY{l+s+s1}{\PYZsq{}}\PY{p}{]}\PY{p}{)}
          \PY{n}{plt}\PY{o}{.}\PY{n}{plot}\PY{p}{(}\PY{n}{cmp\PYZus{}dfStationPM2\PYZus{}5}\PY{p}{[}\PY{l+s+s1}{\PYZsq{}}\PY{l+s+s1}{y}\PY{l+s+s1}{\PYZsq{}}\PY{p}{]}\PY{p}{)}
          \PY{n}{plt}\PY{o}{.}\PY{n}{xlabel}\PY{p}{(}\PY{l+s+s1}{\PYZsq{}}\PY{l+s+s1}{Time}\PY{l+s+s1}{\PYZsq{}}\PY{p}{)}
          \PY{n}{plt}\PY{o}{.}\PY{n}{ylabel}\PY{p}{(}\PY{l+s+s1}{\PYZsq{}}\PY{l+s+s1}{Average PM2.5 Concentration}\PY{l+s+s1}{\PYZsq{}}\PY{p}{)}
          \PY{n}{plt}\PY{o}{.}\PY{n}{grid}\PY{p}{(}\PY{k+kc}{False}\PY{p}{)}
          \PY{n}{plt}\PY{o}{.}\PY{n}{show}\PY{p}{(}\PY{p}{)}
\end{Verbatim}


    \begin{center}
    \adjustimage{max size={0.9\linewidth}{0.9\paperheight}}{output_110_0.png}
    \end{center}
    { \hspace*{\fill} \\}
    
    On va quand l'affiché pour le PM10 par curiosité.

    \begin{Verbatim}[commandchars=\\\{\}]
{\color{incolor}In [{\color{incolor}863}]:} \PY{c+c1}{\PYZsh{}Enfin, décrivons la prévision avec ses limites supérieure et inférieure:}
          \PY{n}{plt}\PY{o}{.}\PY{n}{figure}\PY{p}{(}\PY{n}{figsize}\PY{o}{=}\PY{p}{(}\PY{l+m+mi}{17}\PY{p}{,} \PY{l+m+mi}{8}\PY{p}{)}\PY{p}{)}
          \PY{n}{plt}\PY{o}{.}\PY{n}{plot}\PY{p}{(}\PY{n}{cmp\PYZus{}dfStationPM10}\PY{p}{[}\PY{l+s+s1}{\PYZsq{}}\PY{l+s+s1}{yhat}\PY{l+s+s1}{\PYZsq{}}\PY{p}{]}\PY{p}{)}
          \PY{n}{plt}\PY{o}{.}\PY{n}{plot}\PY{p}{(}\PY{n}{cmp\PYZus{}dfStationPM10}\PY{p}{[}\PY{l+s+s1}{\PYZsq{}}\PY{l+s+s1}{yhat\PYZus{}lower}\PY{l+s+s1}{\PYZsq{}}\PY{p}{]}\PY{p}{)}
          \PY{n}{plt}\PY{o}{.}\PY{n}{plot}\PY{p}{(}\PY{n}{cmp\PYZus{}dfStationPM10}\PY{p}{[}\PY{l+s+s1}{\PYZsq{}}\PY{l+s+s1}{yhat\PYZus{}upper}\PY{l+s+s1}{\PYZsq{}}\PY{p}{]}\PY{p}{)}
          \PY{n}{plt}\PY{o}{.}\PY{n}{plot}\PY{p}{(}\PY{n}{cmp\PYZus{}dfStationPM10}\PY{p}{[}\PY{l+s+s1}{\PYZsq{}}\PY{l+s+s1}{y}\PY{l+s+s1}{\PYZsq{}}\PY{p}{]}\PY{p}{)}
          \PY{n}{plt}\PY{o}{.}\PY{n}{xlabel}\PY{p}{(}\PY{l+s+s1}{\PYZsq{}}\PY{l+s+s1}{Time}\PY{l+s+s1}{\PYZsq{}}\PY{p}{)}
          \PY{n}{plt}\PY{o}{.}\PY{n}{ylabel}\PY{p}{(}\PY{l+s+s1}{\PYZsq{}}\PY{l+s+s1}{Average PM10 Concentration}\PY{l+s+s1}{\PYZsq{}}\PY{p}{)}
          \PY{n}{plt}\PY{o}{.}\PY{n}{grid}\PY{p}{(}\PY{k+kc}{False}\PY{p}{)}
          \PY{n}{plt}\PY{o}{.}\PY{n}{show}\PY{p}{(}\PY{p}{)}
\end{Verbatim}


    \begin{center}
    \adjustimage{max size={0.9\linewidth}{0.9\paperheight}}{output_112_0.png}
    \end{center}
    { \hspace*{\fill} \\}
    
    \hypertarget{et-maintenant}{%
\subsection{Et maintenant?}\label{et-maintenant}}

\hypertarget{and-now}{%
\subsection{And Now ?}\label{and-now}}

    Ces données restent aujourd'hui insuffisantes sur la qualité de l'air au
quotidien, même si elles permettent de voir une concentration de
particules fines bien plus importantes dans le réseau métro qu'à
l'extérieur. Seules des mesures en continu ou plus longues sur
l'ensemble des lignes, ainsi que l'analyse d'autres polluants comme le
monoxyde de carbone et les oxydes d'azote, par exemple, permettraient
d'établir une cartographie précise de la qualité l'air dans le réseau
métro. Interrogé sur ces questions, le Sytral indique poursuivre les
relevés.

L'autorité mettra également en place ``une centrale de mesure en continu
des particules fines en fin 2019 sur le réseau métro TCL''. En mi-2019,
l'INERIS rendra ses analyses sur ces mesures de 2017 reproduites
ci-dessus. De même, un protocole harmonisé de mesures devrait être mis
en place en fin 2019.

Reste que comme regarder le thermomètre ne fait pas baisser la
température, lire simplement ces données ne diminuera pas la
concentration des particules fines. Dès lors quelles seront les actions
mises en place dans le réseau métro ?Personnellement en tant qu'un
pauvre petit étudiant de l'ESME SUDRIA, je ne saurais vous répondre.

Mais, selon le Sytral : ``Une étude de modélisation de la production et
de la migration des particules va également être lancée cette année, mi
2019 afin de mieux cibler les actions permettant de réduire les sources
d'émission et les zones d'accumulation'', de même, ``Les nouveaux
matériels roulants prochainement livrés réduiront significativement le
recours au freinage mécanique et donc l'émission de particules par
frottement''. L'autorité régulatrice pourrait également nouer des
partenariats avec des instituts de recherche pour trouver des solutions
permettant de réduire la concentration de particules fines.

S'il a fallu attendre 17 ans pour avoir de nouvelles mesures de la
qualité de l'air dans le métro, désormais, il semble impensable que ce
type de chiffres ne soient pas communiqués en continu d'ici 2020.

La pollution devrait être l'un des enjeux de la prochaine campagne des
municipales, et la qualité de l'air tout aussi importante qu'elle soit
respirée à l'intérieur, extérieur, ou dans le réseau métro de Lyon. La
pression citoyenne va se faire plus forte.

Des systèmes pour évaluer son exposition au quotidien commencent déjà à
apparaître, à l'image du capteur Flow de l'entreprise Plume.

Ce petit objet connecté à attacher à son sac transforme chaque citoyen
en capteur de pollution. Si les collectivités ne transmettent pas les
chiffres, les start-ups le feront à leur place.

Ce vendredi 12 avril, le Sytral annonce la mise en place d'une cellule
dédiée à la question de la qualité de l'air( enfin hein), composée
notamment de deux élus : Pierre Hémon (Europe Écologie Les Verts) et
Christophe Quiniou (Les Républicains).

Dans un premier temps, cette cellule se chargera de mettre en place des
mesures permanentes à la station Saxe Gambetta. Cette dernière reste
celle où la concentration de particules fines est la plus fortes (en
partie à cause du freinage lorsque les rames de B sont bondées, donc
plus lourdes). Dans cette station, les mesures effectuées en 2017
montraient grâce à mon étude que la concentration en particules fines
(PM10) et très fines (PM2,5) était quatre fois plus élevée par rapport à
l'air extérieur au même moment. Enfin sa c'est que j'ai dis ..

Ces futures mesures permanentes permettront également de voir quel sera
l'impact de l'automatisation de la ligne B et l'arrivée de nouvelles
rames qui pourraient contribuer à faire diminuer la concentration de
particules fines sur le secteur.

Par ailleurs, d'autres mesures seront régulièrement organisées dans le
métro pour continuer l'état des lieux (Youphi), mais aussi observer les
évolutions avec le changement de matériel et le cas échéant permettre de
prendre des décisions pour améliorer la qualité de l'air dans le métro.

Donc mon futur projet je pense serait, d'étudier ce matériel; enfin si
j'ai le temps.

    These data are still insufficient on daily air quality, even if they
show a much higher concentration of fine particles in the metro network
than outside. Only continuous or longer measurements on all lines, as
well as the analysis of other pollutants such as carbon monoxide and
nitrogen oxides, for example, would make it possible to establish an
accurate map of air quality in the metro network. When asked about these
questions, the Sytral indicated that it was continuing the surveys.

The authority will also set up ``a continuous fine particle measurement
centre by the end of 2019 on the TCL metro network''. In mid-2019,
INERIS will submit its analyses on these 2017 measures reproduced above.
Similarly, a harmonised measurement protocol should be in place by the
end of 2019.

However, since looking at the thermometer does not lower the
temperature, simply reading these data will not reduce the concentration
of fine particles. So what actions will be implemented in the metro
network? Personally, as a poor little student of ESME SUDRIA, I cannot
answer you.

But, according to Sytral: ``A study to model particle production and
migration will also be launched this year, mid-2019, in order to better
target actions to reduce emission sources and accumulation areas'', as
well as ``New rolling stock soon to be delivered will significantly
reduce the use of mechanical braking and therefore the emission of
particles by friction''. The regulatory authority could also partner
with research institutes to find solutions to reduce the concentration
of fine particles.

While it took 17 years to get new measurements of air quality in the
metro, it now seems unthinkable that this type of figure will not be
reported continuously by 2020.

Pollution should be one of the issues of the next municipal campaign,
and air quality is just as important whether it is breathed indoors,
outdoors, or in the Lyon metro network. Citizen pressure will increase.

Systems to assess daily exposure are already beginning to appear, such
as Plume's Flow sensor. This small connected object to attach to his bag
transforms each citizen into a pollution sensor.

If local authorities do not provide the figures, start-ups will do it
for them.

This Friday, April 12, the Sytral announces the establishment of a unit
dedicated to the issue of air quality (finally eh), composed of two
elected officials: Pierre Hémon (Europe Écologie Les Verts) and
Christophe Quiniou (Les Républicains).

As a first step, this cell will be responsible for setting up permanent
measures at the Saxony Gambetta station. The latter remains the one with
the highest concentration of fine particles (partly due to braking when
the B trains are crowded, and therefore heavier). At this station,
measurements made in 2017 showed that the concentration of fine (PM10)
and very fine (PM2.5) particles was four times higher than in outdoor
air at the same time. Well, that's what I said\ldots{}

These future permanent measures will also make it possible to see the
impact of the automation of Line B and the arrival of new trains that
could help reduce the concentration of fine particles in the sector.

In addition, other measures will be regularly organised in the metro to
continue the inventory of fixtures (Youphi), but also to observe
developments with the change of equipment and, if necessary, to take
decisions to improve air quality in the metro.

So my future project I think would be to study this material; well, if I
have time.

    Merci Pour cette lecture, j'èspère que vous avez aimé ma presentation.
J'ai préferés le présenter de cette manière. Comme raconter une
histoire, ma copine ``Anais''; le jour où je lui présenté l'un de mes
projet concernant les parkins de Lyon.

Vue qu'elle n'est pas du domaine, elle n'a absolument rien compris; et
cela m'a attristé même avec son master en ecole de commerce.

J'ai compris que nous faisons des travaux juste pour être compris
qu'entre nous. J'ai voulus casser ça, c'est pour cela que j'ai fait un
tel kernel, pour qu'il puisse être lu et compris par le premier venus.

Lui donner envies de lire, comme lire un livre ou un journal.

Encore merci pour votre lecture.

Thank you for reading this, I hope you enjoyed my presentation. I
preferred to present it that way. Like telling a story, my friend
``Anais''; the day I presented her with one of my projects about the
parkins of Lyon.

Since she is not from the field, she understood absolutely nothing; and
it saddened me even with her master's degree in business school.

I understood that we do work just to be understood only between us. I
wanted to break that, that's why I made such a kernel, so that it could
be read and understood by the first people who came.

Make her want to read, like reading a book or a newspaper.

Thank you again for reading.

Junholv OBO,


    % Add a bibliography block to the postdoc
    
    
    
    \end{document}
